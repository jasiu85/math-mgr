\begin{abstract}

W niniejszej pracy opisano iloczyn Weila,
algorytm obliczający jego wartości
oraz jego zastosowania w kryptografii.
Praca zawiera również
krótkie wprowadzenie do krzywych eliptycznych.

Iloczyn Weila $w_n \colon E[n] \times E[n] \to \mu_n$
jest to specyficzna operacja dwuargumentowa
prowadząca z podgrupy $n$-torsyjnej krzywej eliptycznej $E = E(\K)$
(czyli podgrupy składającej się z tych punktów $P$ krzywej $E$,
dla których zachodzi $nP = \ecident$)
w grupę $\mu_n$ pierwiastków $n$-tego stopnia z jedności w ciele $\K$.
Cechy charakterystyczne iloczynu Weila
to dwuliniowość, antysymetria oraz niezdegenerowanie.

Przedstawiony w pracy algorytm obliczający wartości iloczynu Weila
został zaproponowany przez Millera.
Algorytm ten realizuje schemat ,,podwajaj-i-dodawaj'',
który jest znany z algorytmu szybkiego podnoszenia do $n$-tej potęgi.

Zastosowania iloczynu Weila zaprezentowane w pracy obejmują
konstrukcje kryptosystemów pozwalających
na szyfrowanie i podpisywanie wiadomości oparte o tożsamość,
tzn. pozwalające na użycie dowolnego ciągu bitów (np. adresu e-mail)
jako klucza,
oraz redukcję MOV, która umożliwia przeprowadzanie ataków
na systemy kryptograficzne oparte na krzywych eliptycznych.

\end{abstract}
