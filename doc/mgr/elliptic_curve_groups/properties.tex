\section{Własności}

Przytoczymy teraz bez dowodu najważniejsze własności
grup na krzywych eliptycznych.

Przede wszystkim, ze względu na charakter wzorów występujących
w definicji działania grupowego,
ciało, nad którym krzywe rozpatrujemy, nie musi być algebraicznie domknięte.

\begin{fact}
Dana jest krzywa eliptyczna $E(\fieldL)$ oraz ciało $\K \subset \fieldL$.
Wówczas suma dwóch punktów $\K$-wymiernych jest punktem $\K$-wymiernym.
\end{fact}

Jest jasne, że w przypadku ciał algebraicznie domkniętych krzywa eliptyczna
ma nieskończenie wiele elementów.
W przypadku ciał skończonych sytuacja wygląda zupełnie inaczej.

\begin{theorem}[Hasse]
Rząd grupy na dowolnej krzywej eliptycznej $E$
nad dowolnym ciałem skończonym $\GF_q$,
gdzie $q = p^n$ oraz $p$ jest liczbą pierwszą,
spełnia następującą nierówność:
\begin{equation}
q + 1 - 2\sqrt{q} \leq \abs{E} \leq q + 1 + 2\sqrt{q}
\end{equation}
\end{theorem}

Dowód tego twierdzenia można znaleźć np. w pracy \cite{ecintro1},
zaś intuicyjne wyjaśnienie jest następujące.
Jeżeli potraktujemy wielomian charakterystyczny $\kappa$ krzywej eliptycznej
jak funkcję różnowartościową,
to wówczas dla mniej-więcej połowy elementów $a \in \GF_q$
można z wartości $\kappa(a)$ wyciągnąć pierwiastek kwadratowy.
Zatem krzywa ma około $2\frac{q}{2}$ punktów skończonych.
Doliczając punkt w nieskończoności otrzymujemy około $q + 1$ punktów.

Choć nie ma jawnego wzoru
na rząd grupy na krzywej eliptycznej nad ciałem skończonym,
następujące twierdzenie,
którego dowód można odnaleźć w książce \cite{silverman},
charakteryzuje ogólną strukturę takiej grupy.

\begin{theorem}
Grupa na krzywej eliptycznej $E$ nad ciałem $\GF_q$
jest izomorficzna z grupą $(\Z/m\Z) \times (\Z/n\Z)$,
przy czym $n \mid \gcd(m, q-1)$.
\end{theorem}

Na koniec wspomnijmy o algorytmie,
który pozwala dosyć efektywnie policzyć
rząd grupy na krzywej nad ciałem skończonym.
Został on opublikowany po raz pierwszy w pracy \cite{schoof}.

\begin{theorem}[Schoof]
Istnieje algorytm, który pozwala obliczyć rząd grupy na krzywej eliptycznej $E$
nad ciałem $\GF_q$ za pomocą $O(\log^8 q)$ operacji na bitach.
\end{theorem}
