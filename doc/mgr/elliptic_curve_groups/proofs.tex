\section{Dowody niezbędnych stwierdzeń}

Definicja grupy na krzywej eliptycznej
opiera się na wielu własnościach krzywych eliptycznych
podanych w poprzednim rozdziale.
Nie sposób dokładnie zrozumieć strukturę grupy na krzywej eliptycznej,
jeżeli nie zostaną zaprezentowane dowody najbardziej istotnych własności.
Dlatego udowodnimy teraz kluczowe lematy, twierdzenia i wnioski.

\subsection*{Obliczanie współczynników linii}

Kluczową operacją, która posłuży do zdefiniowania działania grupowego
na krzywej eliptycznej,
jest wyznaczanie współczynników
linii przechodzącej przez dwa zadane różne punkty krzywej
lub stycznej do krzywej w zadanym punkcie.
Wyprowadzimy teraz wzory na te współczynniki.

Zaczniemy wyznaczenia współczynników
linii przechodzącej przez dwa różne punkty,
czyli udowodnimy twierdzenie \ref{line_through_two_points_theorem}.

\begin{proof}[Dowód twierdzenia \ref{line_through_two_points_theorem}]
Dana jest krzywa eliptyczna $E$
oraz dwa różne punkty skończone $P$ oraz $Q$ na tej krzywej.
Oznaczmy $P = (a, b)$ oraz $Q = (c, d)$.

Jeżeli $P = \overline{Q}$,
to rozpatrujemy linię
\begin{equation*}
l(x, y) = x - a
\end{equation*}
Jak nietrudno sprawdzić, $l(P) = 0$ oraz $l(Q) = 0$.

Jeżeli $P \neq \overline{Q}$,
to rozpatrujemy linię
\begin{equation*}
l(x, y) = \left(\frac{d-b}{c-a}\right)(x - a) - (y - b)
\end{equation*}
Ponownie, $l(P) = 0$ oraz $l(Q) = 0$.
\end{proof}

Linia przecinająca krzywą w dwóch różnych punktach
na ogół przecina krzywą w jeszcze jednym punkcie.
Następujące twierdzenie mówi,
kiedy są trzy punkty przecięcia
i gdzie jest trzeci punkt.

\begin{theorem}
Dana jest krzywa eliptyczna $E$
oraz dwa różne punkty skończone $P$ oraz $Q$ na tej krzywej.
Wówczas linia $l$ przechodząca przez punkty $P$ i $Q$
ma dywizor $\rdiv(l)$ równy $\divi{P} + \divi{Q} - 2\divi{\ecident}$,
jeśli $P = \overline{Q}$
lub $\divi{P} + \divi{Q} + \divi{R} - 2\divi{\ecident}$,
gdy $P \neq \overline{Q}$.
Punkt $R$ jest skończony,
może zdarzyć się, że $R = P$ lub $R = Q$.
\end{theorem}

\begin{proof}
Oznaczmy $P = (a, b)$ oraz $Q = (a, -b)$
i rozważmy dwa przypadki.

Jeżeli $P = \overline{Q}$,
to stopień linii $l$ jest równy $2$,
zatem nie może ona mieć miejsc zerowych innych niż $P$ i $Q$,
a jej dywizor to $\divi{P} + \divi{Q} - 2\divi{\ecident}$.

Jeżeli $P \neq \overline{Q}$,
to stopień linii $l$ jest równy $3$,
zatem jej dywizor ma postać
$\divi{P} + \divi{Q} + \divi{R} - 3\divi{\ecident}$.
gdzie $R$ jest punktem skończonym krzywej,
trzecim punktem przecięcia linii z krzywą,
być może równym $P$ lub $Q$.
Oznaczmy $R = (e, f)$.
Norma $N(l)$ jest równa:
\begin{equation*}
N(l)(x) = \left(\left(\frac{d-b}{c-a}\right) + b\right)^2 - x^3 - Ax - B
\end{equation*}
Zgodnie z faktem \ref{zero_of_norm_fact}
ma ona pierwiastki $a$, $c$ oraz $e$.
Wartość pierwiastka $e$ znajdujemy ze wzoru Viete'a na sumę pierwiastków:
\begin{equation*}
a + c + e = \left(\frac{d-b}{c-a}\right)^2
\end{equation*}
Stąd łatwo wyliczamy wartość współrzędnej $e$.
Współrzędną $f$ obliczamy z równania $l(R) = 0$.
Otrzymujemy:
\begin{eqnarray}
\label{chord_line_third_point_x_eqn}
e & = & \left(\frac{d-b}{c-a}\right)^2 - a - c \\
\label{chord_line_third_point_y_eqn}
f & = & b + \left(\frac{d-b}{c-a}\right)(e-a)
\end{eqnarray}
\end{proof}

Przejdziemy teraz do linii stycznej w zadanym punkcie,
czyli udowodnimy twierdzenie \ref{line_tangent_at_point_theorem}.

\begin{proof}[Dowód twierdzenia \ref{line_tangent_at_point_theorem}]
Dana jest krzywa eliptyczna $E$
oraz punkt skończony $P$ na tej krzywej.
Oznaczmy $P = (a, b)$.

Jeżeli $P$ jest punktem podwójnym,
to rozpatrujemy linię
\begin{equation*}
l(x, y) = x - a
\end{equation*}
Ma ona stopień równy $2$,
zatem ma dywizor postaci $\divi{P} + \divi{\overline{P}} - 2\divi{\ecident}$.
Skoro $P$ jest punktem podwójnym, to dywizor ten jest w takim razie równy
$2\divi{P} - 2\divi{\ecident}$,
zatem linia $l$ jest styczna do krzywej w punkcie $P$.

Jeżeli $P$ nie jest punktem podwójnym,
to rozpatrujemy linię
\begin{equation*}
l(x, y) = \left(\frac{3a^2 + A}{2b}\right)(x - a) - (y - b)
\end{equation*}
Pokażemy, że $\ord_P(l) > 1$.
W tym celu wykażemy, że jeśli $l = ur$,
gdzie $u$ jest unifikatorem w punkcie $P$,
a $r$ jest funkcją wymierną,
to $r(P) = 0$.
Ponieważ punkt $P$ nie jest podwójny,
bierzemy unifikator $u(x, y) = x - a$.
Wyznaczamy funkcję wymierną $r$.
\begin{eqnarray*}
r
& = & \frac{l}{u} \\
& = & \frac{\left(\frac{3a^2+A}{2b}\right)(x-a) - (y-b)}{x-a} \\
& = & \frac{3a^2+A}{2b} - \frac{y-b}{x-a} \\
& = & \frac{3a^2+A}{2b} - \frac{y^2-b^2}{(x-a)(y+b)} \\
& = & \frac{3a^2+A}{2b} - \frac{x^3+Ax+B - a^3-Aa-B}{(x-a)(y+b)} \\
& = & \frac{3a^2+A}{2b} - \frac{(x-a)(x^2+ax+a^2+A)}{(x-a)(y+b)} \\
& = & \frac{3a^2+A}{2b} - \frac{x^2+ax+a^2+A}{y+b}
\end{eqnarray*}
Widzimy teraz, że $r(P) = 0$,
czyli $\ord_P(l) > 1$.
\end{proof}

Linia styczna może przeciąć krzywą w jeszcze jednym punkcie.
Następujące twierdzenie mówi,
kiedy tak się dzieje
i jakie są współrzędne tego punktu.

\begin{theorem}
Dana jest krzywa eliptyczna $E$
oraz punkt skończony $P$ na tej krzywej.
Wówczas linia $l$ styczna do krzywej w punkcie $P$
ma dywizor $\rdiv(l)$ równy $2\divi{P} - 2\divi{\ecident}$,
jeśli $P = \overline{P}$
lub $2\divi{P} + \divi{Q} - 3\divi{\ecident}$,
gdy $P \neq \overline{P}$.
Punkt $Q$ jest skończony,
może zdarzyć się, że $Q = P$.
\end{theorem}

\begin{proof}
Oznaczmy $P = (a, b)$
i rozważmy dwa przypadki.

Jeżeli $P = \overline{P}$,
to stopień linii $l$ jest równy $2$,
zatem nie może ona mieć miejsc zerowych innych niż $P$,
a jej dywizor to $2\divi{P} - 2\divi{\ecident}$.

Jeżeli $P \neq \overline{P}$,
to stopień linii $l$ jest równy $3$,
zatem jej dywizor ma postać
$2\divi{P}+ \divi{Q} - 3\divi{\ecident}$.
gdzie $Q$ jest punktem skończonym krzywej,
trzecim punktem przecięcia linii z krzywą,
być może równym $P$.
Oznaczmy $Q = (c, d)$.
Norma $N(l)$ jest równa:
\begin{equation*}
N(l)(x) = \left(\left(\frac{3a^2+A}{2b}\right) + b\right)^2 - x^3 - Ax - B
\end{equation*}
Zgodnie z faktem \ref{zero_of_norm_fact}
ma ona pierwiastki $a$ (podwójny) oraz $c$.
Wartość pierwiastka $c$ znajdujemy ze wzorów Viete'a na sumę pierwiastków:
\begin{equation*}
2a + c = \left(\frac{3a^2+A}{2b}\right)^2
\end{equation*}
Stąd łatwo wyliczamy wartość współrzędnej $c$.
Współrzędną $d$ obliczamy z równania $l(Q) = 0$.
Otrzymujemy:
\begin{eqnarray}
\label{tangent_line_third_point_x_eqn}
c & = & \left(\frac{3a^2+A}{2b}\right)^2 - 2a \\
\label{tangent_line_third_point_y_eqn}
d & = & b + \left(\frac{3a^2+A}{2b}\right)(c-a)
\end{eqnarray}
\end{proof}

Niezwykle istotny jest fakt,
że współczynniki linii przechodzącej przez dwa punkty
lub stycznej w punkcie,
jak i współrzędne dodatkowego punktu przecięcia
takiej linii z krzywą eliptyczną dają się wyrazić
jako nieskomplikowane wyrażenia wymierne.

\begin{fact}
Dana jest krzywa eliptyczna $E(\K)$
oraz linia $l$ przechodząca przez punkty $P$ i $Q$ krzywej
(odpowiednio, styczna do krzywej w punkcie $P$).
Wówczas $l = ax + by + c$, gdzie $a,b,c \in \K$
oraz $\rdiv(l) = \divi{P} + \divi{Q} + \divi{R} - 3\divi{\ecident}$
(odpowiednio, $\rdiv(l) = 2\divi{P} + \divi{R} - 3\divi{\ecident}$),
gdzie $R \in E(\K)$.
\end{fact}

\begin{fact}
Dana jest krzywa eliptyczna $E(\fieldL)$, ciało $\K \subset \fieldL$
oraz linia $l$ przechodząca przez punkty $P$ i $Q$ krzywej
(odpowiednio, styczna do krzywej w punkcie $P$).
Jeżeli punkty $P$ i $Q$ (odpowiednio, punkt $P$) są $\K$-wymierne,
to współczynniki linii $l$
oraz punkty występujące w jej dywizorze
również są $\K$-wymierne.
\end{fact}

Istota tych faktów polega na tym,
że ciało $\K$ nie musi być algebraicznie domknięte,
a mimo to zawsze da się wyznaczyć współczynniki linii
oraz dodatkowego punktu przecięcia linii z krzywą.
A priori nie było to oczywiste --
uzyskane wzory mogły być bardziej skomplikowane (np. wymagać pierwiastkowania)
i konkretne liczby niezbędne do wyrażenia współczynników i współrzędnych
znajdowałyby się dopiero w rozszerzeniu algebraicznym $\overline{\K}$
ciała $\K$.

\subsection*{Zerowa grupa Picarda}

Zbadamy teraz bardziej szczegółowo
strukturę zerowej grupy Picarda,
aby później uzasadnić
konstrukcję operacji grupowej na krzywej eliptycznej.

Rozpoczniemy od udowodnienia twierdzenia
\ref{divisor_linear_reduction_theorem}
podanego w poprzednim rozdziale.

\begin{proof}[Dowód twierdzenia \ref{divisor_linear_reduction_theorem}]
Niech $\Delta \in \Div(E)$ będzie dowolnym dywizorem.
Rozważmy składniki, z których składa się dywizor $\Delta$.
Mogą zajść następujące przypadki:
\begin{enumerate}
\item dywizor $\Delta$ zawiera dwa składniki
$a(P)\divi{P}$ oraz $a(Q)\divi{Q}$,
gdzie $P$ oraz $Q$ to takie dwa różne punkty skończone krzywej,
że współczynniki $a(P)$ oraz $a(Q)$ są niezerowe oraz tego samego znaku;
\item dywizor $\Delta$ zawiera składnik $a(P)\divi{P}$,
gdzie $P$ to taki punkt skończony krzywej, że $\abs{a(P)} > 1$;
\item dywizor $\Delta$ ma postać $\divi{P} - \divi{Q} + n\divi{\ecident}$,
gdzie $P$ oraz $Q$ to punkty skończone krzywej;
\item dywizor $\Delta$ ma postać $\pm\divi{P} + n\divi{\ecident}$
lub $n\divi{\ecident}$,
gdzie $P$ to punkt skończony krzywej.
\end{enumerate}

Pokażemy, że w każdym przypadku z wyjątkiem ostatniego
możemy wskazać dywizor $\Delta'$ taki,
że $\Delta \sim \Delta'$, $\deg(\Delta) = \deg(\Delta')$
oraz $\abs{\Delta} > \abs{\Delta'}$.
\begin{enumerate}
\item Dla ustalenia uwagi przyjmimy,
że współczynniki $a(P)$ oraz $a(Q)$ są dodatnie.
Rozważmy dywizor $\Delta' = \Delta - \rdiv(l)$,
gdzie $l$ to linia przechodząca przez punkty $P$ oraz $Q$,
ma ona dywizor $\rdiv(l)$ równy
$\divi{P} + \divi{Q} + \divi{R} - 3\divi{\ecident}$.
Może zdarzyć się, że punkt $R$ jest równy $P$, $Q$ lub $\ecident$.
Łatwo sprawdzić, że zawsze $\abs{\Delta} > \abs{\Delta'}$.
\item Dla ustalenia uwagi przyjmijmy, że współczynnik $a(P)$ jest dodatni.
Rozważmy dywizor $\Delta' = \Delta - \rdiv(l)$,
gdzie $l$ jest linią styczną do krzywej w punkcie $P$,
ma ona dywizor $\rdiv(l)$ równy $2\divi{P} + \divi{R} - 3\divi{\ecident}$.
Również w tym przypadku $\abs{\Delta} > \abs{\Delta'}$,
nawet jeśli punkt $R$ jest równy $P$ lub $\ecident$.
\item Rozważmy linię $l$ przechodzącą przez punkty $P$ oraz $\overline{P}$.
Ma ona dywizor $\rdiv(l)$ równy
$\divi{P} + \divi{\overline{P}} - 2\divi{\ecident}$.
Rozważamy dywizor
$\Delta' =
\Delta - \rdiv(l) =
-\divi{\overline{P}} - \divi{Q} + n'\divi{\ecident}$,
do którego stosujemy rozumowanie opisane w punkcie pierwszym lub drugim
(zależnie od tego, czy $\overline{P} = Q$, czy nie)
i otrzymujemy dywizor $\Delta''$, który ma normę mniejszą niż $\Delta$.
Zwróćmy uwagę, że rozumowanie w tym kroku jest poprawne również wtedy,
gdy $P$ jest punktem podwójnym --
wówczas zamiast linii przechodzącej przez punkty $P$ oraz $\overline{P}$
bierzemy linię styczną do krzywej w punkcie $P$.
\end{enumerate}

Teraz konstruujemy ciąg dywizorów
$\Delta_0, \Delta_1, \Delta_2, \ldots$ taki,
że $\Delta_0 = \Delta$,
a dywizor $\Delta_{i+1}$ otrzymujemy
poprzez zastosowanie do dywizora $\Delta_i$ jednej z opisanych redukcji.
Widać, że wszystkie dywizory w ciągu są sobie równoważne
oraz mają taki sam stopień.
Co więcej, ciąg ten jest skończony,
bo normy kolejnych dywizorów maleją,
a nie mogą tego robić w nieskończoność -- są to liczby naturalne.
Ostatni dywizor w ciągu $\Delta_k$ musi więc mieć postać
$\pm\divi{P} + n\divi{\ecident}$ lub $n\divi{\ecident}$,
gdyż w przeciwnym razie można by kontynuować redukcję.

Jeżeli $\Delta_k = \divi{P} + n\divi{\ecident}$,
to jest to szukany dywizor.
Jeżeli $\Delta_k = n\divi{\ecident}$,
to również jest to szukany dywizor --
zapisujemy go w postaci $\Delta_k = \divi{\ecident} + (n-1)\divi{\ecident}$.
Jeśli zaś $\Delta_k = -\divi{P} + n\divi{\ecident}$,
to szukanym dywizorem jest dywizor
$\Delta_k + \rdiv(l)$,
gdzie $l$ to linia przechodząca przez punkty $P$ oraz $\overline{P}$
(lub styczna do krzywej w punkcie $P$, jeżeli jest to punkt podwójny).
Linia $l$ ma dywizor $\divi{P} + \divi{\overline{P}} - 2\divi{\ecident}$,
zatem $\Delta_k + \rdiv(l) = \divi{\overline{P}} + (n-2)\divi{\ecident}$.
Jest to dywizor równoważny dywizorowi $\Delta$ oraz mający taki sam stopień.
\end{proof}

Twierdzenie \ref{zerodeg_divisor_linear_reduction_theorem}
jest teraz prostym wnioskiem.

\begin{proof}[Dowód twierdzenia \ref{zerodeg_divisor_linear_reduction_theorem}]
Niech $\Delta \in \Div^0(E)$ będzie dowolnym dywizorem o zerowym stopniu.
Zgodnie z twierdzeniem \ref{divisor_linear_reduction_theorem}
istnieje dywizor $\Delta' = \divi{P} + n\divi{\ecident}$
taki, że $\Delta \sim \Delta'$ oraz $\deg(\Delta) = \deg(\Delta')$.
Skoro dywizor $\Delta'$ również ma zerowy stopień,
to $n = -1$, zatem ma on postać $\Delta' = \divi{P} - \divi{\ecident}$.

Przypuśćmy, że istnieje drugi, różny od $P$ punkt $Q$ taki,
że $\divi{P} - \divi{\ecident} \sim \Delta \sim \divi{Q} - \divi{\ecident}$.
To oznacza, że $\divi{P} - \divi{Q} \sim 0$,
czyli $\divi{P} - \divi{Q}$ jest dywizorem głównym.
Korzystamy ponownie z twierdzenia \ref{divisor_linear_reduction_theorem},
i otrzymujemy, że $\divi{P} - \divi{Q} \sim \divi{S} - n\divi{\ecident}$.
Ponownie, $\divi{P} - \divi{Q}$ ma stopień równy zero,
zatem $n = -1$.
Stąd, $\divi{S} - \divi{\ecident} \sim \divi{P} - \divi{Q} \sim 0$,
zatem $\divi{S} - \divi{\ecident}$ również jest główny.
Istnieje zatem funkcja wymierna $s$,
której dywizor jest równy $\divi{S} - \divi{\ecident}$.
Jeżeli $S \neq \ecident$,
to kształtu dywizora widzimy, że funkcja $s$ nie ma biegunów skończonych,
zatem jest to wielomian.
Wielomian zaś, zgodnie z wnioskiem \ref{poly_no_single_zero_corollary}
nie może mieć jednego jednokrotnego miejsca zerowego.
Zatem $S = \ecident$, czyli punkt $P$ jest jedyny.
\end{proof}

Łatwo teraz wskazać bijekcję
postulowaną we wniosku \ref{piczero_curvepts_bijection_corollary}.

\begin{proof}[Dowód wniosku \ref{piczero_curvepts_bijection_corollary}]
Niech $\pi \colon E \to \Pic^0(E)$
będzie funkcją zadaną następującym wzorem:
$\pi(P) = \divi{P} - \divi{\ecident}$.
Dokładnie rzecz ujmując,
punktowi $P$ przyporządkowujemy nie dywizor $\divi{P} - \divi{\ecident}$,
lecz warstwę, do której należy
(pamiętamy, że $\Pic^0(E) = \Div^0(E)/\Prin(E)$).
Funkcja $\pi$ jest różnowartościowa --
jeśli $\pi(P) = \sigma(Q)$,
to dywizor $\divi{P} - \divi{Q}$ musiałby być główny,
co nie jest możliwe,
zgodnie z dowodem twierdzenia \ref{zerodeg_divisor_linear_reduction_theorem}.
Funkcja sigma jest też ,,na'' --
ponownie jest to wniosek z twierdzenia
\ref{zerodeg_divisor_linear_reduction_theorem}.
\end{proof}

\begin{corollary}\label{piczero_representants_coro}
Zerową grupę Picarda krzywej eliptycznej $E$
możemy utożsamić ze zbiorem dywizorów postaci $\divi{P} - \divi{\ecident}$,
gdzie $P \in E$.
\end{corollary}

Gdy mamy wybranych reprezentantów warstw tworzących zerową grupę Picarda,
przyjrzyjmy się, jak wygląda działanie grupowe na tych reprezentantach.

\begin{remark}\label{piczero_addition_remark}
W zerowej grupie Picarda dodawanie wygląda następująco.
Chcemy dodać do siebie dywizory
$\divi{P} - \divi{\ecident}$ oraz $\divi{Q} - \divi{\ecident}$.
Dodajemy je zgodnie z działaniem w grupie $\Div^0(E)$
i otrzymujemy $\divi{P} + \divi{Q} - 2\divi{\ecident}$.
Teraz, zgodnie z twierdzeniem \ref{zerodeg_divisor_linear_reduction_theorem}
redukujemy wynik do postaci $\divi{S} - \divi{\ecident}$.
W tym celu weźmy linię $l_1$ przechodzącą przez punkty $P$ oraz $Q$.
Ma ona dywizor $\rdiv(l) = \divi{P} + \divi{Q} + \divi{R} - 3\divi{\ecident}$.
Mamy
$\divi{P} + \divi{Q} - 2\divi{\ecident} - \rdiv(l_1) =
-\divi{R} + \divi{\ecident}$.
Wykonujemy jeszcze jedną redukcję za pomocą linii $l_2$
przechodzącej przez punkty $R$ oraz $\overline{R}$
(ma ona dywizor
$\rdiv(l_2) = \divi{R} + \divi{\overline{R}} - 2\divi{\ecident}$)
i otrzymujemy
$-\divi{R} + \divi{\ecident} + \rdiv(l_2) =
\divi{\overline{R}} - \divi{\ecident}$.

Jest to przypadek ogólny, należy rozważyć jeszcze przypadki,
kiedy $P = Q$ lub $R = \overline{R}$.
Należy wówczas dobrać odpowiednie linie styczne.
\end{remark}
