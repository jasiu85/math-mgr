\section{Szczególne grupy dywizorów}

Zbadamy teraz kilka ważnych grup dywizorów na krzywej eliptycznej.
Analiza tych grup doprowadzi nas do definicji
działania grupowego na krzywej eliptycznej.

\subsection*{Definicje grup}

Największą rozpatrywaną grupą jest oczywiście
grupa złożona ze wszystkich dywizorów na krzywej eliptycznej.

\begin{definition}
\emph{Grupa dywizorów na krzywej eliptycznej $E$},
oznaczana $\Div(E)$,
jest to abelowa grupa wolna generowana przez punkty krzywej $E$.
\end{definition}

Spośród wszystkich dywizorów na krzywej eliptycznej
wyróżnioną pozycję zajmują te,
które są dywizorami funkcji wymiernych.

\begin{definition}
Dywizor na krzywej eliptycznej jest \emph{główny},
jeśli jest on dywizorem pewnej funkcji wymiernej na tej krzywej.
\end{definition}

Następujące twierdzenie pokazuje, że dywizory główne tworzą grupę.

\begin{theorem}\label{fun_mul_divi_add_theorem}
Dane są funkcje wymierne $r$ oraz $s$ na krzywej eliptycznej.
Wówczas $\rdiv(rs) = \rdiv(r) + \rdiv(s)$.
\end{theorem}

\begin{definition}
\emph{Grupa dywizorów głównych na krzywej eliptycznej $E$},
oznaczana $\Prin(E)$,
składa się ze wszystkich dywizorów głównych na krzywej $E$.
\end{definition}

Grupa $\Prin(E)$ jest oczywiście podgrupą grupy $\Div(E)$,
a że obie są abelowe, istnieje ich grupa ilorazowa.

\begin{definition}
\emph{Grupa Picarda na krzywej eliptycznej $E$},
oznaczana $\Pic(E)$,
jest to grupa ilorazowa $\Div(E)/\Prin(E)$.
\end{definition}

Jak łatwo się przekonać,
dwa dywizory $\Delta_1$ i $\Delta_2$ należą do tej samej warstwy
będącej elementem grupy $\Pic(E)$
wtedy i tylko wtedy, gdy ich różnica jest dywizorem głównym.
Fakt ten prowadzi nas do następującej definicji.

\begin{definition}
Dywizory $\Delta_1$ oraz $\Delta_2$ są \emph{równoważne},
co zapisujemy $\Delta_1 \sim \Delta_2$,
jeśli ich różnica $\Delta_1 - \Delta_2$ jest dywizorem głównym.
\end{definition}

Wniosek \ref{function_order_sum_zero_corollary}
daje następującą częściową charakteryzację dywizorów głównych.

\begin{fact}\label{prin_divi_zero_deg_fact}
Jeżeli $\Delta$ jest dywizorem głównym,
to $\deg(\Delta) = 0$.
\end{fact}

To prowadzi nas do definicji kolejnej wartej uwagi grupy.

\begin{definition}
\emph{Grupa dywizorów stopnia zero na krzywej eliptycznej $E$},
oznaczana $\Div^0(E)$,
to grupa złożona z dywizorów stopnia zero.
\end{definition}

W związku z faktem \ref{prin_divi_zero_deg_fact}
zachodzą następujące inkluzje:
$\Prin(E) \subset \Div^0(E) \subset \Div(E)$.
Wszystkie wymienione grupy są abelowe,
więc znowu istnieją grupy ilorazowe.
Grupa ilorazowa $\Div(E) / \Div^0(E) = \Z$
nie jest specjalnie interesująca,
natomiast grupa $\Div^0(E) / \Prin(E)$
będzie dla nas bardzo ważna.

\begin{definition}
\emph{Zerowa grupa Picarda na krzywej eliptyzne $E$},
oznaczana $\Pic^0(E)$,
jest to grupa ilorazowa $\Div^0(E) / \Prin(E)$.
\end{definition}

\subsection*{Redukcje liniowe}

Konstrukcja różnych grup na krzywej eliptycznej opiera się
o równoważność dywizorów. Dwa dywizory są równoważne,
gdy różnią się o dywizor główny, czyli o dywizor pewnej funkcji wymiernej.
Jak wskazać funkcję wymierną, która ,,jest różnicą'' między dwoma dywizorami?
Następujące dwa twierdzenia pokazują, że taką funkcję wymierną
można opisać jako iloczyn linii.

\begin{theorem}\label{divisor_linear_reduction_theorem}
Dana jest krzywa eliptyczna $E$.
Wówczas dla każdego dywizora $\Delta \in \Div(E)$
istnieje dywizor $\tilde{\Delta} \in \Div(E)$ taki,
że $\Delta \sim \tilde{\Delta}$, $\deg(\Delta) = \deg(\tilde{\Delta})$
oraz $\tilde{\Delta} = \divi{P} + n\divi{\ecident}$,
gdzie $P \in E$.
\end{theorem}

\begin{proof}
Niech $\Delta \in \Div(E)$ będzie dowolnym dywizorem.
Rozważmy składniki, z których składa się dywizor $\Delta$.
Mogą zajść następujące przypadki:
\begin{enumerate}
\item dywizor $\Delta$ zawiera dwa składniki
$a(P)\divi{P}$ oraz $a(Q)\divi{Q}$,
gdzie $P$ oraz $Q$ to takie dwa różne punkty skończone krzywej,
że współczynniki $a(P)$ oraz $a(Q)$ są niezerowe oraz tego samego znaku;
\item dywizor $\Delta$ zawiera składnik $a(P)\divi{P}$,
gdzie $P$ to taki punkt skończony krzywej, że $\abs{a(P)} > 1$;
\item dywizor $\Delta$ ma postać $\divi{P} - \divi{Q} + n\divi{\ecident}$,
gdzie $P$ oraz $Q$ to punkty skończone krzywej;
\item dywizor $\Delta$ ma postać $\pm\divi{P} + n\divi{\ecident}$
lub $n\divi{\ecident}$,
gdzie $P$ to punkt skończony krzywej.
\end{enumerate}

Pokażemy, że w każdym przypadku z wyjątkiem ostatniego
możemy wskazać dywizor $\Delta'$ taki,
że $\Delta \sim \Delta'$, $\deg(\Delta) = \deg(\Delta')$
oraz $\abs{\Delta} > \abs{\Delta'}$.
\begin{enumerate}
\item Dla ustalenia uwagi przyjmimy,
że współczynniki $a(P)$ oraz $a(Q)$ są dodatnie.
Rozważmy dywizor $\Delta' = \Delta - \rdiv(l)$,
gdzie $l$ to linia przechodząca przez punkty $P$ oraz $Q$,
ma ona dywizor $\rdiv(l)$ równy
$\divi{P} + \divi{Q} + \divi{R} - 3\divi{\ecident}$.
Może zdarzyć się, że punkt $R$ jest równy $P$, $Q$ lub $\ecident$.
Łatwo sprawdzić, że zawsze $\abs{\Delta} > \abs{\Delta'}$.
\item Dla ustalenia uwagi przyjmijmy, że współczynnik $a(P)$ jest dodatni.
Rozważmy dywizor $\Delta' = \Delta - \rdiv(l)$,
gdzie $l$ jest linią styczną do krzywej w punkcie $P$,
ma ona dywizor $\rdiv(l)$ równy $2\divi{P} + \divi{R} - 3\divi{\ecident}$.
Również w tym przypadku $\abs{\Delta} > \abs{\Delta'}$,
nawet jeśli punkt $R$ jest równy $P$ lub $\ecident$.
\item Rozważmy linię $l$ przechodzącą przez punkty $P$ oraz $\overline{P}$.
Ma ona dywizor $\rdiv(l)$ równy
$\divi{P} + \divi{\overline{P}} - 2\divi{\ecident}$.
Rozważamy dywizor
$\Delta' =
\Delta - \rdiv(l) =
-\divi{\overline{P}} - \divi{Q} + n'\divi{\ecident}$,
do którego stosujemy rozumowanie opisane w punkcie pierwszym lub drugim
(zależnie od tego, czy $\overline{P} = Q$, czy nie)
i otrzymujemy dywizor $\Delta''$, który ma normę mniejszą niż $\Delta$.
Zwróćmy uwagę, że rozumowanie w tym kroku jest poprawne również wtedy,
gdy $P$ jest punktem podwójnym --
wówczas zamiast linii przechodzącej przez punkty $P$ oraz $\overline{P}$
bierzemy linię styczną do krzywej w punkcie $P$.
\end{enumerate}

Teraz konstruujemy ciąg dywizorów
$\Delta_0, \Delta_1, \Delta_2, \ldots$ taki,
że $\Delta_0 = \Delta$,
a dywizor $\Delta_{i+1}$ otrzymujemy
poprzez zastosowanie do dywizora $\Delta_i$ jednej z opisanych redukcji.
Widać, że wszystkie dywizory w ciągu są sobie równoważne
oraz mają taki sam stopień.
Co więcej, ciąg ten jest skończony,
bo normy kolejnych dywizorów maleją,
a nie mogą tego robić w nieskończoność -- są to liczby naturalne.
Ostatni dywizor w ciągu $\Delta_k$ musi więc mieć postać
$\pm\divi{P} + n\divi{\ecident}$ lub $n\divi{\ecident}$,
gdyż w przeciwnym razie można by kontynuować redukcję.

Jeżeli $\Delta_k = \divi{P} + n\divi{\ecident}$,
to jest to szukany dywizor.
Jeżeli $\Delta_k = n\divi{\ecident}$,
to również jest to szukany dywizor --
zapisujemy go w postaci $\Delta_k = \divi{\ecident} + (n-1)\divi{\ecident}$.
Jeśli zaś $\Delta_k = -\divi{P} + n\divi{\ecident}$,
to szukanym dywizorem jest dywizor
$\Delta_k + \rdiv(l)$,
gdzie $l$ to linia przechodząca przez punkty $P$ oraz $\overline{P}$
(lub styczna do krzywej w punkcie $P$, jeżeli jest to punkt podwójny).
Linia $l$ ma dywizor $\divi{P} + \divi{\overline{P}} - 2\divi{\ecident}$,
zatem $\Delta_k + \rdiv(l) = \divi{\overline{P}} + (n-2)\divi{\ecident}$.
Jest to dywizor równoważny dywizorowi $\Delta$ oraz mający taki sam stopień.
\end{proof}

\begin{lemma}\label{sim_pq_eq_pq_lemma}
Dane są punkty $P$ i $Q$ na krzywej eliptycznej $E$.
Jeżeli $\divi{P} \sim \divi{Q}$,
to $P = Q$.
\end{lemma}

\begin{proof}
Rozważmy przypadek, gdy jeden z punktów $P$ i $Q$ jest równy $\ecident$.
Dla ustalenia uwagi przyjmimy, że jest to punkt $Q$.
Zatem $\divi{P} - \divi{\ecident} \sim 0$,
czyli istnieje funkcja wymierna $r$,
której dywizorem jest $\divi{P} - \divi{\ecident}$.
Z kształtu dywizora widzimy, że jeżeli $P \neq \ecident$, to funkcja $r$
jest wielomianem, który ma jedno jednokrotne miejsce zerowe w punkcie $P$.
Na mocy wniosku \ref{poly_no_single_zero_corollary} nie jest to możliwe,
zatem $P = \ecident$.

Przyjmijmy więc, że punkty $P$ i $Q$ są skończone
i weźmy funkcję wymierną $r$, której dywizorem jest $\divi{P} - \divi{Q}$.
Oznaczmy $P = (a, b)$ i rozważmy dwa przypadki.
\begin{enumerate}
\item $P = \overline{P}$.
Unifikatorem w punkcie $P$ jest funkcja $u(x, y) = y$.
Widzimy, że $\ord_P(r) = 1$, zatem $r = us$, gdzie $s(P) \neq 0$.
Wiemy, że $\rdiv(u) = \divi{P} + \divi{P_1} + \divi{P_2} - 3\divi{\ecident}$,
gdzie $P_1$ i $P_2$ to pozostałe punkty podwójne,
skąd otrzymujemy, że
$\rdiv(\frac{1}{s}) = \divi{Q} + \divi{P_1} + \divi{P_2} - 3\divi{\ecident}$.
Funkcja $\frac{1}{s}$ jest zatem linią.
Linia przechodząca przez punkty $P_1$ i $P_2$ musi przechodzić również przez
trzeci punkt podwójny, czyli przez $P$. Zatem $Q = P$.
\item $P \neq \overline{P}$.
Unifikatorem w punkcie $P$ jest teraz funkcia $u(x, y) = x - a$.
Po analogicznych rachunkach otrzymujemy
$\rdiv(\frac{1}{s}) = \divi{Q} + \divi{\overline{P}} - 2\divi{\ecident}$.
Funkcja $\frac{1}{s}$ jest tym razem linią pionową,
skąd $\overline{Q} = \overline{P}$, czyli $Q = P$.
\end{enumerate}
\end{proof}

\begin{theorem}\label{zerodeg_divisor_linear_reduction_theorem}
Dana jest krzywa eliptyczna $E$.
Wówczas dla każdego dywizora $\Delta \in \Div^0(E)$
istnieje dokładnie jeden dywizor $\tilde{\Delta} \in \Div^0(E)$ taki,
że $\Delta \sim \tilde{\Delta}$, $\deg(\Delta) = \deg(\tilde{\Delta})$
oraz $\tilde{\Delta} = \divi{P} - \divi{\ecident}$,
gdzie $P \in E$.
\end{theorem}

\begin{proof}
Istnienie dywizora $\tilde{\Delta}$ to wniosek z twierdzenia
\ref{divisor_linear_reduction_theorem} --
dostajemy z niego, że istnieje żądany dywizor $\tilde{\Delta}$
z zastrzeżeniem, że ma on postać $\divi{P} + n\divi{\ecident}$.
Ale ponieważ $\deg(\tilde{\Delta}) = \deg(\Delta) = 0$,
widzimy, że $n = -1$, czyli $\tilde{\Delta} = \divi{P} - \divi{\ecident}$.
Jednoznaczność otrzymujemy udowodnionego przed chwilą lematu
\ref{sim_pq_eq_pq_lemma}.
\end{proof}

\subsection*{Zerowa grupa Picarda}

Dzięki twierdzeniom \ref{divisor_linear_reduction_theorem}
oraz \ref{zerodeg_divisor_linear_reduction_theorem}
możemy szczegółowo zanalizować zerową grupę Picarda.

\begin{theorem}\label{piczero_curvepts_bijection_theorem}
Dana jest krzywa eliptyczna $E$.
Niech funkcja $\pi\colon E \to \Pic^0(E)$ będzie określona wzorem:
\begin{equation}
\pi(P) = [\divi{P} + n\divi{\ecident}]
\end{equation}
Wówczas funkcja $\pi$ jest bijekcją.
\end{theorem}

\begin{proof}
Funkcja $\pi$ jest różnowartościowa. Jeżeli $\pi(P) = \pi(Q)$,
to znaczy, że dywizory $\divi{P} - \divi{\ecident}$
oraz $\divi{Q} - \divi{\ecident}$ są w tej samej warstwie,
skąd $\divi{P} - \divi{Q} \sim 0$.
Z lematu \ref{sim_pq_eq_pq_lemma} dostajemy $P = Q$.

Funkcja $\pi$ jest ,,na''. Weźmy dowolną warstwę będącą elementem
zerowej grupy Picarda i wybierzmy z niej dowolny dywizor $\Delta$.
Jest on równoważny pewnemu dywizorowi $\divi{P} - \divi{\ecident}$.
Wówczas $\pi(P)$ jest właśnie tą warstwą.
\end{proof}

Zerową grupę Picarda możemy zatem utożsamić
ze zbiorem punktów krzywej eliptycznej
wybierając z każdej warstwy reprezentanta postaci $\divi{P} - \divi{\ecident}$.
Zobaczmy, jak wygląda dodawanie takich reprezentantów.

\begin{fact}\label{piczero_addition_fact}
Dodawanie dwóch elementów zerowej grupy Picarda
$\divi{P} - \divi{\ecident}$ oraz $\divi{Q} - \divi{\ecident}$
przeprowadzamy następująco.
\begin{enumerate}
\item
Jeżeli $P = \ecident$ (odpowiednio $Q = \ecident$),
to wynik dodawania jest równy
$\divi{Q} - \divi{\ecident}$ (odpowiednio $\divi{P} - \divi{\ecident}$).
\item
W przeciwnym razie dodajemy dywizory
$\divi{P} - \divi{\ecident}$ oraz $\divi{Q} - \divi{\ecident}$
zgodnie z działaniem w grupie $\Div^0(E)$.
Otrzymujemy:
\begin{equation*}
\divi{P} + \divi{Q} - 2\divi{\ecident}
\end{equation*}
\item
Zgodnie z twierdzeniem \ref{zerodeg_divisor_linear_reduction_theorem}
redukujemy wynik za pomocą linii $l_1$ przechodzącej przez punkty $P$ i $Q$
(lub stycznej, jeśli $P = Q$).
Otrzymujemy:
\begin{eqnarray*}
\divi{P} + \divi{Q} - 2\divi{\ecident} - \rdiv(l_1)
& = & \divi{P} + \divi{Q} - 2\divi{\ecident}
- \divi{P} - \divi{Q} - \divi{R} + 3\divi{\ecident}
\\ & = &
-\divi{R} + \divi{\ecident}
\end{eqnarray*}
\item
Jeżeli $P = \overline{Q}$,
to $R = \ecident$
i wynik dodawania jest równy $\divi{\ecident} - \divi{\ecident}$.
\item
W przeciwnym razie wykonujemy jeszcze jedną redukcję za pomocą linii $l_2$
przechodzącej przez punkty $R$ oraz $\overline{R}$
(lub stycznej, gdy $R = \overline{R}$).
Otrzymujemy:
\begin{eqnarray*}
-\divi{R} + \divi{\ecident} + \rdiv(l_2)
& = & -\divi{R} + \divi{\ecident}
+ \divi{R} + \divi{\overline{R}} - 2\divi{\ecident}
\\ & = &
\divi{\overline{R}} - \divi{\ecident}
\end{eqnarray*}
\end{enumerate}
\end{fact}
