\section{Linie}

Aby zdefiniować działanie grupowe na krzywej eliptycznej,
będziemy wielokrotnie posługiwać się
pewną specyficzną klasą wielomianów.
Są to odpowiedniki wielomianów liniowych
i stanowią one algebraiczny odpowiednik
geometrycznego pojęcia linii prostej.

\subsection*{Definicja}

Przyjmujemy następującą definicję linii na krzywej eliptycznej.

\begin{definition}
Dana jest krzywa eliptyczna $E = E(\K)$
oraz trzy elementy $a, b, c \in \K$ takie,
że $a$ i $b$ nie są jednocześnie równe $0$.
\emph{Linia na krzywej eliptycznej $E(\K)$ o współczynnikach $a$, $b$ i $c$}
to wielomian $ax + by + c$.
Zbiór wszystkich linii na krzywej eliptycznej $E$
oznaczamy symbolem $\Lines(E)$.
\end{definition}

\begin{definition}
Dana jest krzywa eliptyczna $E$,
linia $l \in \Lines(E)$
oraz punkt $P \in E$.
Mówimy, że \emph{linia $l$ przecina krzywą $E$ w punkcie $P$},
jeśli $l(P) = 0$.
\end{definition}

\begin{definition}
Dana jest krzywa eliptyczna $E$,
linia $l \in \Lines(E)$
oraz punkt $P \in E$.
Mówimy, że \emph{linia $l$ jest styczna do krzywej $E$ w punkcie $P$},
jeśli $\ord_P(l) > 1$.
\end{definition}

\subsection*{Własności}

Następujące własności są to po prostu podane wcześniej fakty
zastosowane do przypadku linii.

\begin{fact}\label{line_deg_fact}
Dana jest krzywa eliptyczna $E$
oraz linia $l \in \Lines(E)$, $l = ax + by + c$.
Jeżeli $b \neq 0$, to $\deg(l) = 3$,
w przeciwnym razie $\deg(l) = 2$.
\end{fact}

\begin{fact}\label{line_divisor_norm_fact}
Dana jest krzywa eliptyczna $E$
oraz linia $l \in \Lines(E)$, $l = ax + by + c$.
Jeżeli $b \neq 0$, to $\abs{\rdiv(l)} = 3$,
w przeiwnym razie $\abs{\rdiv(l)} = 2$.
\end{fact}

Będziemy mieć dużo do czynienia z liniami,
dlatego musimy uściślić wyniki,
które przedstawia fakt \ref{line_divisor_norm_fact}.

\begin{fact}\label{possible_line_divisors_fact}
Dana jest krzywa eliptyczna $E$
oraz linia $l \in \Lines(E)$.
Niech $P, Q, R \in E$ oznaczają dowolne trzy punkty krzywej $E$,
które są parami różne i różne od punktu w nieskończoności.
Wówczas dywizor $\rdiv(l)$ może mieć jedną z następujących postaci:
\begin{itemize}
\item $\rdiv(l) = \divi{P} + \divi{Q} + \divi{R} - 3\divi{\ecident}$;
\item $\rdiv(l) = 2\divi{P} + \divi{Q} - 3\divi{\ecident}$;
\item $\rdiv(l) = 3\divi{P} - 3\divi{\ecident}$;
\item $\rdiv(l) = \divi{P} + \divi{Q} - 2\divi{\ecident}$;
\item $\rdiv(l) = 2\divi{P} - 2\divi{\ecident}$.
\end{itemize}
\end{fact}

\begin{remark}\label{possible_line_divisors_remark}
Składniki dywizorów wymienionych w fakcie \ref{possible_line_divisors_fact}
są pogrupowane. Czasami nie będziemy ich grupować,
tylko zapisywać w jednej z następujących postaci:
\begin{itemize}
\item $\rdiv(l) = \divi{P} + \divi{Q} + \divi{R} - 3\divi{\ecident}$ --
jest to najbardziej ogólna forma,
może zdarzyć się, że punkty $P$, $Q$ oraz $R$ nie są parami różne
lub że jeden z nich nie jest skończony;
\item $\rdiv(l) = \divi{P} + \divi{\overline{P}} - 2\divi{\ecident}$ --
jest to ogólna postać linii pionowych,
tzn. takich, które mają współczynnik przy zmiennej $y$ równy $0$,
może zdarzyć się, że $P = \overline{P}$.
\end{itemize}
\end{remark}

Następujące twierdzenia pokazują,
w jaki sposób linie przecinają się z krzywymi eliptycznymi.

\begin{theorem}\label{line_through_two_points_theorem}
Dana jest krzywa eliptyczna $E$
oraz punkty skończone $P, Q \in E$, $P \neq Q$.
Wówczas przez punkty $P$ i $Q$
przechodzi dokładnie jedna (z dokładnością do czynnika stałego) linia.
\end{theorem}

\begin{proof}
Oznaczmy $P = (a, b), Q = (c, d)$ i rozważmy dwa przypadki.
\begin{enumerate}
\item
Jeżeli $P = \overline{Q}$,
to rozpatrujemy linię:
\begin{equation}
l(x, y) = x - a
\end{equation}
Jak nietrudno sprawdzić, $l(P) = 0$ oraz $l(Q) = 0$,
ponieważ $c = a$.
\item
Jeżeli $P \neq \overline{Q}$,
to rozpatrujemy linię:
\begin{equation}
l(x, y) = \left(\frac{d-b}{c-a}\right)(x - a) - (y - b)
\end{equation}
Ponownie, $l(P) = 0$ oraz $l(Q) = 0$.
Zauważmy też,
że iloraz $\frac{d-b}{c-a}$ jest dobrze określony,
ponieważ $c \neq a$.
\end{enumerate}

W obu przypadkach można pokazać, że inne linie
przechodzące przez punkty $P$ i $Q$ różnią się od linii $l$ o czynnik stały.
Wystarczy rozpatrzyć odpowiedni układ równań liniowych
na współczynniki linii:
jedno równanie ma postać $l(P) = 0$, a drugie -- $l(Q) = 0$.
Są to dwa równania liniowe trzech zmiennych,
przestrzeń rozwiązań jest więc jednowymiarowa.
Widać też, że wszystkie rozwiązania są proporcjonalne,
ponieważ układ równań jest jednorodny.
\end{proof}

\begin{theorem}
Dana jest krzywa eliptyczna $E$,
punkty skończone $P, Q \in E$, $P \neq Q$.
Niech $l \in \Lines(E)$ będzie linią przechodzącą przez punkty $P$ i $Q$.
Wówczas:
\begin{itemize}
\item jeżeli $P = \overline{Q}$,
to $\rdiv(l) = \divi{P} + \divi{Q} - 2\divi{\ecident}$;
\item jeżeli $P \neq \overline{Q}$,
to istnieje punkt skończony $R \in E$ taki,
że $\rdiv(l) = \divi{P} + \divi{Q} + \divi{R} - 3\divi{\ecident}$,
może przy tym zdarzyć się, że $R = P$ lub $R = Q$.
\end{itemize}
\end{theorem}

\begin{proof}
Rozważmy oba przypadki.
\begin{enumerate}
\item
Jeżeli $P = \overline{Q}$,
to na podstawie twierdzeń
\ref{line_through_two_points_theorem} i \ref{polynomial_ord_deg_theorem}
oraz faktu \ref{line_deg_fact} widzimy,
że linia $l$ ma dwa różne miejsca zerowe lub jedno podwójne.
Drugi przypadek jest niemożliwy,
ponieważ punkty $P$ i $Q$ są jej miejscami zerowymi i są różne.
Zatem $\rdiv(l) = \divi{P} + \divi{Q} - 2\divi{\ecident}$.
\item
Jeżeli $P \neq \overline{Q}$,
to ponownie z twierdzeń
\ref{line_through_two_points_theorem} i \ref{polynomial_ord_deg_theorem}
oraz faktu \ref{line_deg_fact} widzimy,
że linia $l$ ma trzy różne miejsca zerowe, jedno pojedyńcze i jedno podwójne
lub jedno potrójne.
Ostatni przypadek znów jest niemożliwy.
W pozostałych przypadkach można dobrać punkt skończony $R$
(być może równy $P$ lub $Q$) taki,
że $\rdiv(l) = \divi{P} + \divi{Q} + \divi{R} - 3\divi{\ecident}$.
\end{enumerate}
\end{proof}

\begin{theorem}
Dane są dwa różne punkty skończone $P = (a, b)$ i $Q = (c, d)$
na krzywej eliptycznej oraz linia $l$ przechodząca przez te punkty.
Jeżeli $\rdiv(l) = \divi{P} + \divi{Q} + \divi{R} - 3\divi{\ecident}$,
gdzie $R = (e, f)$ jest punktem skończonym krzywej,
to:
\begin{eqnarray}
\label{chord_line_third_point_x_eqn}
e & = & \left(\frac{d-b}{c-a}\right)^2 - a - c \\
\label{chord_line_third_point_y_eqn}
f & = & b + \left(\frac{d-b}{c-a}\right)(e-a)
\end{eqnarray}
\end{theorem}

\begin{proof}
Rozważmy normę $N(l)$ linii $l$. Jest ona równa:
\begin{eqnarray*}
N(l) = l\overline{l}
& = & \left(\left(\frac{d-b}{c-a}\right)(x - a) - (y - b)\right)
      \overline{\left(\left(\frac{d-b}{c-a}\right)(x - a) - (y - b)\right)} \\
& = & \left(\left(\frac{d-b}{c-a}\right)(x-a) + b\right)^2 - x^3 - Ax - B \\
& = & -x^3 + \left(\frac{d-b}{c-a}\right)^2x^2 + \cdots
\end{eqnarray*}
Na mocy wniosku \ref{zero_of_norm_coro}
jej miejsca zerowe to $a$, $c$ oraz $e$.
Współrzędną $e$ znajdujemy ze wzoru Viete'a na sumę pierwiastków:
\begin{eqnarray*}
a + c + e & = & \left(\frac{d-b}{c-a}\right)^2 \\
        e & = & \left(\frac{d-b}{c-a}\right)^2 - a - c
\end{eqnarray*}
Współrzędną $f$ obliczamy z równania $l(R) = 0$:
\begin{eqnarray*}
\left(\frac{d-b}{c-a}\right)(e - a) - (f - b) & = & 0 \\
f & = & b + \left(\frac{d-b}{c-a}\right)(e-a)
\end{eqnarray*}
\end{proof}

\begin{theorem}\label{line_tangent_at_point_theorem}
Przez dowolny punkt skończony $P$ krzywej eliptycznej
można poprowadzić dokładnie jedną (z dokładnością do czynnika stałego)
linię styczną do krzywej w tym punkcie.
\end{theorem}

\begin{proof}
Oznaczmy $P = (a, b)$ i rozważmy dwa przypadki.
\begin{enumerate}
\item
Jeżeli $P$ jest punktem podwójnym, tzn. $P = \overline{P}$ i $b = 0$,
to rozpatrujemy linię:
\begin{equation}
l(x, y) = x - a
\end{equation}
Ma ona stopień równy $2$,
a jej dywizor, zgodnie z uwagą \ref{possible_line_divisors_remark},
ma postać $\divi{P} + \divi{\overline{P}} - 2\divi{\ecident}$.
Skoro $P$ jest punktem podwójnym, to dywizor ten jest równy
$2\divi{P} - 2\divi{\ecident}$,
zatem linia $l$ jest styczna do krzywej w punkcie $P$.
\item
Jeżeli $P$ nie jest punktem podwójnym,
to rozpatrujemy linię:
\begin{equation}
l(x, y) = \left(\frac{3a^2 + A}{2b}\right)(x - a) - (y - b)
\end{equation}
Linia $l$ ma w punkcie $P$ miejsce zerowe, zatem $\ord_P(l) > 0$.
Pokażemy, że $\ord_P(l) > 1$.
W tym celu wykażemy, że jeśli $l = us$,
gdzie $u$ jest unifikatorem w punkcie $P$,
a $s$ jest funkcją wymierną,
to $s(P) = 0$.
Ponieważ punkt $P$ nie jest podwójny,
bierzemy unifikator $u(x, y) = x - a$.
Wyznaczamy funkcję wymierną $s$.
\begin{eqnarray*}
s = \frac{l}{u}
& = & \frac{\left(\frac{3a^2+A}{2b}\right)(x-a) - (y-b)}{x-a} \\
& = & \frac{3a^2+A}{2b} - \frac{y-b}{x-a} \\
& = & \frac{3a^2+A}{2b} - \frac{y^2-b^2}{(x-a)(y+b)} \\
& = & \frac{3a^2+A}{2b} - \frac{x^3+Ax+B - a^3-Aa-B}{(x-a)(y+b)} \\
& = & \frac{3a^2+A}{2b} - \frac{(x-a)(x^2+ax+a^2+A)}{(x-a)(y+b)} \\
& = & \frac{3a^2+A}{2b} - \frac{x^2+ax+a^2+A}{y+b}
\end{eqnarray*}
Widzimy teraz, że $s(P) = 0$,
Zatem $\ord_P(l) > 1$, czyli linia $l$ jest styczna do krzywej w punkcie $P$.
\end{enumerate}
Ponownie, to, że inne linie styczne w punkcie $P$ różnią się od linii $l$
o czynnik stały, wynika z rozpatrzenia odpowiedniego układu równań liniowych.
Jedno równanie ma postać $l(P) = 0$, a drugie --
$s(P) = (\frac{l}{u})(P) = 0$
(można je przekształcić do postaci równania liniowego na współczynniki linii).
\end{proof}

\begin{theorem}
Dany jest punkt skończony $P$ na krzywej eliptycznej
oraz linia $l$ styczna do krzywej w tym punkcie.
Wówczas:
\begin{itemize}
\item jeśli $P = \overline{P}$,
to $\rdiv(l) = 2\divi{P} - 2\divi{\ecident}$;
\item jeśli $P \neq \overline{P}$,
to $\rdiv(l) = 2\divi{P} + \divi{Q} - 3\divi{\ecident}$,
przy czym punkt $Q$ jest skończony
i może zdarzyć się, że $Q = P$.
\end{itemize}
\end{theorem}

\begin{proof}
Rozważmy dwa przypadki.
\begin{enumerate}
\item
Jeżeli $P = \overline{P}$,
to na podstawie twierdzeń
\ref{line_tangent_at_point_theorem} i \ref{polynomial_ord_deg_theorem}
oraz faktu \ref{line_deg_fact} widzimy,
że linia $l$ ma dwa różne miejsca zerowe lub jedno podwójne.
Pierwszy przypadek jest niemożliwy,
ponieważ punkt $P$ jest jej podwójnym miejscem zerowym.
Zatem $\rdiv(l) = 2\divi{P} - 2\divi{\ecident}$.
\item
Jeżeli $P \neq \overline{P}$,
to ponownie z twierdzeń
\ref{line_tangent_at_point_theorem} i \ref{polynomial_ord_deg_theorem}
oraz faktu \ref{line_deg_fact} widzimy,
że linia $l$ ma trzy różne miejsca zerowe, jedno pojedyńcze i jedno podwójne
lub jedno potrójne.
Pierwszy przypadek znów jest niemożliwy.
W pozostałych przypadkach można dobrać punkt skończony $Q$
(być może równy $P$) taki,
że $\rdiv{l} = 2\divi{P} + \divi{Q} - 3\divi{\ecident}$.
\end{enumerate}
\end{proof}

\begin{theorem}
Dany jest punkt skończony $P = (a, b)$ na krzywej eliptycznej
oraz linia $l$ styczna do krzywej w tym punkcie.
Jeżeli $div(l) = 2\divi{P} + \divi{Q} - 3\divi{\ecident}$,
gdzie $Q = (c, d)$ jest punktem skończonym krzywej,
to:
\begin{eqnarray}
\label{tangent_line_third_point_x_eqn}
c & = & \left(\frac{3a^2 + A}{2b}\right)^2 - 2a \\
\label{tangent_line_third_point_y_eqn}
d & = & b + \left(\frac{3a^2 + A}{2b}\right)(c - a)
\end{eqnarray}
\end{theorem}

\begin{proof}
Rozważmy normę $N(l)$ linii $l$. Jest ona równa:
\begin{eqnarray*}
N(l) = l\overline{l}
& = & \left(\left(\frac{3a^2 + A}{2b}\right)(x - a) - (y - b)\right)
      \overline{
      \left(\left(\frac{3a^2 + A}{2b}\right)(x - a) - (y - b)\right)
      } \\
& = & \left(\left(\frac{3a^2+A}{2b}\right)(x-a) + b\right)^2 - x^3 - Ax - B \\
& = & -x^3 + \left(\frac{3a^2+A}{2b}\right)^2x^2 + \cdots
\end{eqnarray*}
Na mocy wniosku \ref{zero_of_norm_coro}
jej miejsca zerowe to $a$ (podwójne) oraz $c$.
Współrzędną $c$ znajdujemy ze wzorów Viete'a na sumę pierwiastków:
\begin{eqnarray*}
2a + c & = & \left(\frac{3a^2+A}{2b}\right)^2 \\
     c & = & \left(\frac{3a^2+A}{2b}\right)^2 - 2a
\end{eqnarray*}
Współrzędną $d$ obliczamy z równania $l(Q) = 0$:
\begin{eqnarray*}
\left(\frac{3a^2 + A}{2b}\right)(c - a) - (d - b) & = & 0 \\
d & = & b + \left(\frac{3a^2 + A}{2b}\right)(c - a)
\end{eqnarray*}
\end{proof}

Niezwykle istotny jest fakt,
że współczynniki linii przechodzącej przez dwa punkty
lub stycznej w punkcie,
jak i współrzędne dodatkowego punktu przecięcia
takiej linii z krzywą eliptyczną dają się wyrazić
jako nieskomplikowane wyrażenia wymierne.

\begin{fact}
Dana jest krzywa eliptyczna $E(\fieldL)$, ciało $\K \subset \fieldL$
oraz linia $l$ przechodząca przez punkty $P$ i $Q$ krzywej.
Oznaczmy $\rdiv(l) = \divi{P} + \divi{Q} + \divi{R} - 3\divi{\ecident}$.
Wówczas, jeżeli punkty $P$ i $Q$ są $\K$-wymierne,
to również punkt $R$ jest $\K$-wymierny,
a współczynniki linii $l$ są elementami ciała $\K$.
a punkt $R$ jest $\K$-wymierny.
\end{fact}

\begin{fact}
Dana jest krzywa eliptyczna $E(\fieldL)$, ciało $\K \subset \fieldL$
oraz linia $l$ styczna do krzywej w punkcie $P$.
Oznaczmy $\rdiv(l) = 2\divi{P} + \divi{Q} - 3\divi{\ecident}$.
Wówczas, jeżeli punkt $P$ jest $\K$-wymierny,
to również punkt $Q$ jest $\K$-wymierny,
a współczynniki linii $l$ są elementami ciała $\K$.
\end{fact}

Istota tych faktów polega na tym,
że ciało $\K$ nie musi być algebraicznie domknięte,
a mimo to zawsze da się wyznaczyć współczynniki linii
oraz dodatkowego punktu przecięcia linii z krzywą.
A priori nie było to oczywiste --
uzyskane wzory mogły być bardziej skomplikowane (np. wymagać pierwiastkowania)
i konkretne liczby niezbędne do wyrażenia współczynników i współrzędnych
znajdowałyby się dopiero w domknięciu algebraicznym $\overline{\K}$
ciała $\K$.
