\section{Definicja}

Zdefiniujemy teraz działanie grupowe na zbiorze punktów krzywej eliptycznej.

\begin{definition}\label{ec_groupop_def}
Dana jest krzywa eliptyczna $E$.
Na zbiorze jej punktów określamy działanie grupowe,
oznaczane symbolem $+$,
w następujący sposób.
\begin{enumerate}

\item\label{ec_groupop_neutral}
Elementem neutralnym działania jest punkt $\ecident$.
\begin{equation}
P + \ecident = \ecident = \ecident + P
\end{equation}
\item\label{ec_groupop_inverse}
Elementem przeciwnym do punktu $P \neq \ecident$,
oznaczanym $-P$,
jest punkt $\overline{P}$.
\begin{equation}
P + \overline{P} = \ecident = \overline{P} + P
\end{equation}
\item\label{ec_groupop_add_generic}
Suma dwóch punktów skończonych $P = (a, b)$ oraz $Q = (c, d)$,
przy czym $Q \neq \overline{P}$, to punkt $R = (e, f)$,
którego współrzędne są określone następującymi wzorami:
\begin{eqnarray}\label{ec_groupop_add_generic_eqn}
e & = & \lambda^2 - a - c \\
f & = & -\lambda(e - a) - b
\end{eqnarray}
Współczynnik $\lambda$ jest określony następująco:
\begin{enumerate}
\item\label{ec_groupop_add_generic_chord}
jeżeli $a \neq c$, to:
\begin{equation}\label{ec_groupop_add_generic_lambda_chord_eqn}
\lambda = \frac{d-b}{c-a}
\end{equation}
\item\label{ec_groupop_add_generic_tangent}
jeżeli $a = c$, to:
\begin{equation}\label{ec_groupop_add_generic_lambda_tangent_eqn}
\lambda = \frac{3a^2+A}{2b}
\end{equation}
\end{enumerate}
\end{enumerate}
\end{definition}

\begin{remark}
Nie wprowadzamy nowego określenia na grupę z tak zdefiniowanym działaniem.
Będziemy ją po prostu nazywać ,,krzywa eliptyczna''
i oznaczać tak samo, jak oznaczamy krzywe.
\end{remark}

\begin{remark}
W podanej definicji w punkcie \ref{ec_groupop_add_generic}
suma dwóch punktów $P$ i $Q$
nie jest określona, gdy $Q = \overline{P}$
lub gdy jeden z punktów jest nieskończony.
Wynik dodawania jest wówczas określony
na podstawie zależności podanych w punkcie
\ref{ec_groupop_neutral} lub \ref{ec_groupop_inverse}.
\end{remark}

\begin{remark}
Przypadek opisany w punkcie \ref{ec_groupop_add_generic_tangent}
zachodzi wtedy,
gdy próbujemy dodać punkt skończony niebędący punktem rzędu dwa
do samego siebie.
\end{remark}

Jak widać, dodanie do siebie dwóch dowolnych punktów krzywej
nie wymaga skomplikowanych obliczeń.
W każdym przypadku współrzędne punktu będącego wynikiem dodawania
są określone za pomocą wyrażeń wymiernych.

\begin{fact}
Dana jest krzywa eliptyczna $E(\fieldL)$ oraz ciało $\K \subset \fieldL$.
Wówczas element odwrotny do punktu $\K$-wymiernego
jest $\K$-wymierny,
jak również suma dwóch $\K$-wymiernych jest $\K$-wymierna.
\end{fact}

Wiemy więc, że działanie grupowe nie wyprowadza poza zbiór $E(\K)$.
Pokażemy, teraz, że faktycznie zadaje ono grupę,
tzn. że spełnia prawa grupowe.
Sprawdzenie łączności bezpośrednio z definicji jest bardzo żmudne.
Na szczęście można zrobić to inaczej.

\begin{lemma}
Przez punkty $P$, $Q$ oraz $R$ na krzywej eliptycznej $E$
przechodzi jedna linia wtedy i tylko wtedy,
gdy $P + Q + R = \ecident$.
\end{lemma}

\begin{proof}
W ogólnym przypadku dodanie dwóch punktów krzywej $P$ i $Q$ polega na:
poprowadzeniu przez $P$ i $Q$ linii,
znalezieniu trzeciego punktu przecięcia $R$ tej linii z krzywą
oraz wzięciu jego sprzężenia $\overline{R}$.
Widzimy to, porównując wzory
\ref{ec_groupop_add_generic_eqn},
\ref{ec_groupop_add_generic_lambda_chord_eqn}
i \ref{ec_groupop_add_generic_lambda_tangent_eqn}
z wzorami
\ref{chord_line_third_point_x_eqn},
\ref{chord_line_third_point_y_eqn},
\ref{tangent_line_third_point_x_eqn}
i \ref{tangent_line_third_point_y_eqn}.
Stąd $P + Q = \overline{R}$.
Ale $\overline{R} = -R$, stąd $P + Q + R = \ecident$.
\end{proof}

Konsekwencje tego lematu stają się jasne,
gdy zestawimy go razem z twierdzeniami
\ref{divisor_linear_reduction_theorem}
oraz \ref{zerodeg_divisor_linear_reduction_theorem},
wnioskami \ref{piczero_curvepts_bijection_corollary}
oraz \ref{piczero_representants_coro}
i z uwagą \ref{piczero_addition_remark}.

\begin{theorem}
Niech $\pi \colon E \to \Pic^0(E)$
będzie bijekcją zdefiniowaną w dowodzie wniosku
\ref{piczero_curvepts_bijection_corollary}.
Wówczas dla dowolnych punktów $P, Q \in E$
zachodzi $\pi(P + Q) = \pi(P) + \pi(Q)$.
\end{theorem}

\begin{proof}
Natychmiastowy.
\end{proof}

\begin{corollary}
Krzywa eliptyczna $E$ wraz z działaniem określonym w definicji
\ref{ec_groupop_def} tworzy grupę abelową.
Grupa ta jest izomorficzna z grupą $\Pic^0(E)$.
\end{corollary}
