\section{Definicja}

Zdefiniujemy teraz działanie grupowe na zbiorze punktów krzywej eliptycznej.

\begin{definition}\label{ec_groupop_def}
Dana jest krzywa eliptyczna $E$.
\emph{Grupa na krzywej eliptycznej $E$}
składa się ze zbioru punktów krzywej $E$
oraz działania, oznaczanego symbolem $+$,
określonego w następujący sposób.
\begin{enumerate}
\item\label{ec_groupop_neutral}
Elementem neutralnym działania jest punkt $\ecident$.
\begin{equation}
P + \ecident = \ecident = \ecident + P
\end{equation}
\item\label{ec_groupop_inverse}
Elementem przeciwnym do punktu $P \neq \ecident$,
oznaczanym $-P$,
jest punkt $\overline{P}$.
\begin{equation}
P + \overline{P} = \ecident = \overline{P} + P
\end{equation}
\item\label{ec_groupop_add_generic}
Suma dwóch punktów skończonych $P = (a, b)$ oraz $Q = (c, d)$,
przy czym $Q \neq \overline{P}$, to punkt $R = (e, f)$,
którego współrzędne są określone następującymi wzorami:
\begin{eqnarray}
\label{ec_groupop_add_generic_eqn_x}
e & = & \lambda^2 - a - c \\
\label{ec_groupop_add_generic_eqn_y}
f & = & -\lambda(e - a) - b
\end{eqnarray}
Współczynnik $\lambda$ jest określony następująco:
\begin{enumerate}
\item\label{ec_groupop_add_generic_chord}
jeżeli $a \neq c$, to:
\begin{equation}\label{ec_groupop_add_generic_lambda_chord_eqn}
\lambda = \frac{d-b}{c-a}
\end{equation}
\item\label{ec_groupop_add_generic_tangent}
jeżeli $a = c$, to:
\begin{equation}\label{ec_groupop_add_generic_lambda_tangent_eqn}
\lambda = \frac{3a^2+A}{2b}
\end{equation}
\end{enumerate}
\end{enumerate}
\end{definition}

\begin{remark}
Nie wprowadzamy nowego oznaczenia na grupę na krzywej eliptycznej --
będziemy ją oznaczać tak samo, jak oznaczamy krzywe.
\end{remark}

\begin{remark}
W podanej definicji w punkcie \ref{ec_groupop_add_generic}
suma dwóch punktów $P$ i $Q$
nie jest określona, gdy $Q = \overline{P}$
lub gdy jeden z punktów jest nieskończony.
Wynik dodawania jest wówczas określony
na podstawie zależności podanych w punkcie
\ref{ec_groupop_neutral} lub \ref{ec_groupop_inverse}.
\end{remark}

\begin{remark}
Przypadek opisany w punkcie \ref{ec_groupop_add_generic_tangent}
zachodzi wtedy,
gdy próbujemy dodać punkt skończony niebędący punktem rzędu dwa
do samego siebie.
\end{remark}

Jak widać, dodanie do siebie dwóch dowolnych punktów krzywej
nie wymaga skomplikowanych obliczeń --
w każdym przypadku współrzędne punktu będącego wynikiem dodawania
są określone jako funkcje wymierne od współrzędnych dodawanych punktów.

\begin{fact}
Dana jest krzywa eliptyczna $E(\fieldL)$ oraz ciało $\K \subset \fieldL$.
Wówczas suma dwóch punktów $\K$-wymiernych jest punktem $\K$-wymiernym.
\end{fact}

Wiemy więc, że działanie grupowe nie wyprowadza poza zbiór $E(\K)$.
Pokażemy, teraz, że faktycznie zadaje ono grupę,
tzn. że spełnia prawa grupowe.
Sprawdzenie łączności bezpośrednio z definicji jest bardzo żmudne.
Na szczęście można zrobić to inaczej.

\begin{theorem}
Niech $\pi \colon E \to \Pic^0(E)$
będzie bijekcją zdefiniowaną w dowodzie wniosku
\ref{piczero_curvepts_bijection_corollary}.
Wówczas dla dowolnych punktów $P, Q \in E$:
\begin{itemize}
\item $\pi(P + Q) = \pi(P) + \pi(Q)$;
\item
$\pi^{-1}\left(
\divi{P} - \divi{\ecident}
+ \divi{Q} - \divi{\ecident}
\right)
=
\pi^{-1}(\divi{P} - \divi{\ecident})
+
\pi^{-1}(\divi{Q} - \divi{\ecident})$.
\end{itemize}
\end{theorem}

\begin{proof}
Natychmiastowy na podstawie
twierdzeń
\ref{divisor_linear_reduction_theorem}
i \ref{zerodeg_divisor_linear_reduction_theorem},
wniosków \ref{piczero_curvepts_bijection_corollary}
i \ref{piczero_representants_coro}
oraz faktu \ref{piczero_addition_fact}.
\end{proof}

\begin{corollary}
Krzywa eliptyczna $E$ wraz z działaniem określonym w definicji
\ref{ec_groupop_def} tworzy grupę abelową.
Grupa ta jest izomorficzna z grupą $\Pic^0(E)$.
\end{corollary}
