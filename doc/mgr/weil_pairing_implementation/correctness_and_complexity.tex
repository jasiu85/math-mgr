\section{Dowód poprawności i oszacowanie złożoności}

\noindent
Zajmiemy się teraz analizą algorytmu Millera:
udowodnimy jego poprawność
oraz oszacujemy złożoność czasową i pamięciową.
Analizę poprzedzimy krótkim wyjaśnieniem
idei stojącej za algorytmem.

\subsection*{Wyjaśnienie}

\noindent
Podstawę, na której opiera się algorytm Millera,
stanowi następująca definicja i płynące z niej wnioski.

\begin{definition}
Dany jest punkt $P$ rzędu $n$ na krzywej eliptycznej $E$.
Niech $R$ będzie dowolnym punktem na krzywej $E$.
\emph{Cząstkowe wartości iloczynu Weila}
to rodzina funkcji wymiernych $r_{P,R}^{(m)}$ na krzywej $E$,
gdzie $m = 0, 1, \ldots, n$,
określona z dokładnością do niezerowego czynnika stałego
poprzez następujące dywizory:
\begin{equation}
\rdiv(r_{P,R}^{(m)}) = m\divi{P+R} - m\divi{R} - \divi{mP} + \divi{\ecident}
\end{equation}
\end{definition}

\noindent
Następujący fakt motywuje nazwę oraz pokazuje związek
cząstkowych wartości iloczynu Weila z samym iloczynem Weila.

\begin{fact}\label{rn_eq_fp_fact}
Zachodzi następująca zależność:
\begin{equation}
\rdiv(f_P'') = \rdiv(r_{P,R}^{(n)})
\end{equation}
\end{fact}

\noindent
Cząstkowe wartości iloczynu Weila
mają następującą kluczową własność.

\begin{lemma}\label{dbl_add_lemma}
Dane są cząstkowe wartości iloczynu Weila $r_{P,R}^{(m)}$
oraz liczby naturalne $k$ i $l$ takie,
że $0 \leq k, l \leq n$ i $0 \leq k+l \leq n$.
Niech $g$ będzie linią przechodzącą przez punkty $kP$ i $lP$,
a $h$ będzie linią pionową przechodzącą przez punkt $(k+l)P$.
Wówczas zachodzi następująca zależność:
\begin{equation}
\rdiv\left(r_{P,R}^{(k+l)}\right) =
\rdiv\left(r_{P,R}^{(k)}\right) + \rdiv\left(r_{P,R}^{(l)}\right) +
\rdiv(g) - \rdiv(h)
\end{equation}
\end{lemma}

\begin{proof}
Wystarczy sprawdzić dywizory obu stron równości:
\begin{eqnarray*}
\rdiv\left(r_{P,R}^{(k+l)}\right)
& = & (k+l)\divi{P+R} - (k+l)\divi{R} - \divi{(k+l)P} + \divi{\ecident} \\
& = & k\divi{P+R} - k\divi{R} - \divi{kP} + \divi{\ecident} + \\
&   & l\divi{P+R} - l\divi{R} - \divi{lP} + \divi{\ecident} + \\
&   & \divi{kP} + \divi{lP} - \divi{(k+l)P} - \divi{\ecident} \\
& = & \rdiv\left(r_{P,R}^{(k)}\right) + \rdiv\left(r_{P,R}^{(l)}\right) +
      \divi{kP} + \divi{lP} + \divi{-(k+l)P} - 3\divi{\ecident} - \\
&   & \divi{-(k+l)P} - \divi{(k+l)P} + 2\divi{\ecident} \\
& = & \rdiv\left(r_{P,R}^{(k)}\right) + \rdiv\left(r_{P,R}^{(l)}\right) +
      \rdiv(g) - \rdiv(h)
\end{eqnarray*}
\end{proof}

\begin{corollary}\label{dbl_add_coro}
Zachodzi następująca zależność:
\begin{equation}
r_{P,R}^{(k+l)} = r_{P,R}^{(k)} r_{P,R}^{(l)} \frac{g}{h}
\end{equation}
\end{corollary}

\noindent
Widzimy teraz, na czym opiera się algorytm Millera.
Lemat \ref{dbl_add_lemma} i płynący z niego wniosek \ref{dbl_add_coro}
sugerują algorytm typu ,,podwajaj-i-dodawaj'',
który pozwala obliczać wartości funkcji $r_{P,R}^{(n)}$.

\subsection*{Poprawność}

\noindent
Pokażemy teraz, że algorytm \ref{miller_alg} jest poprawny,
tzn. że rzeczywiście oblicza wartości ilczynu Weila
i że ma własność stopu.

\begin{lemma}
Procedura \proc{Combine-Partial-Values}
na podstawie wartości
$E$, $A$, $U = kP$, $V = lP$, $u = r_{P,R}^{(k)}(A)$ i $v = r_{P,R}^{(l)}(A)$
oblicza i przekazuje jako wynik
wartość $r_{P,R}^{(k+l)}(A)$.
\end{lemma}

\begin{proof}
Jest to jasne -- procedura ta jest bezpośrednim przełożeniem na pseudokod
wniosku \ref{dbl_add_coro}.
\end{proof}

\begin{remark}
Jest możliwe, że procedura Procedura \proc{Combine-Function-Values}
przekaże jako wynik specjalną stałą \const{error}.
Sytuację tę omówimy niebawem.
\end{remark}

\begin{lemma}\label{compute_value_correct_lemma}
Procedura \proc{Compute-Value} w algorytmie \ref{miller_alg}
na podstawie wartości
$E$, $n$, $P$, $R$ i $A$
oblicza i przekazuje jako wynik
wartość $r_{P,R}^{(n)}(A)$,
o ile nie kończy się błędem
(tzn. nie przekazuje jako wyniku stałej \const{error}).
\end{lemma}

\begin{proof}
Procedura ta jest typowym algorytmem typu ,,podwajaj-i-dodawaj''.
Jego poprawność łatwo udowodnić przez indukcję.
Będziemy rozpatrywać zapis liczby $n$ w postaci dwójkowej:
\begin{equation*}
n = \sum_{k = 0}^{\lfloor\log_2 n\rfloor} n_k2^k
\end{equation*}

\noindent
Jak łatwo sprawdzić,
przed rozpoczęciem pętli \kw{while}
zmienna $u$ ma wartość $r_{P,R}^{(0)}(A)$,
a zmienna $v$ ma wartość $r_{P,R}^{(1)}(A)$.

\noindent
Po zakończeniu $k$-tego przebiegu pętli
zmienna $u$ ma wartość $r_{P,R}^{(m_k)}(A)$,
gdzie $m_k = \sum_{l=0}^k n_l2^l$,
a zmienna $v$ ma wartość $r_{P,R}^{(2^k)}(A)$.

\noindent
Widać stąd, że po zakończeniu $(\lfloor\log_2 n\rfloor+1)$-szego kroku pętli
zmienna $u$ będzie miała żądaną wartość $r_{P,R}^{(n)}$.
\end{proof}

\begin{remark}
Istotą procedury \proc{Compute-Value} jest to,
że pozwala ona obliczyć wartość funkcji $r_{P,R}^{(n)}$ w zadanym punkcie $A$
bez konieczności obliczania wyrażenia wymiernego
określającego funkcję $r_{P,R}^{(n)}$.
Obliczenie tego wyrażenia byłoby możliwe --
wystarczy w procedurze \proc{Combine-Partial-Values}
zastąpić mnożenie elementów ciała skończonego
formalnym mnożeniem wyrażeń wymiernych.
Jest to jednak mniej efektywne podejście.
\end{remark}

\begin{remark}\label{return_error_case_remark}
Procedury \proc{Combine-Partial-Values} i \proc{Compute-Value}
mogą zakończyć się błędem, tzn. przekazać jako wynik stałą \const{error}.
Sytuacja ta występuje wtedy,
gdy pewne wartości wyliczone w tych procedurach są równe zero.
Kontynuowanie obliczeń w tej sytuacji nie ma sensu,
ponieważ doprowadziłoby to do mnożenia lub dzielenia przez zero
i nie byłoby wówczas możliwe poprawne obliczenie wartości iloczynu Weila,
ponieważ są one niezerowe.
Aby rozwiązać ten problem,
należałoby potencjalne mnożenie lub dzielenie przez zero
zastąpić śledzeniem krotności pojawiających się zer i biegunów
(co jest dosyć żmudne)
lub zamienić operacje arytmetyczne na elementach ciała skończonego
na operacje arytmetyczne na wyrażeniach wymiernych nad tym ciałem
(co nie jest efektywne).
\end{remark}

\begin{theorem}
Jeżeli procedura \proc{Weil-Pairing} w algorytmie \ref{miller_alg}
zatrzymuje się,
to jako wynik przekazuje wartość $w(P, Q)$.
\end{theorem}

\begin{proof}
Jest to jasne. Jeżeli żadna z wartości $a$, $b$, $c$, $d$
obliczanych w tej procedurze nie jest równa stałej \const{error},
to widzimy, że wyrażenie $(a/b)\cdot(d/c)$ przekazywane przez procedurę
faktycznie jest równe $w(P, Q)$,
co wynika z poprawności procedury \proc{Compute-Value}
(lemat \ref{compute_value_correct_lemma}),
faktu \ref{rn_eq_fp_fact}
i wzoru \ref{base_formula_eqn}.
\end{proof}

\begin{theorem}
Procedura \proc{Weil-Pairing} w algorytmie \ref{miller_alg}
prawie na pewno kończy się.
\end{theorem}

\begin{proof}
Nawet jeżeli punkty $R$ i $S$ wylosowane w procedurze \proc{Weil-Pairing}
są takie, że punkty $P$, $R$, $P+R$, $Q$, $S$, $Q+S$ i $\ecident$
są parami różne
i wyrażenie \ref{base_formula_eqn} jest dobrze określone,
to procedura \proc{Compute-Value} wciąż może zakończyć się błędem.
Dzieje się tak dlatego,
że podczas obliczania wartości \proc{Compute-Value}$(E, n, P, R, A)$,
gdzie $A = S$ lub $A = Q+S$,
punkt $A$ może pokryć się z jednym z punktów
występujących w procedurach
\proc{Combine-Partial-Values} i \proc{Compute-Value}.
Gdyby przeprowadzać obliczenia symbolicznie,
tzn. konstruować wyrażenia wymierne określające funkcje $r_{P,R}^{(m)}$,
to obliczone ostatecznie wyrażenie $r_{P,R}^{(n)}$
nie miałoby w punkcie $S$ ani $Q+S$ miejsca zerowego ani bieguna
i możnaby obliczyć wartość $r_{P,R}^{(n)}(A)$.
Procedura \proc{Compute-Value} nie prowadzi jednak obliczeń symbolicznych,
więc może wystąpić błąd opisany w uwadze \ref{return_error_case_remark}.

\noindent
Obliczenie wartości \proc{Compute-Value}$(E, n, P, R, A)$ powiedzie się,
jeżeli punkt $A$ nie pokryje się z żadnym z punktów
$R$, $P, 2P, 4P, 8P, \ldots, 2^{\lfloor\log_2 n\rfloor}P$,
$T_1, T_2, T_3, \ldots, T_{\lfloor\log_2 n\rfloor}$,
gdzie $T_k = \left(\sum_{l=0}^k n_l2^l\right)P$.
Punktów tych jest łącznie $O(\log n)$.

\noindent
Wniosek stąd, że obliczenie każdej z wartości $a$, $b$, $c$, $d$
w jednym przebiegu pętli \kw{while} w procedurze \proc{Weil-Pairing}
zakończy się błędem
z prawdopodobieństwem niewiększym niż $O(\frac{\log n}{q})$,
gdzie $q$ to rozmiar ciała skończonego, nad którym dana jest krzywa $E$.

\noindent
Przyjmijmy teraz, że $q = p^e$.
Jeżeli prawdopodobieństwo wystąpienia błędu jest zbyt duże,
możemy losować punkty $R$ i $S$ z odpowiednio większego ciała
o rozmiarze $q' = p^{e'}$, gdzie $e' = ef$ i $f > 1$.
Dla odpowiednio dużej wartości $q'$
prawdopodobieństwo wystąpienia błędu
stanie się mniejsze niż $\frac{1}{2}$,
co będzie oznaczać,
że procedura \proc{Weil-Pairing} prawie na pewno kończy się,
a oczekiwana liczba przebiegów pętli \kw{while} jest stała.
\end{proof}

\subsection*{Złożoność czasowa i pamięciowa}

\noindent
Oszacujemy teraz złożoność asymptotyczną algorytmu Millera.
Złożoność czasową będziemy mierzyć ilością niezbędnych operacji na bitach,
a pamięciową -- ilością wymaganych dodatkowych bitów pamięci.

\begin{remark}
Złożoność asymptotyczną będziemy mierzyć
w zależności od dwóch naturalnych w tej sytuacji parametrów:
rozmiaru $q$ ciała skończonego, nad którym dana jest krzywa eliptyczna
oraz liczby $n$ oznaczającej rząd punktów w podgrupie torsyjnej,
w której rozpatrujemy iloczyn Weila.
\end{remark}

\noindent
Rozpoczniemy od ustalenia złożoności operacji elementarnych
i procedur, które uznaliśmy za dane.

\begin{remark}
Złożoność asymptotyczna operacji uznanych za elementarne jest następująca:
\begin{itemize}
\item
addytywne operacje arytmetyczne na liczbach całkowitych wymagają
$O(\log n)$ operacji na bitach i $O(\log n)$ bitów pamięci;
\item
multiplikatywne operacje arytmetyczne na liczbach całkowitych wymagają
$O(\log^2 n)$ operacji na bitach i $O(\log n)$ bitów pamięci;
\item
porównania liczb całkowitych wymagają
$O(\log n)$ operacji na bitach i $O(1)$ bitów pamięci;
\item
addytywne operacje arytmetyczne na elementach ciała skończonego wymagają
$O(\log q)$ operacji na bitach i $O(\log q)$ bitów pamięci;
\item
multiplikatywne operacje arytmetyczne na elementach ciała skończonego wymagają
$O(\log^2 q)$ operacji na bitach i $O(\log q)$ bitów pamięci;
\item
porównania elementów ciała skończonego wymagają
$O(\log q)$ operacji na bitach i $O(1)$ bitów pamięci;
\item
porównania punktów na krzywej eliptycznej wymagają
$O(\log q)$ operacji na bitach i $O(1)$ bitów pamięci;
\item
operacje typu $p[o]$, czyli odczytanie cechy $p$ obiektu $o$, wymagają
$O(1)$ operacji na bitach i $O(1)$ bitów pamięci.
\end{itemize}
\end{remark}

\begin{remark}
Uzależnienie złożoności operacji na liczbach całkowitych od parametru $n$
wynika stąd,
że we wszystkich procedurach wykorzystywanych w algorytmie Millera
nie pojawiają się liczby większe niż $n$.
\end{remark}

\begin{remark}
Złożoność asymptotyczna procedur uznanych za dane jest następująca:
\begin{itemize}
\item
wykonanie procedury \proc{Random-Integer}
wymaga $O(\log n)$ operacji na bitach i $O(\log n)$ bitów pamięci;
\item
wykonanie procedury \proc{Random-Finite-Field-Element}
wymaga $O(\log q)$ operacji na bitach i $O(\log q)$ bitów pamięci;
\item
wykonanie procedury \proc{Finite-Field-Square-Root}
wymaga średnio $O(\log^3 q)$ operacji na bitach i $O(\log q)$ bitów pamięci;
\item
wykonanie procedury \proc{Curve-Finite-Point}
wymaga $O(\log q)$ operacji na bitach i $O(\log q)$ bitów pamięci;
\item
wykonanie procedury \proc{Curve-Point-Conjugate}
wymaga $O(\log q)$ operacji na bitach i $O(\log q)$ bitów pamięci;
\item
wykonanie procedury \proc{Line-On-Curve}
wymaga $O(\log q)$ operacji na bitach i $O(\log q)$ bitów pamięci;
\end{itemize}
\end{remark}

\noindent
Podamy teraz złożoność ogólnych procedur
zaprezentowanych w tym rozdziale.

\begin{fact}
Wykonanie procedury \proc{Line-Value-At-Finite-Point}
wymaga $O(\log^2 q)$ operacji na bitach i $O(\log q)$ bitów pamięci.
\end{fact}

\begin{fact}
Wykonanie procedury \proc{Vertical-Line-Through-Curve-Point}
wymaga \linebreak $O(\log q)$ operacji na bitach i $O(\log q)$ bitów pamięci.
\end{fact}

\begin{fact}
Wykonanie procedury \proc{Line-Through-Different-Curve-Finite-Points}
wymaga $O(\log^2 q)$ operacji na bitach i $O(\log q)$ bitów pamięci.
\end{fact}

\begin{fact}
Wykonanie procedury \proc{Tangent-Line-Curve-Finite-Point}
wymaga $O(\log^2 q)$ operacji na bitach i $O(\log q)$ bitów pamięci.
\end{fact}

\begin{fact}
Wykonanie procedury \proc{Line-Through-Curve-Points}
wymaga $O(\log^2 q)$ operacji na bitach i $O(\log q)$ bitów pamięci.
\end{fact}

\begin{fact}
Wykonanie procedury \proc{Add-Curve-Points}
wymaga $O(\log^2 q)$ operacji na bitach i $O(\log q)$ bitów pamięci.
\end{fact}

\begin{lemma}
Wykonanie procedury \proc{Multiply-Curve-Points}
wymaga $O(\log n \log^2 q)$ operacji na bitach
i $O(\log n + \log q)$ bitów pamięci.
\end{lemma}

\begin{proof}
Procedura ta realizuje schemat ,,podwajaj-i-dodawaj''.
Pojedyńczy przebieg pętli \kw{while} wymaga
$O(\log^2 q)$ operacji na bitach,
a przebiegów takich jest $O(\log n)$.
Procedura potrzebuje dodatkowej pamięci na stałą ilość
punktów krzywej i liczb całkowitych.
\end{proof}

\begin{lemma}
Wykonanie procedury \proc{Random-Curve-Finite-Point}
wymaga średnio \linebreak
$O(\log^3 q)$ operacji na bitach i $O(\log q)$ bitów pamięci.
\end{lemma}

\begin{proof}
Procedura ta jest zrandomizowana.
Obliczenie wartości $d$ w jednym przebiegu pętli
wymaga $O(\log^2 q)$ operacji na bitach.
Wykonanie pierwiastkowania, aby obliczyć wartość $b$,
wymaga średnio $O(\log^3 q)$ operacji na bitach.
Tak więc pojedyńczy przebieg pętli \kw{while}
wymaga średnio $O(\log^3 q)$ operacji na bitach.

\noindent
Policzmy oczekiwaną liczbę przebiegów potrzebną do znalezienia punktu krzywej.
Krzywa składa się z $q \pm O(\sqrt{q})$ punktów skończonych.
Jeśli pominąć punkty rzędu dwa,
to punkty skończone krzywej można połączyć w pary postaci
punkt i punkt sprzężony.
Z tego wniosek, że dla $\frac{q}{2} \pm O(\sqrt{q})$ elementów ciała $\K$
wartość $d$ będzie kwadratem pewnego elementu ciała.

\noindent
Prawdopodobieństwo zakończenia się pętli w danym kroku
wynosi zatem $\frac{1}{2} \pm O(\frac{1}{\sqrt{q}})$.
Stąd procedura ta prawie na pewno kończy się,
a oczekiwana liczba przebiegów pętli \kw{while} wynosi $2$,
zatem oczekiwana liczba wymaganych operacji na bitach
to średnio $O(\log^3 q)$.
Ponadto, procedura potrzebuje stałej ilości zmiennych pomocniczych,
w których przechowywane są elementy ciała,
stąd wymaga $O(\log q)$ bitów pamięci.
\end{proof}

\noindent
Przejdziemy teraz do obliczenia złożoności asymptotycznej algorytmu Millera.

\begin{fact}
Wykonanie procedury \proc{Combine-Partial-Values}
wymaga $O(\log^2 q)$ operacji na bitach i $O(\log q)$ bitów pamięci.
\end{fact}

\begin{theorem}
Wykonanie procedury \proc{Compute-Value}
wymaga $O(\log n \log^2 q)$ operacji na bitach
i $O(\log n + \log q)$ bitów pamięci.
\end{theorem}

\begin{proof}
Obliczenia wstępne przed pętlą \kw{while}
mają złożoność czasową $O(\log^2 q)$.
Każdy krok pętli \kw{while} również ma złożoność czasową $O(\log^2 q)$.
Procedura ta realizuje schemat ,,podwajaj-i-dodawaj'',
przebiegów pętli jest maksymalnie $O(\log n)$.
Stąd łączna złożoność czasowa to $O(\log n \log^2 q)$.
Procedura potrzebuje stałą ilość zmiennych pomocniczych,
które przechowują liczby całkowite i elementy ciała,
zatem potrzebuje $O(\log n + \log q)$ bitów pamięci.
\end{proof}

\begin{corollary}
Wykonanie jednego przebiegu pętli w procedurze \proc{Weil-Pairing}
wymaga $O((\log q + \log n) \log^2 q)$ operacji na bitach
i $O(\log q + \log n)$ bitów pamięci.
\end{corollary}

\begin{remark}
Oczekiwana liczba przebiegów pętli w procedurze \proc{Weil-Pairing}
zależy od relacji między wielkością parametrów $q$ i $n$.

\noindent
Jeżeli $q \geq n$, to oczekiwana liczba przebiegów pętli jest stała.
Prawdopodobieństwo wystąpienia błędu w jednym przebiegu pętli wynosi
$O(\frac{\log n}{q}) = O(\frac{\log q}{q}) < \frac{1}{2}$,
zatem oczekiwana liczba przebiegów pętli jest stała.

\noindent
Jeżeli $q < n$, to przeprowadzamy obliczenia w ciele $\GF(q')$,
a nie w ciele $\GF(q)$.
Wielkość $q'$ należy dobrać tak, aby zachodziła zależność $q' \geq n$.
Niech $q' = q^f$.
Wystarczy teraz przyjąć, że $f = \lceil\log_2 n\rceil$.
Wówczas $q' = q^f > 2^f = n$.
Oczekiwana liczba przebiegów pętli jest wtedy stała,
a jeden przebieg wymaga
$O((\log q' + \log n)\log^2 q') = O(\log^3 q \log^3 n)$ operacji na bitach
i $O(\log q' + \log n) = Q(\log q \log n)$ bitów pamięci.
\end{remark}
