\section{Opis implementacji}

\noindent
Częścią niniejszej pracy jest implementacja
opisanego w tym rozdziale algorytmu Millera,
którą teraz krótko omówimy.

\subsection*{Język programowania}

\noindent
Do implementacji został użyty język Python \cite{python}.
Podsumujmy krótko jego najważniejsze cechy.
\begin{itemize}
\item
Python ma charakterystyczną składnię.
Najbardziej rzuca się w oczy sposób oznaczania bloków kodu
(tzn. ciał pętli, treści procedur, gałęzi instrukcji warunkowych itd.) --
blok kodu jest ,,wcięty'',
tzn. zawiera większą liczbę białych znaków na początkach linii
niż kod otaczający.
\item
Python jest językiem interpretowanym,
tzn. instrukcje pythonowe nie są kompilowane do języka niższego poziomu
(np. do kodu maszynowego),
lecz każdorazowo parsowane i wykonywane bezpośrednio przez ,,interpreter''.
\item
Python pozwala stosować wiele paradygmatów programowania:
imperatywny, obiektowy i funkcyjny (częściowo).
\item
Python jest językiem dynamicznie typowanym,
tzn. wszelkiego rodzaju kontrola typów
(np. sprawdzanie, że argumentami operacji dodawania są liczby)
jest wykonywana w trakcie interpretowania programu.
\item
Python dostarcza mechanizm automatycznego zarządzania pamięcią
oparty o zliczanie referencji.
\item
Python dysponuje niezmiernie bogatą biblioteką standardową,
która oprócz usług dostępnych we wszystkich językach programowania
(dostęp do plików, niskopoziomowy dostęp do sieci itp.)
dostarcza mechanizmy służące
do prowadzenia wysokopoziomowej komunikacji sieciowej
(protokoły HTTP, SMTP, FTP itp.);
obsługi plików skompresowanych (TAR, GZIP, BZIP2);
tworzenia interfejsów graficznych (Tk);
prowadzenia komunikacji międzyprocesowej;
przetwarzania danych multimedialnych itd.
\item
Python może być uruchomiony na wielu systemach operacyjnych:
wszystkich nowszych wersjach systemu Windows firmy Microsoft,
praktycznie dowolnej dystrybucji systemu Linux,
a także na wielu innych systemach uniksowych,
w tym na Mac OS X firmy Apple.
\item
Nad rozwojem języka Python czuwa jego twórca Guido van Rossum
oraz społeczność skupiona wokół fundacji Python Software Foundation,
dzięki czemu proces wytwarzania kolejnych wersji języka
jest otwarty, przejrzysty i dostępny dla osób postronnych.
\item
Python udostępniany jest na licencji Python License,
która jest zgodna z wymogami OSI, jest zatem licencją wolną.
Oznacza to, że z Pythona można korzystać bezpłatnie
w praktycznie dowolnych zastosowaniach, w tym akademickich i komercyjnych.
\end{itemize}

\noindent
Wybór języka Python do zaimplementowania algorytmu Millera
podyktowany był następującymi względami:
\begin{itemize}
\item
Składnia języka Python jest niezwykle podobna
do zastosowanego w pracy sposobu zapisywania pseudokodu.
Dzięki temu wszystkie przedstawione algorytmy
można było bez wysiłku przetłumaczyć
na składnię Pythona.
\item
Elementem biblioteki standardowej Pythona
jest implementacja ,,długich'' liczb całkowitych,
tzn. liczb całkowitych o dowolnej ilości bitów.
Dzięki temu implementacja procedur była prostsza.
\item
Dynamiczne typowanie, automatyczne zarządzanie pamięcią
oraz brak konieczności kompilacji
pozwalają na bardzo łatwe pisanie prototypów w Pythonie.
Dzięki temu implementacja algorytmu Millera powstała bardzo szybko.
\end{itemize}

\subsection*{Struktura implementacji}

\noindent
Pełna komputerowa implementacja algorytmu Millera
obejmuje nie tylko przedstawione w tym rozdziale algorytmy,
ale także pewną ilość dodatkowych elementów.
Całość można podzielić na ,,warstwy'',
czyli takie części, że każda kolejna jest zależna od poprzedniej.
\begin{enumerate}
\item
Pierwszą warstwą jest implementacja liczb całkowitych dowolnej precyzji.
Koncepcją implementacji jest system pozycyjny,
w którym podstawę stanowi potęga liczby dwa, np. $2^{32}$ lub $2^{64}$.
Implementacji ,,długich'' liczb całkowitych
dostarcza biblioteka standardowa Pythona.
\item
Drugą warstwą jest implementacja ciał skończonych.
Operacje na elementach skończonych są zaimplementowane
wyłącznie za pomocą operacji na ,,długich'' liczbach całkowitych
oraz standardowych konstrukcji językowych Pythona.
Przyjęto obiektowy model reprezentacji danych.
Warstwę tę można podzielić na mniejsze warstwy.
\begin{itemize}
\item
Implementacja pierścienia liczb całkowitych polega na opakowaniu
dostępnych w Pythonie ,,długich'' liczb całkowitych w postać obiektowej.
\item
Implementacja pierścienia ilorazowego nad zadanym pierścieniem bazowym
polega na zaimplementowaniu w pierścieniu bazowym
operacji dzielenia z resztą oraz rozszerzonego algorytmu Euklidesa.
Operacje w pierścieniu ilorazowym polegają
na wykonywaniu operacji w pierścieniu bazowym modulo zadany element.
Odwrotności w pierścieniu ilorazowym obliczane są
za pomocą algorytmu Euklidesa.
\item
Implementacja pierścienia wielomianów nad zadanym pierścieniem bazowym
polega na reprezentowaniu wielomianów
za pomocą tablic elementów pierścienia bazowego.
\end{itemize}
Ciało skończone $\F(p)$
możemy przedstawić jako pierścień ilorazowy
pierścienia liczb całkowitych podzielonego przez $p$.
Ciało skończone $\GF(q)$, gdzie $q = p^e$,
przedstawiamy jako pierścień ilorazowy
pierścienia wielomianów nad ciałem $\F(p)$
podzielony przez wielomian stopnia $e$
nierozkładalny nad ciałem $\F(p)$.
\item
Trzecią warstwę stanowi implementacja prostego modelu obiektowego
reprezentującego krzywe eliptyczne oraz punkty i linie na krzywych.
Warstwa ta dostarcza jedynie reprezentację danych,
nie udostępnia natomiast żadnych operacji na tych danych.
\item
Czwartą i ostatnią warstwę
stanowi implementacja procedur przedstawionych w tym rodziale.
Tę warstwę również można podzielić na części.
\begin{itemize}
\item
Implementacja akcesorów (np. $x[P]$, $\id{identity}[E]$, $a[l]$)
polega ona na odpowiednim odwoływaniu się
do modelu obiektowego z warstwy trzeciej.
\item
Implementacja procedur uznanych za dane,
w tym konstruktorów
(np. \proc{Curve-Finite-Point},
\proc{Line-On-Curve}),
które korzystają z modelu obiektowego z warstwy trzeciej
oraz algorytmów
(np. \proc{Finite-Field-Element-Square-Root}).
\item
Implementacja ogólnych procedur przedstawionych w tym rozdziale
(np. \proc{Line-Through-Curve-Points},
\proc{Add-Curve-Points}).
\item
Implementacja algorytmu Millera,
w tym procedury \proc{Weil-Pairing}.
\end{itemize}
\end{enumerate}

\subsection*{Wydajność}

\noindent
Wydajność otrzymanej implementacji algorytmu Millera jest niska.
Fakt ten jest wywołany kilkoma czynnikami.

\noindent
Po pierwsze, zastosowany język Python oprócz wszystkich wymienionych zalet
ma też jedną zasadniczą wadę -- nie jest szybki.
Przyczyną tego stanu rzeczy jest przede wszystkim to,
że jest to język interpretowany.
W zależności od zastosowań może on być kilka do kilkudziesięciu razy
wolniejszy od języków kompilowanych, np. od C++.

\noindent
Po drugie, sposób zastosowany przy implementacji ciał skończonych
nie jest wydajny.
Przedstawienie ciał skończonych jako pierścieni ilorazowych powoduje,
że wykonanie pojedyńczej operacji arytmetycznej w ciele skończonym
przekłada się na wiele operacji w pierścieniach bazowych,
a w konsekwencji na jeszcze więcej operacji na liczbach całkowitych.
Mimo niskiej wydajności ten sposób reprezentacji został zastosowany,
ponieważ jest przejrzysty i ogólny.

\noindent
Po trzecie, sam algorytm Millera w postaci podanej w niniejszej pracy
charakteryzuje się dużą stałą w złożoności asymptotycznej.
Uważna analiza pozwala wyeliminować losowość z algorytmu (por. \cite{miller}).

\noindent
W związku z powyższym należy uznać, że stanowiąca część niniejszej pracy
implementacja algorytmu Millera ma charakter poglądowy
i nie nadaje się do zastosowań praktycznych.

\subsection*{System Sage}

\noindent
Narzędziem, które okazało się niezwykle pomocne podczas implementacji,
jest system Sage \cite{sage}.
Jest to wolny odpowiednik systemów
Mathematica \cite{mathematica} czy Maple \cite{maple}.
Oto niektóre z funkcji, które udostępnia:
\begin{itemize}
\item
przeprowadzanie obliczeń symbolicznych;
\item
symboliczne i numeryczne rozwiązywanie
różnego rodzaju równań i układów równań;
\item
wykonywanie operacji arytmetycznych
na elementach różnych obiektów algebraicznych --
ciał, pierścieni itp.;
\item
implementacja wielu algorytmów teorioliczbowych i kryptograficznych.
\end{itemize}
