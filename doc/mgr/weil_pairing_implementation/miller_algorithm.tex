\section{Algorytm Millera}

\noindent
Przedstawimy teraz pseudokod algorytmu Millera,
który oblicza wartości iloczynu Weila
na podstawie wzoru \ref{base_formula_eqn}.
Najpierw opiszemy zasady obowiązujące przy zapisywaniu pseudokodu,
a następnie podamy treść procedur pomocniczych
oraz właściwego algorytmu.

\begin{remark}
Aby uniknąć niejasności przy zapisywaniu pseudokodu,
przyjmujemy następujące ustalenia:
\begin{itemize}
\item pseudokod będziemy zapisywać w formie podobnej do tej
stosowanej w książce \cite{clrs};
\item operacje arytmetyczne na liczbach całkowitych
($+$, $-$, $\cdot$, $\kw{div}$, $\kw{mod}$)
uznajemy za elementarne;
\item porównania liczb całkowitych
($=$, $\neq$, $<$, $\leq$, $>$, $\geq$)
uznajemy za elementarne;
\item operacje arytmetyczne na elementach ciał skończonych
($+$, $-$, $\cdot$, $/$)
uznajemy za elementarne;
\item porównania elementów ciał skończonych
($=$, $\neq$)
uznajemy za elementarne;
\item porównania punktów na krzywej eliptycznej
($=$, $\neq$)
uznajemy za elementarne;
\item ustalamy, że notacja $p[o]$ oznacza odczytanie cechy $p$ obiektu $o$,
w szczególności:
\begin{itemize}
\item $\id{zero}[K]$, $\id{one}[K]$ itd.
oznaczają odpowiednio
zero, jedynkę itd. w ciele $K$;
\item $A[E]$, $B[E]$, $\id{field}[E]$, $\id{identity}[E]$
oznaczają odpowiednio
parametry krzywej eliptycznej $E$,
ciało, nad którym krzywa $E$ jest zdefiniowana
i punkt w nieskończoności krzywej $E$;
\item $x[P]$ i $y[P]$ oznaczają współrzędne skończonego punktu $P$
na krzywej eliptycznej;
\item $a[l]$ $b[l]$ i $c[l]$ oznaczają odpowiednio
współczynnik stojący przy zmiennej $x$,
współczynnik stojący przy zmiennej $y$
i wyraz wolny linii $l$ na krzywej eliptycznej;
\end{itemize}
\item zakładamy, że dysponujemy następującymi procedurami:
\begin{itemize}
\item \proc{Random-Integer}$(n)$, 
która wybiera losowo z rozkładem jednostajnym i przekazuje jako wynik
liczbę całkowitą $a$ z przedziału $[0; n)$, tzn. $0 \leq a < n$;
\item \proc{Random-Finite-Field-Element}$(K)$,
która wybiera losowo z rozkładem jednostajnym i przekazuje jako wynik
element ciała skończonego $K$;
\item \proc{Finite-Field-Element-Square-Root}$(K, a)$,
która oblicza i przekazuje jako \linebreak wynik
dowolny pierwiastek kwadratowy elementu $a \in K$
lub stałą \const{error}, jeśli taki pierwiastek nie istnieje;
\item \proc{Curve-Finite-Point}$(E, a, b)$,
która konstruuje i przekazuje jako wynik
punkt skończony na krzywej eliptycznej $E$ o współrzędnych $a$ i $b$;
\item \proc{Curve-Point-Conjugate}$(E, P)$,
która oblicza i przekazuje jako wynik
punkt sprzężony do punktu $P$ na krzywej eliptycznej $E$;
\item \proc{Line-On-Curve}$(E, a, b, c)$,
która konstruuje i przekazuje jako wynik
linię na krzywej eliptycznej $E$ o współczynnikach $a$, $b$ i $c$.
\end{itemize}
\end{itemize}
\end{remark}

\noindent
Ponieważ wybraliśmy dosyć ubogi zestaw operacji elementarnych,
przed opisem procedur związanych bezpośrednio z algorytmem Millera
musimy w postaci pseudokodu przedstawić
ogólne algorytmy związane z krzywymi eliptycznymi.
Część z nich to ujęte w postaci pseudokodu
wzory wyprowadzone wcześniej w pracy.

\noindent
Rozpoczniemy od procedury obliczającej
wartość linii w punkcie krzywej eliptycznej.
Jest ona trywialna, ale nie uznaliśmy jej za operację elementarną,
dlatego wypada ją zdefiniować.

\begin{algorithm}
Dana jest linia $l$ oraz punkt skończony $P$ na krzywej eliptycznej $E$.
Następująca procedura
na podstawie wartości $E$, $l$ i $P$
oblicza i przekazuje jako wynik
wartość $l(P)$:

\begin{codebox}
\Procname{$\proc{Line-Value-At-Curve-Finite-Point}(E, l, P)$}
\li \Assert $P \neq \id{identity}[E]$
\li \Return $a[l]\cdot x[P] + b[l]\cdot y[P] + c[l]$
\end{codebox}
\end{algorithm}

\noindent
Teraz podamy procedury wyznaczające linie przechodzące przez zadane punkty.
Najpierw rozważymy przypadki szczególne,
dla których wyprowadziliśmy wcześniej odpowiednie wzory.

\begin{algorithm}
Dany jest punkt $P$ na krzywej eliptycznej $E$.
Następująca procedura
na podstawie wartości $E$ i $P$
oblicza i przekazuje jako wynik
linię pionową przechodzącą przez punkt $P$:

\begin{codebox}
\Procname{$\proc{Vertical-Line-Through-Curve-Point}(E, P)$}
\li $K \gets \id{field}(E)$
\li \If $P = \id{identity}[E]$
\li     \Then
            \Return $\proc{Line-On-Curve}(E, \id{zero}[K], \id{zero}[K], \id{one}[K])$
\li     \Else
            $a \gets x[P]$
\li         \Return $\proc{Line-On-Curve}(E, \id{one}[K], \id{zero}[K], -a)$
        \End
\end{codebox}
\end{algorithm}

\begin{algorithm}
Dane są dwa różne punkty skończone $P$ i $Q$ na krzywej eliptycznej $E$.
Następująca procedura
na podstawie wartości $E$, $P$ i $Q$
oblicza i przekazuje jako wynik
linię przechodzącą przez punkty $P$ i $Q$:

\begin{codebox}
\Procname{$\proc{Line-Through-Different-Curve-Finite-Points}(E, P, Q)$}
\li \Assert $P \neq \id{identity}[E] \kw{ and } P \neq \id{identity}[E]$
\li \Assert $P \neq Q$
\li $K \gets \id{field}[E]$
\li $a \gets x[P]$
\li $b \gets y[P]$
\li $c \gets x[Q]$
\li $d \gets y[Q]$
\li \If $a = c$
\li     \Then
            \Return $\proc{Line-On-Curve}(E, \id{one}[K], \id{zero}[K], -a)$
\li     \Else
            $\lambda \gets (d-b)/(c-a)$
\li         \Return $\proc{Line-On-Curve}(E, \lambda, -\id{one}[K], -(\lambda\cdot a - b))$
        \End
\end{codebox}
\end{algorithm}

\begin{algorithm}
Dany jest punkt skończony $P$ na krzywej eliptycznej $E$.
Następująca procedura
na podstawie wartości $E$ i $P$
oblicza i przekazuje jako wynik
linię styczną do krzywej $E$ przechodzącą przez punkt $P$:

\begin{codebox}
\Procname{$\proc{Tangent-Line-Through-Curve-Finite-Point}(E, P)$}
\li \Assert $P \neq \id{identity}[E]$
\li $K \gets \id{field}[E]$
\li $a \gets x[P]$
\li $b \gets y[P]$
\li \If $b = \id{zero}[K]$
\li     \Then
            \Return $\proc{Line-On-Curve}(E, \id{one}[K], \id{zero}[K], -a)$
\li     \Else
            $\lambda \gets (\id{three}[K]\cdot a \cdot a + A[E])/(\id{two}[K]\cdot b)$
\li         \Return $\proc{Line-On-Curve}(E, \lambda, -\id{one}[K], -(\lambda\cdot a - b))$
        \End
\end{codebox}
\end{algorithm}

\noindent
Podane procedury zbierzemy teraz w jedną całość:
procedurę, która wyznacza linię przechodzącą przez dowolne dwa punkty krzywej.

\begin{algorithm}
Dane są punkty $P$ i $Q$ na krzywej eliptycznej $E$.
Następująca procedura
na podstawie wartości $E$, $P$ i $Q$
oblicza i przekazuje jako wynik
linię przechodzącą przez punkty $P$ i $Q$
(lub styczną do krzywej w punkcie $P$, gdy $P = Q$):

\begin{codebox}
\Procname{$\proc{Line-Through-Curve-Points}(E, P, Q)$}
\li $K \gets \id{field}[E]$
\li \If $P = \id{identity}[E] \kw{ and } Q = \id{identity}[E]$
\li     \Then
            \Return $\proc{Line-On-Curve}(E, \id{zero}[K], \id{zero}[K], \id{one}[K])$
        \End
\li \If $P = \id{identity}[E]$
\li     \Then
            \Return $\proc{Vertical-Line-Through-Curve-Point}(E, Q)$
        \End
\li \If $Q = \id{identity}[E]$
\li     \Then
            \Return $\proc{Vertical-Line-Through-Curve-Point}(E, P)$
        \End
\li \If $P \neq Q$
\li     \Then
            \Return $\proc{Line-Through-Different-Curve-Finite-Points}(E, P, Q)$
\li     \Else
            \Return $\proc{Tangent-Line-Through-Curve-Finite-Point}(E, P)$
        \End
\end{codebox}
\end{algorithm}

\noindent
Kolejne ogólne procedury to dodawanie dwóch punktów krzywej
oraz mnożenie punktu przez liczbę całkowitą.

\begin{algorithm}
Dane są punkty $P$ i $Q$ na krzywej eliptycznej $E$.
Następująca procedura
na podstawie wartości $E$, $P$ i $Q$
oblicza i przekazuje jako wynik
punkt $P + Q$:

\begin{codebox}
\Procname{$\proc{Add-Curve-Points}(E, P, Q)$}
\li $K \gets \id{field}[E]$
\li \If $P = \id{identity}[E]$
\li     \Then
            \Return $Q$
        \End
\li \If $Q = \id{identity}[E]$
\li     \Then
            \Return $P$
        \End
\li \If $P = \proc{Curve-Point-Conjugate}(E, Q)$
\li     \Then
            \Return $\id{identity}[E]$
        \End
\li $a \gets x[P]$
\li $b \gets y[P]$
\li $c \gets x[Q]$
\li $d \gets y[Q]$
\li \If $P \neq Q$
\li     \Then
            $\lambda \gets (d - b)/(c - a)$
\li     \Else
            $\lambda \gets (\id{three}[K]\cdot a\cdot a + A[E])/(\id{two}[K]\cdot b)$
        \End
\li $e \gets \lambda\cdot\lambda - a - c$
\li $f \gets -\lambda\cdot(e - a) - b$
\li \Return $\proc{Curve-Finite-Point}(E, e, f)$
\end{codebox}
\end{algorithm}

\begin{algorithm}
Dany jest punkt $P$ na krzywej eliptycznej $E$ oraz liczba całkowita $n$.
Następująca procedura
na podstawie wartości $E$, $P$ i $n$
oblicza i przekazuje jako wynik
punkt $nP$:

\begin{codebox}
\Procname{$\proc{Multiply-Curve-Point}(E, P, n)$}
\li $K \gets \id{field}[E]$
\li \If $P = \id{identity}[E]$
\li     \Then
            \Return $\id{identity}[E]$
        \End
\li \If $y[P] = \id{zero}[K]$
\li     \Then
            \If $n \kw{ mod } 2 = 0$
\li             \Then
                    \Return $\id{identity}[E]$
\li             \Else
                    \Return $P$
                \End
        \End
\li \If $n = 0$
\li     \Then
            \Return $\id{identity}[E]$
        \End
\li \If $n > 0$
\li     \Then
            $m \gets n$
\li     \Else
            $m \gets -n$
        \End
\li $R \gets \id{identity}[E]$
\li \While $m > 0$
\li     \Do
            \If $m \kw{ mod } 2 \neq 0$
\li             \Then
                    $R \gets \proc{Add-Curve-Points}(R, P)$
                \End
\li         $P \gets \proc{Add-Curve-Points}(P, P)$
\li         $m \gets m \kw{ div } 2$
        \End
\li \If $n > 0$
\li     \Then
            \Return $R$
\li     \Else
            \Return $\proc{Curve-Point-Conjugate}(E, R)$
        \End
\end{codebox}
\end{algorithm}

\noindent
Ostatnią ogólną procedurą jest losowanie punktu na krzywej eliptycznej.

\begin{algorithm}
Dana jest krzywa eliptyczna $E$ nad ciałem skończonym.
Następująca procedura
na podstawie wartości $E$
wybiera losowo z rozkładem jednostajnym i przekazuje jako wynik
punkt skończony na krzywej $E$:

\begin{codebox}
\Procname{$\proc{Random-Curve-Finite-Point}(E)$}
\li $K \gets \id{field}[E]$
\li \While $\const{true}$
\li     \Do
            $a \gets \proc{Random-Finite-Field-Element}(K)$
\li         $d \gets a \cdot a \cdot a + A[E] \cdot a + B[E]$
\li         \If $d = \id{zero}[K]$
\li             \Then
                    \If $\proc{Random-Integer}(2) = 0$
\li                     \Then
                            \Return $\proc{Curve-Finite-Point}(E, a, \id{zero}[K])$
                        \End
\li             \Else
                    $b \gets \proc{Finite-Field-Element-Square-Root}(K, d)$
\li                 \If $b \neq \const{error}$
\li                     \Then
                            \If $\proc{Random-Integer}(2) = 0$
\li                             \Then
                                    \Return $\proc{Curve-Finite-Point}(E, a, b)$
\li                             \Else
                                        \Return $\proc{Curve-Finite-Point}(E, a, -b)$
                                \End
                        \End
                \End
        \End
\end{codebox}
\end{algorithm}

\noindent
Możemy teraz przejść do opisania algorytmu Millera.

\begin{algorithm}\label{miller_alg}
Dane są punkty $P$ i $Q$ rzędu $n$
na krzywej eliptycznej $E$ nad ciałem skończonym.
Następujące procedury
na podstawie wartości $E$, $n$, $P$ i $Q$
obliczają i przekazują jako wynik
wartość $w(P, Q)$:

\begin{codebox}
\Procname{$\proc{Combine-Partial-Values}(E, A, U, V, u, v)$}
\li $K \gets \id{field}[E]$
\li $g \gets \proc{Line-Through-Curve-Points}(E, U, V)$
\li $h \gets \proc{Vertical-Line-Through-Curve-Point}(\proc{Add-Curve-Points}(E, U, V))$
\li $s \gets \proc{Line-Value-At-Curve-Finite-Point}(E, g, A)$
\li $t \gets \proc{Line-Value-At-Curve-Finite-Point}(E, h, A)$
\li \If $s = \id{zero}[K]$ \kw{or} $t = \id{zero}[K]$
\li     \Then \Return $\const{error}$
    \End
\li \Return $u \cdot v \cdot (s / t)$
\end{codebox}

\begin{codebox}
\Procname{$\proc{Compute-Value}(E, n, P, R, A)$}
\li $K \gets \id{field}[E]$
\li $g \gets \proc{Line-Through-Curve-Points}(E, P, R)$
\li $h \gets \proc{Vertical-Line-Through-Curve-Point}(E, \proc{Add-Curve-Points}(E, P, R))$
\li $s \gets \proc{Line-Value-At-Curve-Finite-Point}(E, g, A)$
\li $t \gets \proc{Line-Value-At-Curve-Finite-Point}(E, h, A)$
\li \If $s = \id{zero}[K] \kw{ or } t = \id{zero}[K]$
\li \Then
        \Return $\const{error}$
    \End
\li $U \gets \id{identity}[E]$
\li $V \gets P$
\li $u \gets \id{one}[K]$
\li $v \gets t / s$
\li \While $n > 0$
\li     \Do
            \If $v = \const{error}$
\li             \Then
                    \Return $\const{error}$
                \End
\li         \If $n \kw{ mod } 2 \neq 0$
\li             \Then
                    $u \gets \proc{Combine-Partial-Values}(E, A, U, V, u, v)$
\li                 $U \gets \proc{Add-Curve-Points}(U, V)$
\li                 \If $u = \const{error}$
\li                     \Then
                            \Return $\const{error}$
                        \End
                \End
\li         $v \gets \proc{Combine-Partial-Values}(E, A, V, V, v, v)$
\li         $V \gets \proc{Add-Curve-Points}(E, V, V)$
\li         $n \gets n \kw{ div } 2$
        \End
\li \Return $u$
\end{codebox}

\begin{codebox}
\Procname{$\proc{Weil-Pairing}(E, n, P, Q)$}
\li \Assert $\proc{Multiply-Curve-Point}(E, P, n) = \id{identity}[E]$
\li \Assert $\proc{Multiply-Curve-Point}(E, Q, n) = \id{identity}[E]$
\li $K \gets \id{field}[E]$
\li \If $P = Q \kw{ or } P = \id{identity}[E] \kw{ or } Q = \id{identity}[E]$
\li     \Then
            \Return $\id{one}[K]$
        \End
\li \While $\const{true}$
\li     \Do
            $R \gets \proc{Random-Curve-Finite-Point(E)}$
\li         $S \gets \proc{Random-Curve-Finite-Point(E)}$
\li         $R' \gets \proc{Add-Curve-Points}(E, P, R)$
\li         $S' \gets \proc{Add-Curve-Points}(E, Q, S)$
\li         \If $R' \neq \id{identity}[E] \kw{ and } S' \neq \id{identity}[E]$
\li             \Then
                    $a \gets \proc{Compute-Value}(E, n, P, R, S')$
\li                 $b \gets \proc{Compute-Value}(E, n, P, R, S)$
\li                 $c \gets \proc{Compute-Value}(E, n, Q, S, R')$
\li                 $d \gets \proc{Compute-Value}(E, n, Q, S, R)$
\li                 \If $a \neq \const{error} \kw{ and } b \neq \const{error} \kw{ and } c \neq \const{error} \kw{ and } d \neq \const{error}$
\li                     \Then
                            \Return $(a / b)\cdot(d / c)$
                        \End
                \End
        \End
\end{codebox}
\end{algorithm}
