\section{Podstawowy wzór}

Algorytm będzie obliczał wartości iloczynu Weila
na podstawie definicji alternatywnej.
Niestety, fragmentem wzoru \ref{weil_pairing_alt_eqn}
jest wyrażenie $\frac{g_Q}{g_P}(\ecident)$,
które zawiera elementy niewygodne z obliczeniowego punktu widzenia --
miejsca zerowe i bieguny oraz punkt w nieskończoności.
Dlatego ,,przesuniemy'' licznik i mianownik
we wzorze \ref{weil_pairing_alt_eqn} tak,
aby operować tylko na wartościach skończonych.

\begin{definition}
Dana jest krzywa eliptyczna $E$ oraz punkty $P, Q \in E[n]$.
Niech $R, S \in E$ będą dowolnymi punktami krzywej takimi,
że punkty $P$, $Q$, $R$, $S$, $P+R$, $Q+S$ i $\ecident$ są parami różne.
Funkcje wymierne $\widehat{g_P}, \widehat{g_Q} \in K(E)$
określamy poprzez podanie ich dywizorów
\begin{eqnarray*}
\rdiv(\widehat{g_P}) & = & n\divi{P+R} - n\divi{R} \\
\rdiv(\widehat{g_Q}) & = & n\divi{Q+S} - n\divi{S}
\end{eqnarray*}
Określamy funkcję $w''(P, Q)\colon E[n] \to \K$
następującym wzorem:
\begin{equation}\label{base_formula_eqn}
w''(P, Q) = (-1)^n\frac{\widehat{g_P}(Q+S)}{\widehat{g_P}(S)}
            \frac{\widehat{g_Q}(R)}{\widehat{g_Q}(P+R)}
\end{equation}
\end{definition}

\begin{remark}
Podobnie jak w przypadku funkcji $g_P$ i $g_Q$,
funkcje $\widehat{g_P}$ i $\widehat{g_Q}$
są określone z dokładnością do czynnika stałego.
Nie wpływa to na wartość funkcji $w''(P, Q)$.
\end{remark}

\begin{remark}
Ze względu na wybór punktów $R$ i $S$
we wzorze \ref{base_formula_eqn} nie pojawia się zero ani nieskończoność.
\end{remark}

Pokażemy teraz, że funkcje $w'(P, Q)$ i $w''(P, Q)$ są równe.
Głównym narzędziem użytym w dowodzie będzie prawo wzajemności Weila.

\begin{theorem}
Dana jest krzywa eliptyczna $E$ oraz punkty $P, Q \in E[n]$.
Wówczas:
\begin{equation}
w'(P, Q) = w''(P, Q)
\end{equation}
\end{theorem}

\begin{proof}
Z postaci dywizorów określających funkcje $\widehat{g_P}$ i $\widehat{g_Q}$
łatwo zauważyć,
że $\widehat{g_P} = g_P \circ t_R$ oraz $\widehat{g_Q} = g_Q \circ t_S$.
Potrzebna nam będzie jednak inna reprezentacja.
Z lematu \ref{dupa} wiemy,
że dywizor $\divi{P+R} - \divi{P} - \divi{R} + \divi{\ecident}$ jest główny,
podobnie dywizor $\divi{Q+S} - \divi{Q} - \divi{S} + \divi{\ecident}$.
Niech $h_P$ i $h_Q$ będą określone przez te dywizory, tzn.:
\begin{eqnarray*}
\rdiv(h_P) & = & \divi{P+R} - \divi{P} - \divi{R} + \divi{\ecident} \\
\rdiv(h_Q) & = & \divi{Q+S} - \divi{Q} - \divi{S} + \divi{\ecident}
\end{eqnarray*}
Można teraz sprawdzić,
że $\widehat{g_P} = g_Ph_P^n$ oraz $\widehat{g_Q} = g_Qh_Q^n$.

Zastosujmy prawo wzajemności Weila do funkcji $g_P$ i $h_Q$.
Wykorzystamy przy tym własności symbolu lokalnego
oraz to, że punkty $P$, $Q+S$, $Q$, $S$ i $\ecident$ są parami różne.
Otrzymujemy:
\begin{eqnarray}
1
& = & \prod_{T \in E} \lsym{g_P}{h_Q}{T}
\nonumber \\
& = & \lsym{g_P}{h_Q}{Q+S}\lsym{g_P}{h_Q}{Q}\lsym{g_P}{h_Q}{S}
      \lsym{g_P}{h_Q}{P}\lsym{g_P}{h_Q}{\ecident}
\nonumber \\
& = & \frac{1}{h_Q^n(P)}\frac{g_P(Q+S)}{g_P(Q)g_P(S)}
      (-1)^{-n}\left(\frac{g_P}{h_Q^{-n}}\right)(\ecident)
\end{eqnarray}
W analogiczny sposób stosujemy prawo wzajemności Weila do funkcji
$h_P$ i $g_Q$. Otrzymujemy:
\begin{eqnarray}
1
& = & \prod_{T \in E} \lsym{h_P}{g_Q}{T} \nonumber \\
& = & \lsym{h_P}{g_Q}{P+R}\lsym{h_P}{g_Q}{P}\lsym{h_P}{g_Q}{R}
      \lsym{h_P}{g_Q}{Q}\lsym{h_P}{g_Q}{\ecident} \nonumber \\
& = & h_P^n(Q)\frac{g_Q(P)g_Q(R)}{g_Q(P+R)}
      (-1)^{-n}\left(\frac{h_P^{-n}}{g_Q}\right)(\ecident)
\end{eqnarray}
Stosujemy prawo wzajemności Weila jeszcze raz do funkcji $h_P$ i $h_Q$:
\begin{eqnarray}
1
& = & \prod_{T \in E} \lsym{h_P}{h_Q}{T}
\nonumber \\
& = & \lsym{h_P}{h_Q}{P+R}
      \lsym{h_P}{h_Q}{P}
      \lsym{h_P}{g_Q}{R}
\nonumber \\
&   & \lsym{h_P}{h_Q}{Q+S}
      \lsym{h_P}{g_Q}{Q}
      \lsym{h_P}{h_Q}{S}
      \lsym{h_P}{h_Q}{\ecident}
\nonumber \\
& = & \frac{h_P(Q+S)}{h_P(S)}
      \frac{h_Q(R)}{h_Q(P+R)}
      \frac{h_P(Q)}{h_Q(P)}
      \left(\frac{h_P}{h_Q}\right)(\ecident)
\end{eqnarray}
Ostatnią równość podnosimy do $n$-tej potęgi i otrzymujemy:
\begin{equation}
1 =
\frac{h_P^n(Q+S)}{h_P^n(S)}
\frac{h_Q^n(R)}{h_Q^n(P+R)}
\frac{h_P^n(Q)}{h_Q^n(P)}
\left(\frac{h_P^n}{h_Q^n}\right)(\ecident)
\end{equation}

Lewe strony równości \ref{dupa1}, \ref{dupa2} i \ref{dupa3}
możemy domnożyć do wzoru \ref{weil_pairing_one_eqn}
nie zmieniając wyniku.
Otrzymujemy:
\begin{eqnarray*}
w'(P,Q)
& = & \frac{g_P(Q)}{g_Q(P)}(-1)^n\left(\frac{g_Q}{g_P}\right)(\ecident) \\
& = & \frac{g_P(Q)}{g_Q(P)}(-1)^n\left(\frac{g_Q}{g_P}\right)(\ecident) \\
&   & \frac{1}{h_Q^n(P)}\frac{g_P(Q+S)}{g_P(Q)g_P(S)}
      (-1)^{-n}\left(\frac{g_P}{h_Q^{-n}}\right)(\ecident) \\
&   & h_P^n(Q)\frac{g_Q(P)g_Q(R)}{g_Q(P+R)}
      (-1)^{-n}\left(\frac{h_P^{-n}}{g_Q}\right)(\ecident) \\
&   & \frac{h_P^n(Q+S)}{h_P^n(S)}
      \frac{h_Q^n(R)}{h_Q^n(P+R)}
      \frac{h_P^n(Q)}{h_Q^n(P)}
      \left(\frac{h_P^n}{h_Q^n}\right)(\ecident) \\
& = & \frac{g_P(Q+S)h_P^n(Q+S)}{g_P(S)h_P^n(S)}
      \frac{g_Q(R)h_{Q,S}(R)}{g_Q(P+R)h_Q(P+R)}
      (-1)^n\left(
      \frac{g_Pg_Qh_P^nh_P^{-n}}{g_Qg_Ph_Q^nh_Q^{-n}}
      \right)(\ecident) \\
& = & (-1)^n\frac{\widehat{g_P}(Q+S)}{\widehat{g_P}(S)}
      \frac{\widehat{g_Q}(R)}{\widehat{g_Q}(P+R)} \\
& = & w''(P,Q)
\end{eqnarray*}
\end{proof}
