\section{Szyfrowanie oparte na tożsamości}

\noindent
Opiszemy teraz system szyfrujący
opracowany przez Boneha i Franklina \cite{bonehfranklin}.
Iloczyn Weila odgrywa w tym systemie kluczową rolę.

\noindent
System pozwala szyfrować wiadomości w sposób asymetryczny,
tzn. występują w nim klucze publiczne do szyfrowania wiadomości
oraz klucze prywatne do ich odczytywania.
W skład systemu wchodzą cztery algorytmy,
których przeznaczenie jest następujące.

\begin{itemize}
\item
Algorytm tworzenia parametrów systemu
jest wykorzystywany przez zarządcę systemu raz,
podczas tworzenia instancji systemu.
Wynikiem jego działania są parametry systemu,
które zarządca powinien udostępnić
wszystkim użytkownikom chcącym korzystać z systemu
oraz klucz główny, którego nie powinien ujawniać.

\item
Algorytm tworzenia klucza prywatnego
jest wykorzystywany przez zarządcę wtedy,
gdy nowy użytkownik chce rozpocząć korzystanie z systemu.
Za pomocą tego algorytmu zarządca
tworzy dla nowego użytkownika klucz prywatny,
którego użytkownik nie powinien ujawniać.
Klucz prywatny jest wyznaczany jednoznacznie
na podstawie parametrów systemu, klucza głównego
i dowolnie wybranego ciągu bitów.
Ciąg ten będzie identyfikatorem nowego użytkownika,
który inni użytkownicy będą wykorzystywać,
aby wysyłać do niego zaszyfrowane wiadomości.

\item
Wiadomości są szyfrowane za pomocą algorytmu wykorzystującego
parametry systemu i identyfikator odbiorcy.
Dodatkowo, w procesie szyfrowania używany jest losowo wybrany parametr,
dlatego w wyniku wielokrotnego zaszyfrowania tej samej wiadomości
można otrzymać różne kryptogramy.

\item
Kryptogramy odczytywane są za pomocą algorytmu wykorzystującego
parametry systemu i klucz prywatny odbiorcy.
\end{itemize}

\noindent
Jak widać, kluczową cechą odróżniającą ten system od innych rozwiązań
jest możliwość użycia dowolnego ciągu bitów jako klucza publicznego.
O systemie mającym tę cechę mówimy, że jest ,,oparty na tożsamości''.
Takie podejście, jak zaraz zobaczymy, oferuje możliwości,
których nie dają inne kryptosystemy.

\subsection*{Motywacja}

\noindent
Przedstawiamy dwa typy problemów,
których rozwiązanie za pomocą kryptosystemów
nieopierających się na tożsamości
jest albo niemożliwe,
albo bardzo trudne.
Natomiast łatwo jest rozwiązać je za pomocą systemu Boneha-Franklina.

\begin{itemize}
\item
Dana jest grupa użytkowników (np. instytucja państwowa lub korporacja),
którzy chcą przesyłać między sobą zaszyfrowane wiadomości.
Zastosowanie w tym celu systemu Boneha-Franklina daje następujące korzyści.

\begin{itemize}
\item
Użytkownicy nie muszą przechowywać kluczy publicznych innych użytkowników.
Ułatwia to nawiązanie korespondencji z nową osobą,
bo łatwiej jest przekazać innym kanałem (np. za pomocą wizytówki)
swój identyfikator (np. adres poczty elektronicznej)
niż swój klucz publiczny.
Dalej, nie ma ryzyka utracenia listy kluczy publicznych innych użytkowników.
Listę taką zawsze można odtworzyć, ale może być to bardziej żmudne
niż odtworzenie listy adresów poczty elektronicznej.

\item
Użytkownicy nie muszą przechowywać nawet swojego klucza prywatnego --
zawsze można ponownie poprosić zarządcę systemu o jego wygenerowanie
(uprzednio potwierdzając swoją tożsamość innymi metodami).
Utrata klucza prywatnego w innych systemach może spowodować
nieodwracalne konsekwencje.

\item
Można wysyłać wiadomości do przyszłych członków grupy,
dla których nie został jeszcze utworzony klucz prywatny.
Wystarczy, że znany jest identyfikator, którym będą się posługiwać.

\item
Łatwo jest wykluczać użytkowników z grupy.
Wystarczy, że kluczem publicznym będzie ciąg bitów będący wynikiem
połączenia identyfikatora odbiorcy z rokiem (odpowiednio, rokiem i miesiącem)
wysłania wiadomości.
W ten sposób każdy członek grupy musi co rok (odpowiednio, co miesiąc)
poprosić zarządcę o wygenerowanie nowego klucza prywatnego,
który będzie ważny przez kolejny rok (odpowiednio, miesiąc).

\item
Można wysyłać wiadomości, które będą mogły być odczytane dopiero w przyszłości.
Wystarczy zmodyfikować poprzedni pomysł:
zamiast bieżącej daty można do identyfikatora odbiorcy
dokleić datę przyszłą, która określa,
kiedy wiadomość będzie mogła być odczytana.

\item
Członkom grupy można nadawać poziomy uprawnień do odczytywania wiadomości.
W tym celu kluczem publicznym powinien być ciąg bitów składający się
z identyfikatora odbiorcy, daty oraz nazwy uprawnienia,
które jest wymagane do odczytu nadawanej wiadomości.
Jest to uogólnienie poprzednich dwóch pomysłów.
\end{itemize}

\noindent
Ponieważ nie trzeba przechowywać kluczy publicznych,
szyfrowanie elektronicznej korespondencji staje się bardziej dostępne
dla przeciętnego użytkownika poczty elektronicznej.
Zauważmy przy tym, że nierozwiązany pozostaje problem przesyłania
zaszyfrowanych wiadomości między członkami różnych grup,
np. między użytkownikami dwóch różnych serwerów poczty elektronicznej.
Dlatego też nie można do końca zrezygnować
z infrastruktury klucza publicznego.

\noindent
Zauważmy, że zastosowanie systemu Boneha i Franklina w opisany sposób
prowadzi do następującego problemu: zarządca systemu może odczytać
dowolną zaszyfrowaną wiadomość.
Z kryptograficznego punktu widzenia jest to zjawisko zdecydowanie niepożądane,
jednak obecnie praktykuje się właśnie takie rozwiązania:
administrator serwera poczty elektronicznej ma dostęp do wszystkich wiadomości.

\item
Przypuśćmy, że pewien użytkownik
systemu opartego na infrastrukturze klucza publicznego
chce zabezpieczyć się przed ujawnieniem swojego klucza prywatnego.
Sytuacja taka może mieć miejsce,
jeżeli będzie przechowywał klucz prywatny na niezaufanych urządzeniach
(np. laptopie lub telefonie komórkowym, które mogą zostać skradzione)
lub powierzy go niezaufanym osobom
(np. swoim asystentom, których zadaniem jest
pomoc w prowadzeniu korespondencji).
W tym celu może wykorzystać system Boneha-Franklina.
Powinien utworzyć własną instancję systemu
i poprosić wszystkich, którzy wysyłają do niego wiadomości,
żeby szyfrowali je kluczem,
który powstaje z połączenia daty wysłania wiadomości
z kategorią tematyczną wiadomości.
Opisane problemy można wówczas rozwiązać następująco.

\begin{itemize}
\item
Użytkownik ma dostęp do całej swojej korespondencji,
ponieważ jest w posiadaniu klucza głównego
i może z jego pomocą odszyfrować dowolną wiadomość.

\item
Swoim asystentom może wydać klucze prywatne,
które pozwalają na odczytanie wiadomości jedynie z tej kategorii tematycznej,
za którą są odpowiedzialni.

\item
Na zagrożone kradzieżą urządzenie
użytkownik może nagrać klucze prywatne pozwalające na odczytywanie wiadomości
wysłanych tylko w zadanym okresie czasu (np. jeden tydzień).
Jeżeli urządzenie zostanie skradzione,
złodziej nie będzie w stanie odczytać
wiadomości wysłanych po upływie zadanego okresu.
\end{itemize}
\end{itemize}

\subsection*{Szczegółowy opis działania systemu}

\noindent
Przedstawimy teraz działanie wszystkich algorytmów
wchodzących w skład systemu Boneha-Franklina.
Opiszemy również rodzaje danych pojawiających się w systemie.

\begin{algorithm}[Tworzenie parametrów systemu]
Następujący algorytm wybiera i przekazuje jako wynik
parametry instancji systemu Boneha-Franklina
i jej klucz główny.

\begin{codebox}
\Procname{$\proc{BF-Create-Instance}()$}
\li
Wybieramy liczbę pierwszą $q$.
\li
Wybieramy grupy cykliczne $\G_1$ i $\G_2$ rzędu $q$.
\li
Wybieramy odwzorowanie dwuliniowe $b\colon \G_1 \times \G_1 \to G_2$.
\li
Wybieramy liczbę naturalną $m$.
\li
Ustalamy, że przestrzenią wiadomości $\mathcal{M}$
jest zbiór $\{0, 1\}^m$.
\li
Ustalamy, że przestrzenią kryptogramów $\mathcal{C}$
jest zbiór $\G_1^\star \times \mathcal{M}$.
\li
Ustalamy, że przestrzenią identyfikatorów użytkowników $\mathcal{I}$
jest zbiór $\{0, 1\}^\star$.
\li
Wybieramy funkcje haszujące
$H_1\colon \mathcal{I} \to \G_1^\star$
i $H_2\colon \G_2 \to \mathcal{M}$.
\li
Wybieramy generator $P$ grupy $\G_1$.
\li
Wybieramy element $s$ z grupy $(\Z / q\Z)^\star$.
\li
Obliczamy wartość $Q = sP$.
\li
Parametry systemu to krotka
$\left\langle q, \G_1, \G_2, b, m, H_1, H_2, P, Q \right\rangle$.
\li
Klucz główny to element $s$.
\end{codebox}
\end{algorithm}

\begin{algorithm}[Tworzenie klucza prywatnego]
Dany jest identyfikator $\const{id}$.
Następujący algorytm na podstawie wartości $\const{id}$
oraz klucza głównego $s$ instancji systemu Boneha-Franklina i jej parametrów
oblicza i przekazuje jako wynik
klucz prywatny odpowiadający identyfikatorowi $\const{id}$.

\begin{codebox}
\Procname{$\proc{BF-Create-Private-Key}(
    \const{id},
    s,
    \left\langle q, \G_1, \G_2, b, m, H_1, H_2, P, Q \right\rangle
)$}
\li
Obliczamy wartość $R = H_1(\const{id})$.
\li
Obliczamy wartość $S = sR$.
\li
Klucz prywatny to punkt $S$.
\end{codebox}
\end{algorithm}

\begin{algorithm}[Szyfrowanie wiadomości]
Dana jest wiadomość $M$ i identyfikator adresata $\const{id}$.
Następujący algorytm na podstawie wartości $M$ i $\const{id}$
oraz parametrów instancji systemu Boneha-Franklina
oblicza i przekazuje jako wynik
kryptogram odpowiadający wiadomości $M$.

\begin{codebox}
\Procname{$\proc{BF-Encrypt}(
    M,
    \const{id},
    \left\langle q, \G_1, \G_2, b, m, H_1, H_2, P, Q \right\rangle
)$}    
\li
Obliczamy wartość $R = H_1(\const{id})$.
\li
Wybieramy element $r$ z grupy $(\Z / q\Z)^\star$.
\li
Obliczamy wartość $U = rP$.
\li
Obliczamy wartość $V = M\ \kw{xor}\ b(R, Q)^r$.
\li
Kryptogram to para
$\left\langle U, V \right\rangle$.
\end{codebox}
\end{algorithm}

\begin{algorithm}[Odczytywanie kryptogramu]
Dany jest kryptogram $C$ w postaci $C = \left\langle U, V \right\rangle$
i klucz prywatny $S$ odpowiadający identyfikatorowi $\const{id}$.
Następujący algorytm
na podstawie wartości $C$ i $S$
oraz parametrów instancji systemu Boneha-Franklina
oblicza i przekazuje jako wynik
wiadomość odpowiadającą kryptogramowi $C$.

\begin{codebox}
\Procname{$\proc{BF-Decrypt}(
    \left\langle U, V \right\rangle,
    S,
    \left\langle q, \G_1, \G_2, b, m, H_1, H_2, P, Q \right\rangle
)$}
\li
Obliczamy wartość $M = V\ \kw{xor}\ b(S, U)$.
\li
Wiadomość to wartość $M$.
\end{codebox}
\end{algorithm}

\begin{remark}
Zauważmy, że nie sprecyzowaliśmy, jaka jest struktura grup $\G_1$ i $\G_2$.
Kwestię tę omówimy za chwilę.
\end{remark}

\noindent
Pokażmy przede wszystkim, że w systemie Boneha-Franklina
prawidłowo działa szyfrowanie i odczytywanie wiadomości.

\begin{theorem}
Dana jest wiadomość $M$,
identyfikator adresata $\const{id}$,
odpowiadający mu klucz prywatny $S$,
i parametry $\mathcal{P}$ instancji systemu Boneha-Franklina.
Wówczas zachodzi następująca zależność:
\begin{equation*}
\proc{BF-Decrypt}(
    \proc{BF-Encrypt}(
        M,
        \const{id},
        \mathcal{P}
    ),
    S,
    \mathcal{P}
) = M
\end{equation*}
\end{theorem}

\begin{proof}
Wiadomość $M$ jest szyfrowana symetrycznie za pomocą operacji \kw{xor}.
Wystarczy zatem sprawdzić,
że klucze użyte przy szyfrowaniu i odczytywaniu wiadomości są takie same.
Istotnie:
\begin{eqnarray*}
b(S, U)
& = & b(sR, U) \\
& = & b(R, U)^s \\
& = & b(R, rP)^s \\
& = & b(R, P)^{sr} \\
& = & b(R, sP)^r \\
& = & b(R, Q)^r
\end{eqnarray*}
\end{proof}

\noindent
Sednem sposobu szyfrowania w systemie Boneha-Franklina
jest wykorzystanie prostego algorytmu symetrycznego do szyfrowania wiadomości,
stosowanie w tym algorytmie kluczy jednorazowych
oraz sprytny sposób przekazania informacji o użytym kluczu.
Sekretną informacją potrzebną do odczytania wiadomości
jest losowa wartość $r$, którą nadawca ,,rozdziela'' na dwie części.
Jedynie odbiorca jest w stanie połączyć je z powrotem w całość
i odczytać wiadomość.
Złączenie to odbywa się za pomocą operacji dwuliniowej $b$.
Sposób, w jaki sekret jest rozdzielony i przekazany w częściach,
koncepcyjnie przypomina działanie protokołu Diffiego-Hellmana.

\noindent
Konkretna realizacja systemu Boneha-Franklina
wymaga wybrania konkretnych grup $\G_1$ i $\G_2$
oraz odwzorowania dwuliniowego $b$.
W swojej pracy Boneh i Franklin opisują następującą realizację systemu
opartą na superosobliwych krzywych eliptycznych.
\begin{itemize}
\item
Podstawą systemu jest superosobliwa krzywa eliptyczna $E_{0,1}(\F(p))$,
przy czym $q \mid p+1$.
Zauważmy, że w tej sytuacji $E[q] \subset E_{0,1}(\GF(p^2))$.
\item
Rolę grupy $\G_1$ pełni grupa generowana
przez punkt $P$ rzędu $q$ na krzywej $E$.
\item
Grupa $\G_2$ to grupa pierwiastków $q$-tego stopnia z jedności
w ciele $\GF(p^2)$.
\item
Niech $\phi$ będzie funkcją określoną
równaniem \ref{supersingular_curve_automorphism_eqn}.
Odwzorowanie dwuliniowe $b$ używane w systemie
jest określone następująco:
\begin{equation}\label{modified_weil_pairing_eqn}
b(P, Q) = w(P, \phi(Q))
\end{equation}
\end{itemize}

\begin{remark}
Zauważmy, że iloczyn Weila jest zdegenerowany na krzywej $E_{0,1}(\F(p))$,
dlatego w definicji funkcji $b$ jeden z argumentów jest zmodyfikowany
za pomocą automorfizmu $\phi$.
Można sprawdzić, że uzyskane w ten sposób odwzorowanie jest dwuliniowe.
\end{remark}

\noindent
W pracy Boneha i Franklina można znaleźć analizę bezpieczeństwa systemu.
Zaznaczmy, że aby uzyskać gwarancje bezpieczeństwa na tyle rozsądne,
żeby można było korzystać z systemu w praktyce,
należy zastosować zmodyfikowaną wersję opisanego tutaj schematu.
Opis tej modyfikacji można znaleźć w pracy \cite{fujisakiokamoto}.
