\section{Redukcja MOV}

Pierwsze zastosowanie iloczynu Weila
związane jest
z problemem logarytmu dyskretnego oraz protokołem Diffiego-Hellmana.

Podajmy najpierw definicje obu problemów.

\begin{problem}[Logarytm dyskretny]
Dana jest grupa cykliczna $\G$,
jej generator $g \in \G$
oraz jej element $a \in \G$.
Znaleźć liczbę całkowitą $k \in \Z$ taką, że:
\begin{equation}\label{discrete_log_eqn}
g^k = a
\end{equation}
\end{problem}

\begin{remark}
Zwyczajowo problem logarytmu dyskretnego opisuje się,
stosując zapis multiplikatywny działania grupowego.
W przypadku zapisu addytywnego równanie \ref{discrete_log_eqn}
przybiera postać:
\begin{equation}\label{discrete_log_additive_eqn}
kg = a
\end{equation}
\end{remark}

\begin{remark}
Problem logarytmu dyskretnego można rozpatrywać w przypadku dowolnej grupy.
Należy wówczas rozpatrywać jej podgrupę $(g)$
generowaną przez pewien jej element $g$.
\end{remark}

\begin{remark}
Problem logarytmu dyskretnego w grupie $\Z / n\Z$ jest prosty,
ponieważ dysponujemy algorytmem obliczającym rozwiązania
równania \ref{discrete_log_additive_eqn} w grupie $\Z / n\Z$.
\end{remark}

\begin{remark}
Problem logarytmu dyskretnego w grupie multiplikatywnej ciała $\GF(p^e)$
jest obecnie uważany za problem trudny obliczeniowo.
\end{remark}

Jak wiemy, grupa multiplikatywna ciała $\GF(p^e)$
jest izomorficzna z grupą $\Z / (p^e-1)\Z$.
Mimo to problemy logarytmu dyskretnego w tych grupach
różnią się trudnością,
co jest spowodowane różnicą w sposobie,
w jaki reprezentujemy elementy obu grup
oraz obliczamy wyniki działania grupowego.

Problemem spokrewnionym z problemem logarytmu dyskretnego
jest zagadnienie złamania protokołu Difiego-Hellmana,
zwane w skrócie ,,problemem Diffiego-Hellmana''.

\begin{problem}[Diffie-Hellman]
Dana jest grupa cykliczna $\G$,
jej generator $g \in G$
oraz elementy $a = g^k$ i $b = g^l$,
gdzie $k, l \in \Z$.
Znaleźć element $c \in \G$ taki, że:
\begin{equation}
c = g^{kl}
\end{equation}
\end{problem}

\begin{remark}
Podobnie jak w przypadku problemu logarytmu dyskretnego,
możemy stosować zapis addytywny (szukamy wówczas wartości $klg$)
oraz rozpatrywać problem w dowolnej grupie.
\end{remark}

Pokrewieństwo obu problemów polega na tym,
że jeden można zredukować do drugiego.

\begin{theorem}
Problem Diffiego-Hellmana w grupie $\G$ można
w czasie wielomianowym w sposób deterministyczny
zredukować do problemu logarytmu dyskretnego w grupie $\G$.
\end{theorem}

\begin{proof}
Redukcja jest bardzo prosta.
Jeżeli dysponujemy algorytmem rozwiązującym
problem logarytmu dyskretnego w grupie $\G$,
to postępujemy następująco:
\begin{enumerate}
\item na podstawie wartości $\G$, $g$ i $a = g^k$ obliczamy wartość $k$;
\item na podstawie wartości $\G$, $g$ i $b = g^l$ obliczamy wartość $l$;
\item obliczamy $kl$;
\item obliczamy $g^{kl}$.
\end{enumerate}
Jak widać, aby rozwiązać egzemplarz problemu Diffiego-Hellmana
wystarczy dwa razy zastosować rozwiązanie problemu logarytmu dyskretnego.
\end{proof}

Czy problem Diffiego-Hellmana można rozwiązać inaczej?
Być może wystarczy odpowiednio wykorzystać inne operacje,
które daje się efektywnie obliczać.
Zagadnienie to jest o tyle istotne,
że na trudności problemu Diffiego-Hellmana
opiera się wiele kryptosystemów,
których implementacje są wykorzystywane na codzień,
m.in. w protokole SSL używanym w sieci Internet.
Wydaje się, że oba problemy są równoważne (por. \ref{maurer}),
a to oznacza trudność problemu Diffiego-Hellmana,
a zatem bezpieczeństwo używanych protokołów.

Oba podane problemy zostały przedstawione w wersji obliczeniowej.
W teorii złożoności obliczeniowej często rozpatruje się wersje decyzyjne
problemów. Problem Diffiego-Hellmana przybiera wówczas następującą postać.

\begin{problem}
Dana jest grupa $\G$,
jej generator $g \in \G$
oraz elementy $a = g^k$, $b = g^l$ i $c = g^m$
gdzie $k, l, m \in \Z$.
Stwierdzić, czy zachodzi następująca zależność:
\begin{equation}
kl \equiv m \quad (\mathrm{mod}\ \abs{\G})
\end{equation}
\end{problem}

Jest jasne, że problem decyzyjny można zredukować do problemu obliczeniowego,
zatem jest on nietrudniejszy. Może się zdarzyć, że jest dużo prostszy.

W przypadku krzywych eliptycznych obie wersje problemu Diffiego-Hellmana
są następujące.

\begin{problem}
Dana jest krzywa eliptyczna $E(\K)$,
punkt $G \in E$ skończonego rzędu $n$
oraz punkty $P = kG$ i $Q = lG$,
gdzie $k, l \in \Z$.
Znaleźć punkt $R \in E$ taki, że:
\begin{equation}
R = klG
\end{equation}
\end{problem}

\begin{problem}
Dana jest krzywa eliptyczna $E(\K)$,
punkt $G \in E$ skończonego rzędu $n$
oraz punkty $P = kG$, $Q = lG$ i $R = mG$,
gdzie $k, l, m \in \Z$.
Stwierdzić, czy zachodzi następująca zależność:
\begin{equation}
kl \equiv m \quad (\mathrm{mod}\ n)
\end{equation}
\end{problem}

Czy problem Diffiego-Hellmana na krzywej eliptycznej jest trudny?
Okazuje się, że iloczyn Weila ma wpływ na tę kwestię.

\begin{theorem}[Redukcja MOV]
Obliczeniowy problem logarytmu dyskretnego
na krzywej eliptycznej $E = E(\GF(p^e))$
można w czasie wielomianowym w sposób deterministyczny zredukować
do obliczeniowego problemu logarytmu dyskretnego
w grupie multiplikatywnej ciała $\GF(p^{ef})$.
\end{theorem}

\begin{proof}
Niech punkt $G \in E$ rzędu $n$
oraz punkt $P = kG$ będą instancją
obliczeniowego problemu logarytmu dyskretnego na krzywej $E$.
Rozważmy rozszerzenie $\GF(p^{ef})$ ciała $\GF(p^e)$ dostatecznie duże,
aby istniał punkt $H \in E(\GF(p^{ef})$ taki,
że $\ord(H) = n$ oraz $H \notin (G)$.
Jest to możliwe, gdy $p \nmid n$.
Wartość $w(G, H)$ jest wówczas
pierwiastkiem pierwotnym $n$-tego stopnia z jedności.
Oznaczmy $\mu = w(G, H)$ i policzmy wartość $w(P, H)$:
\begin{eqnarray*}
w(P, H)
& = & w(kG, H) \\
& = & w(G, H)^k \\
& = & \mu^k
\end{eqnarray*}
Elementy $\mu$ i $w(P, H)$ stanowią zatem
egzemplarz problemu logarytmu dyskretnego
w grupie multiplikatywnej ciała $\GF(p^{ef})$,
który ma takie samo rozwiązanie, jak egzemplarz pierwotnego problemu.
\end{proof}

\begin{corollary}
Obliczeniowy problem Diffiego-Hellmana na krzywej eliptycznej $E(\GF(p^e))$
jest nietrudniejszy od obliczeniowego problemu logarytmu dyskretnego
w grupie multiplikatywnej ciała $\GF(p^{ef})$.
\end{corollary}

\begin{theorem}
Decyzyjny problem Diffiego-Hellmana
na krzywej superosobliwej $E = E_{0,1}(\F(p))$
jest łatwy.
\end{theorem}

\begin{proof}
Niech punkt $G \in E$ rzędu $n$
oraz punkty $P = kG$, $Q = lG$ i $R = mG$
będą instancją decyzyjnego problemu Diffiego-Hellmana na krzywej $E$.
Oznaczmy $G = (a, b)$.
Niech $\xi$ oznacza pierwiastek trzeciego stopnia z jedności w ciele $\GF(p^2)$.
Jak wiemy, punkt $H = (\xi a, b)$ również ma rząd $n$
i razem z punktem $G$ generuje grupę $E[n]$.

Oznaczmy odwzorowanie $(x, y) \to (\xi x, y)$ przez $f$.
Zauważmy, że zachodzą następujące zależności:
\begin{eqnarray*}
w(P, f(Q)) & = & w(G, H)^{kl} \\
w(R, f(G)) & = & w(G, H)^m
\end{eqnarray*}
Ponieważ $w(G, H)$ to pierwiastek pierwotny $n$-tego stopnia z jedności,
widzimy, że $kl \equiv m \quad (\mathrm{mod}\ n)$ wtedy i tylko wtedy, gdy
$w(P, f(Q)) = w(R, f(G))$.
\end{proof}

\begin{remark}
Kluczową rolę w dowodzie odegrał fakt, że izomorfizm $f$
między grupami generowanymi przez punkty $G$ i $H$ daje się łatwo obliczać.
Rozumowanie to można uogólnić na wszystkie krzywe,
dla których jesteśmy w stanie wskazać analogiczny izomorfizm,
który można efektywnie obliczać.
\end{remark}
