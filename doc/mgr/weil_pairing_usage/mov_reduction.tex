\section{Redukcja MOV}

\noindent
Pierwsze zastosowanie iloczynu Weila
to redukcja Menezesa-Okamoto-Vanstone'a \cite{menezesokamotovanstone}.
Jest to zastosowanie ,,złe'',
ponieważ pozwala przeprowadzać atak kryptograficzny
na systemy oparte na krzywych eliptycznych.
Odkrycie redukcji MOV spowodowało,
że niebezpieczne stało się budowanie systemów kryptograficznych
opartych na superosobliwych krzywych eliptycznych.
Jest to o tyle niefortunne,
że krzywe superosobliwe wyjątkowo dobrze nadają się do
realizacji komputerowej
i wiązano z nimi nadzieje na praktyczne, wydajne kryptosystemy.

\noindent
Redukcja MOV związana jest
z problemem logarytmu dyskretnego oraz protokołem Diff\-iego-Hell\-mana,
dlatego w pierszej kolejności podamy definicje obu problemów.

\begin{problem}[Logarytm dyskretny]
Dana jest grupa cykliczna $\G$,
jej generator $g$
oraz jej element $a$.
Znaleźć liczbę całkowitą $k \in \Z$ taką, że:
\begin{equation}\label{discrete_log_eqn}
g^k = a
\end{equation}
\end{problem}

\begin{remark}
Zwyczajowo problem logarytmu dyskretnego opisuje się,
stosując zapis multiplikatywny działania grupowego.
W przypadku zapisu addytywnego równanie \ref{discrete_log_eqn}
przybiera następującą postać:
\begin{equation}\label{discrete_log_additive_eqn}
kg = a
\end{equation}
\end{remark}

\begin{remark}
Problem logarytmu dyskretnego można rozpatrywać w przypadku dowolnej grupy.
Należy wówczas rozpatrywać jej podgrupę $(g)$
generowaną przez pewien jej element $g$.
\end{remark}

\begin{remark}
Trudność problemu logarytmu dyskretnego w danej grupie
zależy od sposobu reprezentacji elementów tej grupy
za pomocą ciągów bitów.
I tak na przykład problem logarytmu dyskretnego w grupie $\Z / (p-1)\Z$
jest łatwy,
zaś w grupie multiplikatywnej ciała $\F(p)$,
która jest przecież izomorficzna z grupą $\Z / (p-1)\Z$,
jest obecnie uznawany za trudny.
\end{remark}

\noindent
Problemem spokrewnionym z problemem logarytmu dyskretnego
jest zagadnienie złamania protokołu Difiego-Hellmana,
zwane w skrócie ,,problemem Diffiego-Hellmana''.

\begin{problem}[Protokół Diffiego-Hellmana]
Dana jest grupa cykliczna $\G$,
jej generator $g$
oraz elementy $a$ i $b$.
Niech $k$ i $l$ będą takimi liczbami całkowitymi,
że $a = g^k$ i $b = g^l$.
Znaleźć (znając tylko wartości $g$, $a$ i $b$)
element $c \in \G$ taki, że:
\begin{equation}
c = g^{kl}
\end{equation}
\end{problem}

\begin{remark}
Podobnie jak w przypadku problemu logarytmu dyskretnego,
możemy stosować zapis addytywny (szukamy wówczas wartości $klg$)
oraz rozpatrywać problem w dowolnej grupie.
\end{remark}

\noindent
Pokrewieństwo obu problemów polega na tym,
że jeden można zredukować do drugiego.

\begin{theorem}
Problem Diffiego-Hellmana w grupie $\G$ można
w czasie wielomianowym w sposób deterministyczny
zredukować do problemu logarytmu dyskretnego w grupie $\G$.
\end{theorem}

\begin{proof}
Redukcja jest bardzo prosta.
Jeżeli dysponujemy algorytmem rozwiązującym
problem logarytmu dyskretnego w grupie $\G$,
to postępujemy następująco:
\begin{enumerate}
\item na podstawie wartości $\G$, $g$ i $a$ obliczamy wartość $k$;
\item na podstawie wartości $\G$, $g$ i $b$ obliczamy wartość $l$;
\item obliczamy $kl$;
\item obliczamy $g^{kl}$.
\end{enumerate}
Jak widać, aby rozwiązać egzemplarz problemu Diffiego-Hellmana
wystarczy dwa razy zastosować rozwiązanie problemu logarytmu dyskretnego.
\end{proof}

\noindent
Czy problem Diffiego-Hellmana można rozwiązać inaczej?
Zagadnienie to jest o tyle istotne,
że na trudności problemu Diffiego-Hellmana
opiera się wiele kryptosystemów,
których implementacje są wykorzystywane na co dzień,
m.in. w protokole SSL używanym w sieci Internet.
Wydaje się, że oba problemy są równoważne (por. \cite{maurer}),
a to oznacza trudność problemu Diffiego-Hellmana,
a zatem bezpieczeństwo używanych kryptosystemów.

\noindent
Oba podane problemy zostały przedstawione w wersji obliczeniowej.
W teorii złożoności obliczeniowej często rozpatruje się wersje decyzyjne
problemów. Problem Diffiego-Hellmana przybiera wówczas następującą postać.

\begin{problem}
Dana jest grupa $\G$,
jej generator $g$
oraz jej elementy $a$, $b$ i $c$.
Niech $k$, $l$ i $m$ będą takimi liczbami całkowitymi,
że $a = g^k$, $b = g^l$ i $c = g^m$.
Stwierdzić (znając tylko wartości $g$, $a$, $b$ i $c$),
czy zachodzi następująca zależność:
\begin{equation}
kl \equiv m \quad (\mathrm{mod}\ \abs{\G})
\end{equation}
\end{problem}

\noindent
W przypadku krzywych eliptycznych
obliczeniowa i decyzyjna wersja problemu Diffiego-Hellmana
przybierają następującą postać.

\begin{problem}
Dana jest krzywa eliptyczna $E$ nad ciałem skończonym,
punkt $G$ rzędu $n$ na krzywej $E$
oraz punkty $P$ i $Q$ będące wielokrotnościami punktu $G$.
Niech $k$ i $l$ będą takimi liczbami całkowitymi,
że $kG = P$ i $lG = Q$.
Znaleźć (znając tylko wartości $G$, $P$ i $Q$)
punkt $R$ na krzywej $E$ taki, że:
\begin{equation}
R = klG
\end{equation}
\end{problem}

\begin{problem}
Dana jest krzywa eliptyczna $E$ nad ciałem skończonym,
punkt $G$ rzędu $n$ na krzywej $E$
oraz punkty $P$, $Q$ i $R$ będące wielokrotnościami punktu $G$.
Niech $k$, $l$ i $m$ będą takimi liczbami całkowitymi,
że $kG = P$, $lG = Q$ i $mG = R$.
Stwierdzić (znając tylko wartości $G$, $P$, $Q$ i $R$),
czy zachodzi następująca zależność:
\begin{equation}
kl \equiv m \quad (\mathrm{mod}\ n)
\end{equation}
\end{problem}

\noindent
Czy problemy Diffiego-Hellmana na krzywej eliptycznej są trudne?
Okazuje się, że iloczyn Weila ma wpływ na tę kwestię.

\begin{theorem}[Redukcja MOV]
Obliczeniowy problem logarytmu dyskretnego
na krzywej eliptycznej $E(\GF(p^e))$
można w czasie wielomianowym w sposób deterministyczny zredukować
do obliczeniowego problemu logarytmu dyskretnego
w grupie multiplikatywnej ciała $\GF(p^{ef})$.
\end{theorem}

\begin{proof}
Niech punkt $G$ rzędu $n$ na krzywej $E$
oraz punkt $P$ będący wielokrotnością punktu $G$
będą instancją obliczeniowego problemu logarytmu dyskretnego na krzywej $E$.
Rozważmy rozszerzenie $\GF(p^{ef})$ ciała $\GF(p^e)$ dostatecznie duże,
aby istniał punkt $H \in E(\GF(p^{ef})$ taki,
że $\ord(H) = n$ oraz $H \notin (G)$.
Jest to możliwe, gdy $p \nmid n$.
Wartość $w(G, H)$ jest wówczas
pierwiastkiem pierwotnym $n$-tego stopnia z jedności.
Oznaczmy $\mu = w(G, H)$, $P = kG$ i policzmy wartość $w(P, H)$:
\begin{eqnarray*}
w(P, H)
& = & w(kG, H) \\
& = & w(G, H)^k \\
& = & \mu^k
\end{eqnarray*}
Elementy $\mu$ i $w(P, H)$ stanowią zatem
egzemplarz problemu logarytmu dyskretnego
w grupie multiplikatywnej ciała $\GF(p^{ef})$,
który ma takie samo rozwiązanie, jak egzemplarz pierwotnego problemu.
\end{proof}

\begin{corollary}
Obliczeniowy problem Diffiego-Hellmana na krzywej eliptycznej $E(\GF(p^e))$
jest nietrudniejszy od obliczeniowego problemu logarytmu dyskretnego
w grupie multiplikatywnej ciała $\GF(p^{ef})$.
\end{corollary}

\begin{theorem}
Decyzyjny problem Diffiego-Hellmana
na krzywej superosobliwej \linebreak $E_{0,1}(\F(p))$
jest łatwy.
\end{theorem}

\begin{proof}
Niech punkt $G$ rzędu $n$ na krzywej $E$
oraz punkty $P$, $Q$ i $R$ będące wielokrotnościami punktu $G$
będą instancją decyzyjnego problemu Diffiego-Hellmana na krzywej $E$.
Oznaczmy $G = (a, b)$.
Niech $\xi$ oznacza pierwiastek trzeciego stopnia z jedności w ciele $\GF(p^2)$.
Jak wiemy, punkt $H = (\xi a, b)$ również ma rząd $n$
i razem z punktem $G$ generuje grupę $E[n]$.

\noindent
Oznaczmy odwzorowanie $(x, y) \to (\xi x, y)$ przez $f$.
Oznaczmy $P = kG$, $Q = lG$ i $R = mG$.
Zauważmy, że zachodzą następujące zależności:
\begin{eqnarray*}
w(P, f(Q)) & = & w(G, H)^{kl} \\
w(R, f(G)) & = & w(G, H)^m
\end{eqnarray*}
Ponieważ $w(G, H)$ to pierwiastek pierwotny $n$-tego stopnia z jedności,
widzimy, że $kl \equiv m \quad (\mathrm{mod}\ n)$ wtedy i tylko wtedy, gdy
$w(P, f(Q)) = w(R, f(G))$.
\end{proof}

\begin{remark}
Kluczową rolę w dowodzie odgrywa fakt,
że istnieje nietrywialny, łatwo obliczalny automorfizm grupy $E[n]$.
Rozumowanie to można uogólnić na wszystkie krzywe,
dla których jesteśmy w stanie wskazać analogiczny automorfizm.
\end{remark}
