\section{Szczególne krzywe eliptyczne}

\noindent
Krzywe eliptyczne, które są używane w systemach kryptograficznych,
są dobierane ze szczególną starannością.
Wskażemy pewien szczególny rodzaj krzywych eliptycznych,
które wyjątkowo dobrze nadają się do zastosowań
związanych z iloczynem Weila.

\begin{definition}
Dana jest liczba pierwsza $p$ taka, że $p \equiv 2\ (\mathrm{mod}\ 3)$.
\emph{Superosobliwa krzywa eliptyczna nad ciałem $\F(p)$}
to krzywa $E_{0,1}(\F(p))$.
\end{definition}

\begin{remark}
Krzywe superosobliwe definiuje się tak naprawdę w zupełnie inny,
bardziej ogólny sposób. Podana definicja to jedynie przykład
pewnej rodziny krzywych, które są superosobliwe w sensie ogólnej definicji.
Inną rodziną krzywych superosobliwych są krzywe postaci $E_{1,0}(\F(p))$,
gdzie $p \equiv 3\ (\mathrm{mod}\ 4)$.
\end{remark}

\noindent
Krzywe superosobliwe są interesujące,
struktura grupy na takich krzywych jest prosta,
co pokazują następujące twierdzenia.

\begin{theorem}
Krzywa superosobliwa $E_{0,1}(\F(p))$ składa się z $p + 1$ punktów.
\end{theorem}

\begin{proof}
Rozważmy równanie $x^3 = c$ w ciele $\F(p)$.
Weźmy dowolne dwa elementy $a, b \in \F(p)$ takie, że $a^3 = c$ i $b^3 = c$.
Gdy $c = 0$, to $a = b = 0$.
Gdy $c \neq 0$, zapisujemy elementy $a$ i $b$ jako potęgi generatora:
$a = g^k$, $b = g^l$.
Wiemy, że $a^3 = b^3$, skąd dostajemy $g^{3(k-l)} = 1$.
Zatem $p-1 \mid 3(k-l)$.
Ponieważ liczba $p-1$ jest względnie pierwsza z $3$,
otrzymujemy $p-1 \mid k-l$.
Stąd $a = b$.

\noindent
Z rozważań tych wynika następujący wniosek:
funkcja $x^3 + 1$ jest bijekcją w zbiorze $\F(p)$,
w jej obrazie znajdują się wszystkie reszty kwadratowe,
zatem krzwa superosobliwa $E_{0,1}(\F(p))$ zawiera dokładnie $p+1$ punktów.
\end{proof}

\begin{theorem}
Krzywa superosobliwa $E = E_{0,1}(\F(p))$
jest izomorficzna z grupą $\Z / (p+1)\Z$.
\end{theorem}

\begin{proof}
Zgodnie z twierdzeniem \ref{finite_curve_structure_theorem}
krzywa $E$ jest izomorficzna z grupą $(\Z / k\Z) \times (\Z / l\Z)$,
gdzie $kl = p+1$ oraz $l \mid \gcd(k, p-1)$.
Skoro $k \mid p+1$, to $l \mid \gcd(p+1, p-1)$,
zatem $l = 1$ lub $l = 2$.
W pierwszym przypadku twierdzenie jest udowodnione.
Pokażemy, że drugi jest niemożliwy.

\noindent
Przyjmijmy, że $l = 2$.
Jeżeli $2 \nmid k$,
to grupa $(\Z / k\Z) \times (\Z / l\Z)$ jest izomoriczna
z żądaną grupą $\Z / (p+1)\Z$.
Natomiast jeśli $2 \mid k$, to na krzywej $E$ są dwa punkty rzędu dwa,
a więc wielomian $x^3 + 1$ ma w ciele $\F(p)$ dwa miejsca zerowe.
To jest niemożliwe, bo wielomian ten jest bijekcją w zbiorze $\F(p)$,
jak zauważyliśmy wcześniej.
\end{proof}

\noindent
W kontekście iloczynu Weila krzywe superosobliwe są ważne dlatego,
że łatwo wskazywać ich podgrupy $n$-torsyjne.

\begin{theorem}
Dana jest liczba pierwsza $p$ taka, że $p \equiv 2\ (\mathrm{mod}\ 3)$
oraz liczba pierwsza $n \mid p + 1$.
Wówczas podgrupa $n$-torsyjna krzywej eliptycznej $E_{0,1}(\GF(p^2))$
jest izomorficzna z grupą $(\Z / n\Z) \times (\Z / n\Z)$.
\end{theorem}

\begin{proof}
Ponieważ $3 \mid p^2-1$, w ciele $\GF(p^2)$
istnieje nietrywialny pierwiastek trzeciego stopnia z jedności.
Oznaczmy go przez $\xi$.
Zauważmy teraz, że odwzorowanie $(x, y) \to (\xi x, y)$
jest automorfizmem krzywej $E_{0,1}(\GF(p^2))$.

\noindent
Niech $P = (a, b)$ będzie takim punktem krzywej $E_{0,1}(\F(p))$,
że $nP = \ecident$.
Teraz, punkt $Q = (\xi a, b) \in E_{0,1}(\GF(p^2))$ również ma rząd $n$,
bo jest obrazem punktu $P$ poprzez opisany automorfizm.
Zauważmy przy tym, że $Q \notin E_{0,1}(\F(p))$,
bo $\xi \notin \F(p)$.

\noindent
Wynika stąd już, że podgrupa $n$-torsyjna krzywej $E_{0,1}(\GF(p^2))$
ma żądaną postać. Co więcej, punkty $P$ i $Q$ są jej generatorami.
\end{proof}
