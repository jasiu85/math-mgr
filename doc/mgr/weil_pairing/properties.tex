\section{Własności}

Zbadamy teraz podstawowe własności iloczynu Weila.

\subsection*{Pierwiastki z jedności}

Zaczniemy od pokazania, że wartości, jakie przyjmuje iloczyn Weila,
to pierwiastki z jedynki.

\begin{theorem}
Dana jest krzywa eliptyczna $E$ oraz punkty $P, Q \in E[n]$.
Wówczas $w(P,Q)^n = 1$.
\end{theorem}

\todolipsum{Pierwiastki z jedności.}

\subsection*{Odwzorowanie dwuliniowe}

\todolipsum{Odwzorowanie dwuliniowe.}
