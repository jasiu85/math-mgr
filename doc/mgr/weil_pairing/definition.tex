\section{Definicja}

Mimo tego, że wprowadziliśmy już
wiele elementów teorii krzywych eliptycznych,
wciąż jeszcze brakuje nam pewnych wiadomości,
aby móc zdefiniować iloczyn Weila.
Powinny być one przedstawione w poprzednich rozdziałach,
lecz ze względu na ograniczoną objętość pracy
przedstawiamy je dopiero teraz
w bardzo ograniczonym ujęciu.

\subsection*{Podgrupy $n$-torsyjne}

Zaczniemy od podania dziedziny,
na której działa iloczyn Weila.
Są to pewne specyficzne podgrupy
grupy na krzywej eliptycznej.

\begin{definition}
Dana jest krzywa eliptyczna $E$.
\emph{Mnożenie punktu krzywej przez liczbę całkowitą $n$}
to funkcja $[n] \colon E \to E$
określona następującym wzorem:
\begin{equation}
[n](P) =
\fundefthree
{\underbrace{P + P + \cdots + P}_{n\textrm{ razy}}}{n > 0}
{\ecident}{n = 0}
{-[-n](P)}{n < 0}
\end{equation}
Wyrażenie $[n](P)$ zapisujemy skrótowo jako $nP$.
\end{definition}

Użycie określenia ,,mnożenie'' oraz skrótowego zapisu $nP$
motywują własności funkcji $[n]$,
które przypominają cechy operacji mnożenia liczb całkowitych.

\begin{fact}
Dana jest krzywa eliptyczna $E$, punkty $P, Q \in E$
oraz liczby całkowite $m$ i $n$.
Wówczas:
\begin{eqnarray*}
m(P + Q) & = & mP + mQ \\
(m + n)P & = & mP + nP \\
m(-P) & = & -(mP) \\
(-m)P & = & -(mP)
\end{eqnarray*}
\end{fact}

Jesteśmy zainteresowani takimi punktami $P$ krzywej,
które spełniają równanie $nP = \ecident$.
Jak nietrudno stwierdzić na podstawie podanych własności mnożenia,
tworzą one grupę.

\begin{definition}
Dana jest krzywa eliptyczna $E$
oraz liczba naturalna $n$.
\emph{Podgrupa $n$-torsyjna} grupy na krzywej $E$,
oznaczana $E[n]$,
to podgrupa złożona z tych punktów $P \in E$,
które spełniają zależność:
\begin{equation}
nP = \ecident
\end{equation}
\end{definition}

Podajemy bez dowodu kilka podstawowych własności podgrup $n$-torsyjnych.

\begin{fact}
Dana jest krzywa eliptyczna $E$
oraz liczby naturalne $m$ i $n$.
Jeżeli $m \mid n$, to $E[m] \subset E[n]$.
\end{fact}

\begin{theorem}
Dana jest krzywa eliptyczna $E$ nad ciałem algebraicznie domkniętym $\K$
oraz liczba naturalna $n$.
Jeżeli $\fieldchar(\K) = 0$ lub $\gcd(n, \fieldchar(\K)) = 1$,
to podgrupa $n$-torsyjna $E[n]$ ma rząd równy $n^2$
i jest izomorficzna z grupą $(\Z / n\Z)\times(\Z / n\Z)$.
\end{theorem}

\begin{corollary}
Dana jest krzywa eliptyczna $E$ nad ciałem algebraicznie domkniętym.
Podgrupa $E[1]$ jest trywialna -- składa się tylko z punktu $\ecident$.
Podgrupa $E[2]$ składa się z czterech punktów:
punktu $\ecident$ oraz trzech punktów rzędu dwa.
\end{corollary}

\subsection*{Pewne szczególne dywizory}

Elementy podgrupy $n$-torsyjnej to rozwiązania równania $nQ = \ecident$.
Rozważymy teraz bardziej ogólne równanie $nQ = P$.
Doprowadzi nas to do definicji pewnego rodzaju dywizorów,
które pojawiają się w definicji iloczynu Weila.

\begin{theorem}
Dana jest krzywa eliptyczna $E$, jej podgrupa $E[n]$ oraz punkt $P \in E[n]$.
Wówczas:
\begin{itemize}
\item istnieje taki punkt $Q_0 \in E[n^2]$, że $P = nQ_0$;
\item każdy punkt $Q \in E[n^2]$ spełniający równanie
$P = nQ$ można przedstawić w postaci $Q = Q_0 + R$,
gdzie $R \in E[n]$.
\end{itemize}
\end{theorem}

\begin{proof}
Wystarczy przeanalizować sposób, w jaki grupa $(\Z/n\Z)\times(\Z/n\Z)$
jest zanurzona w grupie $(\Z/n^2\Z)\times(\Z/n^2\Z)$.
\end{proof}

\begin{corollary}\label{point_division_corollary}
Jest $n^2$ punktów $Q \in E[n^2]$ spełniających równanie $P = nQ$,
gdzie $P \in E[n]$.
Zbiór tych punktów można przedstawić w postaci $\{Q_0 + R \mid R \in E[n]\}$,
gdzie $Q_0$ jest dowolnym takim punktem.
\end{corollary}

Rozważania te motywują definicję ,,dzielenia'' punktu.
Takie dzielenie jest podobne do pierwiastkowania liczb zespolonych --
jest niejednoznaczne, ale zbiór wszystkich możliwych wyników dzielenia
przejawia pewną strukturę.
Zbiór ten będziemy reprezentować za pomocą dywizora.

\begin{definition}
Dana jest krzywa eliptyczna $E$ nad ciałem algebraicznie domkniętym.
\emph{Dzielenie punktu krzywej przez liczbę całkowitą $n$}
to funkcja $[n]^{-1} \colon E[n] \to \Div(E)$
określona następującym wzorem:
\begin{equation}
[n]^{-1}(P) = \sum_{nQ = P} \divi{Q}
\end{equation}
\end{definition}

Następujące twierdzenie przedstawia własność tego rodzaju dywizorów,
która będzie nas najbardziej interesować.

\begin{theorem}\label{point_division_divisor_principle_theorem}
Dana jest krzywa eliptyczna $E$ oraz punkt $P \in E[n]$.
Wówczas:
\begin{eqnarray*}
[n]^{-1}(P) \sim n^2\divi{\ecident}
\end{eqnarray*}
\end{theorem}

Dowód tego twierdzenia poprzedzimy bardzo użytecznym lematem.

\begin{lemma}\label{divi_reduction_lemma}
Dana jest krzywa eliptyczna $E$ oraz punkty $P, Q \in E$.
Wówczas dywizor $\divi{P+Q} - \divi{P} - \divi{Q} + \divi{\ecident}$
jest główny.
\end{lemma}

\begin{proof}
Rozważmy funkcję $\frac{f_1}{f_2}$,
gdzie $f_1$ to linia przechodząca przez punkty $(P+Q)$ i $-(P+Q)$,
a $f_2$ to linia przechodząca przez punkty $P$, $Q$ i $-(P+Q)$.
Jak nietrudno sprawdzić,
$\rdiv(\frac{f_1}{f_2}) = \divi{P+Q} - \divi{P} - \divi{Q} + \divi{\ecident}$
(również wtedy, gdy $P = Q$, $P = -Q$, $P = \ecident$ lub $Q = \ecident$).
\end{proof}

\begin{corollary}\label{divi_sum_reduction_coro}
Dane jest krzywa eliptyczna $E$ oraz punkty $P, Q \in E$.
Wówczas $\divi{P} + \divi{Q} \sim \divi{P+Q} + \divi{\ecident}$.
\end{corollary}

Zauważmy, że z pomocą tego lematu i płynącego z niego wniosku
możemy sformułować metodę redukcji dywizorów
podobne do twierdzenia \ref{point_division_divisor_principle_theorem}.
Ograniczymy się jednak do zastosowania tej metody
do przypadku dywizorów postaci $[n]^{-1}(P)$.

\begin{proof}[Dowód twierdzenia \ref{point_division_divisor_principle_theorem}]
Niech $\alpha$ i $\beta$ oznaczają generatory grupy $E[n]$.
Wybierzmy dowolny punkt $Q \in E[n^2]$ taki, że $P = nQ$
Zgodnie z wnioskiem \ref{point_division_corollary}
dywizor $[n]^{-1}(P)$ możemy zapisać w postaci:
\begin{equation*}
[n]^{-1}(P) = \sum_{k=0}^{n-1} \sum_{l=0}^{n-1} \divi{Q + k\alpha + l\beta}
\end{equation*}
Teraz wielokrotnie zastosujemy wniosek \ref{divi_sum_reduction_coro},
aby zamienić sumę dywizorów na dywizor sumy.
Najpierw $n-1$ razy redukujemy wewnętrzną sumę i otrzymujemy:
\begin{eqnarray*}
\sum_{l=0}^n \divi{Q + k\alpha + l\beta}
& =    & \divi{Q + k\alpha} + \divi{Q + k\alpha + \beta} +
         \sum_{l=2}^{n-1} \divi{Q + k\alpha + l\beta} \\
& \sim & \divi{\ecident} + \divi{2Q + 2k\alpha + \beta} +
         \divi{Q + k\alpha + 2\beta} +
         \sum_{l=3}^{n-1} \divi{Q + k\alpha + l\beta} \\
& \sim & 2\divi{\ecident} + \divi{3Q + 3k\alpha + 3\beta} +
         \divi{Q + k\alpha + 3\beta} +
         \sum_{l=4}^{n-1} \divi{Q + k\alpha + l\beta} \\
& \sim & 3\divi{\ecident} + \divi{4Q + 4k\alpha + 6\beta} +
         \divi{Q + k\alpha + 4\beta} +
         \sum_{l=5}^{n-1} \divi{Q + k\alpha + l\beta} \\
& \sim & \ldots \\
& \sim & (n-1)\divi{\ecident} + \divi{nQ + nk\alpha + \binom{n}{2}\beta}
\end{eqnarray*}
Wynik ten podstawiamy do zewnętrznej sumy,
po czym w analogiczny sposób redukujemy ją $n-1$ razy.
Otrzymujemy:
\begin{eqnarray*}
\sum_{k=0}^n (n-1)\divi{\ecident} +
             \divi{nQ + nk\alpha + \binom{n}{2}\beta}
& \sim & (n^2-1)\divi{\ecident} +
         \divi{n^2Q + n\binom{n}{2}\alpha + n\binom{n}{2}\beta} \\
& \sim & (n^2-1)\divi{\ecident} + \divi{n^2Q} \\
& \sim & n^2\divi{\ecident}
\end{eqnarray*}
\end{proof}

\begin{corollary}
Dywizor $[n]^{-1}(P) - [n]^{-1}(\ecident)$ jest główny.
\end{corollary}

\subsection*{Iloczyn Weila}

Dysponujemy już wszystkimi niezbędnymi informacjami,
aby móc podać definicję iloczynu Weila.

\begin{definition}\label{weil_pairing_def}
Dana jest krzywa eliptyczna $E$ nad ciałem $\K$.
\emph{Iloczyn Weila} to funkcja
$w\colon E[n] \times E[n] \to \K$
określona następującym wzorem:
\begin{equation}\label{weil_pairing_eqn}
w(P, Q) = \frac{f_P \circ t_Q}{f_P}(\ecident)
\end{equation}
Oznaczenia użyte w podanym wzorze są następujące:
\begin{itemize}
\item $f_P$ to dowolna funkcja wymierna taka,
że $\rdiv(f_P) = [n]^{-1}(P) - [n]^{-1}(\ecident)$;
\item $t_Q$ to przesunięcie o $Q$, tzn. $t_Q(R) = R+Q$.
\end{itemize}
\end{definition}

\begin{remark}
Zauważmy, że podana definicja określa nie jedną funkcję,
lecz całą rodzinę funkcji --
po jednej dla każdej z grup $E[n]$, gdzie $n \in \N$.
\end{remark}

Sprawdźmy, że iloczyn Weila jest dobrze określony

\begin{lemma}\label{weil_pairing_ignore_const_lemma}
Wartość iloczynu Weila nie zależy od wyboru funkcji $f_P$.
\end{lemma}

\begin{proof}
Na mocy wniosku \ref{fun_divi_equiv_to_const_lemma}
dwie funkcje wymierne $f_P$ i $f_P'$ o takim samym dywizorze
różnią się o czynnik stały różny od zera,
zatem możemy przyjąć $f_P = cf_P'$.
Podstawiamy tę zależność do wzoru \ref{weil_pairing_eqn} i otrzymujemy:
\begin{equation*}
\frac{f_P \circ t_Q}{f_P}(\ecident) =
\frac{(cf_P') \circ t_Q}{cf_P'}(\ecident) =
\frac{c(f_P' \circ t_Q)}{cf_P'}(\ecident) =
\frac{f_P' \circ t_Q}{f_P'}(\ecident)
\end{equation*}
Widzimy stąd, że wartość iloczynu Weila pozostanie taka sama
niezależnie od tego, jakiej funkcji użyjemy,
ponieważ wszystkie one różnią się o czynnik stały,
który pojawia się zarówno w liczniku, jak i w mianowniku.
\end{proof}

\begin{lemma}\label{weil_pairing_same_divi_lemma}
Funkcje $f_P$ i $f_P \circ t_Q$ mają taki sam dywizor.
\end{lemma}

\begin{proof}
Niech $R \in E[n^2]$ będzie dowolnym punktem takim, że $nR = P$.
Zauważmy, że $f_P(S) = 0$ wtedy i tylko wtedy,
gdy $(f_P \circ t_Q)(S - Q) = 0$.
Podobnie, $f_P(R) = \infty$ wtedy i tylko wtedy,
gdy $(f_P \circ t_Q)(S - Q) = \infty$.
Wszystkie miejsca zerowe i bieguny funkcji $f_P$ są jednokrotne.
Dywizor funkcji $f_P \circ t_Q$ możemy w takim razie zapisać
w następującej postaci:
\begin{equation*}
\rdiv(f_P \circ t_Q) =
\sum_{S \in E[n]} \divi{R + S - Q} - \sum_{S \in E[n]} \divi{S - Q}
\end{equation*}
Wykonujemy ,,zamianę zmiennych''. Oznaczmy $S' = S - Q$.
Zauważmy, że ponieważ $S, Q \in E[n]$, to również $S' \in E[n]$.
Co więcej, przesunięcie całego zbioru $E[n]$ o $-Q$ nie zmienia go,
tak więc nie zmienia się zakres wartości zmiennej $S'$.
Otrzymujemy:
\begin{eqnarray*}
\rdiv(f_P \circ t_Q) =
& = & \sum_{S' \in E[n]} \divi{R + S} - \sum_{S' \in E[n]} \divi{S'} \\
& = & \rdiv(f_P)
\end{eqnarray*}
\end{proof}

\begin{corollary}
Funkcja $\frac{f_P \circ t_Q}{f_P}$ jest stała i niezerowa.
\end{corollary}

\begin{proof}
Stosujemy twierdzenie \ref{fun_mul_divi_add_theorem}
i udowodniony przed chwilą lemat
do obliczenia dywizora funkcji $\frac{f_P \circ t_Q}{f_P}$
i otrzymujemy:
\begin{equation*}
\rdiv\left(\frac{f_P \circ t_Q}{f_P}\right) =
\rdiv(f_P \circ t_Q) - \rdiv(f_P) = 0
\end{equation*}
Funkcja mająca zerowy dywizor jest stała i niezerowa,
zgodnie z wnioskiem \ref{zero_div_const_fun_coro}.
\end{proof}

\begin{remark}
Wybór punktu $\ecident$ jako argumentu funkcji $\frac{f_P \circ t_Q}{f_P}$
we wzorze \ref{weil_pairing_eqn} był dowolny,
bo funkcja ta i tak jest stała.
Dlatego też czasem będziemy nadużywać notacji i zapisywać
iloczyn Weila jako:
\begin{equation*}
w(P, Q) = \frac{f_P \circ t_Q}{f_P}
\end{equation*}
\end{remark}

\begin{remark}
Iloczyn Weila nie jest funkcją stałą --
współczynnik $c$ w tożsamości $f_P \circ t_Q = cf_P$
może zmieniać się w zależności od punktu $Q$ mimo tego,
że funkcja $f_P \circ t_Q$ zawsze ma taki sam dywizor jak funkcja $f_P$.
\end{remark}

\begin{theorem}
Iloczyn Weila $w(P, Q)$ jest dobrze określony,
tzn. jego wartość zależy tylko od punktów $P$ i $Q$.
\end{theorem}

\begin{proof}
Natychmiastowy na podstawie lematów
\ref{weil_pairing_ignore_const_lemma} i \ref{weil_pairing_same_divi_lemma}.
\end{proof}
