\section{Definicja}

Poznaliśmy już wiele pojęć związanych z krzywymi eliptycznymi,
wciąż jednak nie wystarczająco dużo,
aby móc zdefiniować iloczyn Weila.
Dlatego najpierw podamy definicje kilku pomocniczych pojęć.

\subsection*{Podgrupy $n$-torsyjne}

Iloczyn Weila określony to operacja dwuargumentowa
określona na pewnych specyficznych podgrupach
grupy na krzywej eliptycznej,
które teraz zdefiniujemy.

\begin{definition}
Dana jest krzywa eliptyczna $E$.
\emph{Mnożenie punktu przez liczbę całkowitą $n$}
to funkcja,
oznaczana symbolem $[n] \colon E \to E$,
przyporządkowująca punktowi $P \in E$ punkt, oznaczany $nP$,
określony następującym wzorem:
\begin{equation}
[n](P) =
\fundefthree
{nP = \underbrace{P + P + \cdots + P}_{n\textrm{ razy}}}{n > 0}
{\ecident}{n = 0}
{-[-n](P)}{n < 0}
\end{equation}
\end{definition}

Definicja ta nie zaskakuje -- mnożenie to wielokrotne dodawanie,
dlatego też ma podstawowe własności analogiczne do mnożenia liczb.

\begin{fact}
Dana jest krzywa eliptyczna $E$, punkty $P, Q \in E$
oraz liczby całkowite $m$ i $n$.
Wówczas:
\begin{eqnarray*}
m(P + Q) & = & mP + mQ \\
(m + n)P & = & mP + nP \\
m(-P) & = & -(mP) \\
(-m)P & = & -(mP)
\end{eqnarray*}
\end{fact}

Jesteśmy zainteresowani takimi punktami $P$ krzywej,
które spełniają równanie $nP = 0$.
Jak nietrudno stwierdzić na podstawie podanych własności mnożenia,
tworzą one grupę.

\begin{definition}
Dana jest krzywa eliptyczna $E$
oraz liczba naturalna $n$.
\emph{Podgrupa $n$-torsyjna} grupy na krzywej $E$,
oznaczana $E[n]$,
to podgrupa złożona z tych punktów $P \in E$,
które spełniają zależność:
\begin{equation}
nP = 0
\end{equation}
\end{definition}

Podajemy bez dowodu kilka podstawowych własności podgrup $n$-torsyjnych.

\begin{fact}
Dana jest krzywa eliptyczna $E$
oraz liczby naturalne $m$ i $n$.
Jeżeli $m \mid n$, to $E[m] \subset E[n]$.
\end{fact}

\begin{theorem}
Dana jest krzywa eliptyczna $E$ nad ciałem algebraicznie domkniętym
oraz liczba naturalna $n$.
Wówczas podgrupa $n$-torsyjna $E[n]$ ma rząd równy $n^2$
i jest izomorficzna z grupą $(\Z / n\Z)\times(\Z / n\Z)$.
\end{theorem}

\begin{corollary}
Dana jest krzywa eliptyczna $E$ nad ciałem algebraicznie domkniętym.
Podgrupa $E[1]$ jest trywialna -- składa się tylko z punktu $\ecident$.
Podgrupa $E[2]$ składa się z czterech punktów -- punktu $\ecident$
oraz trzech punktów rzędu dwa.
\end{corollary}
