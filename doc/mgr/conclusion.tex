\chapter*{Podsumowanie}
\addcontentsline{toc}{chapter}{Podsumowanie}

\noindent
Zgodnie z postawioną na początku pracy tezą
zobaczyliśmy na przykładzie iloczynu Weila,
w jaki sposób pozornie abstrakcyjne matematyczne pojęcie
może mieć bardzo realny wpływ na życie codzienne.

\noindent
Możliwości, które daje iloczyn Weila,
nie są wykorzystywane w praktyce.
Wszystkie popularne systemy kryptograficzne
oparte na krzywych eliptycznych
wykorzystują je tylko w podstawowym zakresie.
Aby ten stan rzeczy zmienił się,
należy podjąć następujące kroki:
\begin{itemize}
\item
udoskonalić implementację algorytmu Millera;
\item
zmodyfikować sam algorytm tak, aby wyeliminować z niego losowość;
\item
zaproponować takie modyfikacje protokołów
powiązanych z szyfrowaniem,
infrastrukturą kluczy publicznych
i pocztą elektroniczną,
które umożliwią użycie kryptosystemów opartych na iloczynie Weila;
\item
zaimplementować kryptosystemy oparte na iloczynie Weila
w bibliotekach szyfrujących oraz serwerach i programach pocztowych.
\end{itemize}

\noindent
Powyższe kierunki dalszych prac to zagadnienia typowo praktyczne.
Są również możliwe dalsze badania teoretyczne:
\begin{itemize}
\item
konstruowanie kolejnych systemów kryptograficznych
na podstawie iloczynu Weila;
\item
konstruowanie systemów kryptograficznych
na podstawie odwzorowań podobnych do iloczynu Weila,
np. na podstawie odwzorowania Tate'a;
\item
przeprowadzanie ataków kryptograficznych za pomocą iloczynu Weila;
\item
analiza odporności istniejących systemów kryptograficznych
opartych na krzywych eliptycznych
na potencjalny atak oparty na iloczynie Weila.
\end{itemize}

\noindent
Miejmy nadzieję, że iloczyn Weila,
opierające się na nim kryptosystemy
oraz wszelkie inne ciekawe zdobycze kryptografii
staną się kiedyś popularne i użyteczne,
a dzięki nim matematyka stanie się nauką bardziej lubianą i docenianą.
