\chapter*{Podsumowanie}
\addcontentsline{toc}{chapter}{Podsumowanie}

\noindent
Cel wyznaczony na początku pracy został osiągnięty.
Zobaczyliśmy na przykładzie iloczynu Weila,
że pozornie abstrakcyjne matematyczne pojęcie
może odnaleźć drogę do zastosowania w praktyce
i że droga ta nie musi być ani długa, ani trudna:
kryptosystemy wykorzystujące iloczyn Weila
nie odbiegają poziomem skomplikowania
od najpopularniejszych obecnie kryptosystemów,
a implementacja algorytmu Millera będąca częścią pracy
powstała szybko, nie nastręczała większych trudności,
a powstały kod źródłowy jest przejrzysty.

\noindent
Kryptosystemy oparte na iloczynie Weila nie są jeszcze tak rozpowszechnione,
jak np. kryptosystem RSA czy protokół Diffiego-Hellmana.
Za adekwatne kryterium popularności można uznać pytanie,
czy istnieje kryptosystem oparty na iloczynie Weila taki,
że istnieje jego wolna, otwartoźródłowa implementacja,
która jest częścią jednej z wiodących dystrybucji systemu Linuks,
lub taki,
który jest objęty jednym ze standardów regulujących działanie
sieci Internet (np. RFC \cite{rfc}).
W chwili pisania niniejszej pracy odpowiedź na to pytanie jest negatywna.
Aby ten stan rzeczy zmienił się,
należy podjąć następujące kroki:
\begin{itemize}
\item
udoskonalić implementację algorytmu Millera;
\item
zmodyfikować sam algorytm tak, aby wyeliminować z niego losowość;
\item
zaproponować takie modyfikacje standardów internetowych
powiązanych z szyfrowaniem,
infrastrukturą kluczy publicznych
i pocztą elektroniczną,
które umożliwią użycie kryptosystemów opartych na iloczynie Weila
na szeroką skalę;
\item
zaimplementować kryptosystemy oparte na iloczynie Weila
w popularnych wolnoźródłowych bibliotekach szyfrujących
oraz serwerach i przeglądarkach poczty elektronicznej.
\end{itemize}

\noindent
Powyższe kierunki dalszych prac to zagadnienia typowo praktyczne.
Są również możliwe dalsze badania teoretyczne:
\begin{itemize}
\item
konstruowanie kolejnych systemów kryptograficznych
na podstawie iloczynu Weila;
\item
konstruowanie systemów kryptograficznych
na podstawie odwzorowań podobnych do iloczynu Weila,
np. na podstawie odwzorowania Tate'a;
\item
przeprowadzanie ataków kryptograficznych za pomocą iloczynu Weila;
\item
analiza odporności istniejących systemów kryptograficznych
opartych na krzywych eliptycznych
na potencjalny atak oparty na iloczynie Weila;
\item
projektowanie systemów kryptograficznych odpornych na ataki
wykorzystujące iloczyn Weila.
\end{itemize}

\noindent
Miejmy nadzieję, że iloczyn Weila,
opierające się na nim kryptosystemy
oraz wszelkie inne ciekawe zdobycze kryptografii
trafią kiedyś do codziennego użytku,
a dzięki nim matematyka będzie postrzegana jako bardziej przydatna
i stanie się bardziej lubiana.
