\section{Definicja}

Na potrzeby niniejszej pracy wprowadzamy
następującą definicję krzywej eliptycznej.

\begin{definition}\label{elliptic_curve_definition}
Dane jest ciało $\K$ oraz dwa jego elementy $A, B \in \K$.
Wówczas \emph{krzywa eliptyczna nad ciałem $\K$ o parametrach $A$ oraz $B$},
oznaczana $E_{A,B}(\K)$ (w skrócie $E(\K)$ lub $E$),
to zbiór składający się
ze wszystkich par $(x, y) \in \K \times \K$, które spełniają równość
\begin{equation}\label{elliptic_curve_equation}
y^2 = x^3 + Ax + B
\end{equation}
oraz jednego dodatkowego elementu oznaczanego symbolem $\ecident$.
Element $\ecident$ zwany jest
\emph{punktem w nieskończoności} lub \emph{identycznością},
a pozostałe elementy krzywej -- \emph{punktami skończonymi}.
Równanie \ref{elliptic_curve_equation} zwane jest \emph{równaniem krzywej}.
\end{definition}

\begin{definition}
Dany jest punkt skończony $P = (a, b)$ na krzywej eliptycznej.
\emph{Punkt sprzężony do punktu $P$}, oznaczany $\overline{P}$,
to punkt $(a, -b)$.
\end{definition}

\begin{definition}
Dana jest krzywa eliptyczna.
\emph{Punkt rzędu dwa} na tej krzywej
to punkt skończony postaci $(a, 0)$.
\end{definition}

Pochodzenie określeń
,,punkt w nieskończoności'', ,,identyczność'' oraz ,,punkt rzędu dwa''
stanie się jasne w dalszej części pracy.

W definicji \ref{elliptic_curve_definition} ciało $\K$ może być dowolne,
w szczególności może być skończone lub nie
oraz może mieć dowolną charakterystykę.
Charakterystyka równa $2$ lub $3$ jest źródłem wielu trudności,
np. już sama definicja krzywej eliptycznej
nie jest odpowiednia w takiej sytuacji.

\begin{remark}
Zakładamy odtąd, o ile nie będzie zaznaczone inaczej,
że charakterystyka ciała $\K$, nad którym rozważamy krzywwe eliptyczne,
jest różna od $2$ oraz $3$.
\end{remark}

Jest jeszcze jedno źródło trudności,
którym nie będziemy zajmować się w niniejszej pracy.

\begin{definition}
Krzywa eliptyczna $E_{A,B}(\K)$ jest \emph{zdegenerowana},
jeżeli wielomian
$$x^3 + Ax + B$$
ma w domknięciu algebraicznym $\overline{\K}$ ciała $\K$
pierwiastek wielokrotny.
Jeżeli krzywa nie jest zdegenerowana,
to mówimy, że jest \emph{niezdegenerowana}.
\end{definition}

\begin{remark}
Zakładamy odtąd, o ile nie będzie zaznaczone inaczej,
że krzywwe eliptyczne, które rozpatrujemy,
są niezdegenerowane.
\end{remark}

\begin{fact}
Jeżeli charakterystyka ciała $\K$ nie jest równa $2$ ani $3$,
to krzywa eliptyczna nad ciałem $\K$ o parametrach $A$ oraz $B$
jest zdegenerowana wtedy i tylko wtedy,
gdy $\frac{A^3}{27} + \frac{B^2}{4} = 0$.
\end{fact}
