\section{Linie}

Specyficzną klasą wielomianów nad krzywą eliptyczną
są odpowiedniki wielomianów liniowych dwóch zmiennych.
Stanowią one algebraiczny odpowiednik
geometrycznego pojęcia linii prostej.

\subsection*{Definicja}

Przyjmujemy następującą definicję linii.

\begin{definition}
Dana jest krzywa eliptyczna $E(\overline{\K})$.
\emph{Linia nad krzywą $E$}
to wielomian $l$ w postaci $l(x, y) = ax + by +c$,
gdzie $a, b, c \in \K$ oraz $a$ i $b$ nie są jednocześnie równe zero.
\end{definition}

\begin{definition}
Dana jest krzywa eliptyczna $E$ oraz linia $l$ nad tą krzywą.
Mówimy, że \emph{linia $l$ przecina krzywą $E$ w punkcie $P$},
jeśli $l(P) = 0$.
\end{definition}

\subsection*{Własności}

Następujące własności są to po prostu podane wcześniej fakty
zastosowane do przypadku linii.

\begin{fact}
Dana jest linia $l = ax + by + c$.
Jej stopień wynosi $3$, gdy $b \neq 0$
oraz $2$, gdy $b = 0$.
\end{fact}

\begin{fact}\label{line_divisor_norm}
Dana jest linia $l$.
Wówczas $\abs{\rdiv(l)} \in \{2,3\}$.
\end{fact}

Następująca własność jest istotnym uściśleniem faktu \ref{line_divisor_norm}.

\begin{theorem}
Dana jest linia $l$ nad krzywą eliptyczną $E$.
Niech $P$, $Q$ oraz $R$ oznaczają dowolne trzy parami różne punkty krzywej $E$.
Wówczas jej dywizor $\rdiv{l}$ może mieć jedną z następujących postaci:
\begin{itemize}
\item $\rdiv{l} = \divi{P} + \divi{Q} + \divi{R} - 3\divi{\ecident}$;
\item $\rdiv{l} = 2\divi{P} + \divi{Q} - 3\divi{\ecident}$;
\item $\rdiv{l} = 3\divi{P} - 3\divi{\ecident}$;
\item $\rdiv{l} = \divi{P} + \divi{Q} - 2\divi{\ecident}$;
\item $\rdiv{l} = 2\divi{P} - 2\divi{\ecident}$.
\end{itemize}
Wszystkie wymienione przypadki zdarzają się.
\end{theorem}

\begin{corollary}
Przez dowolne dwa punkty skończone krzywej eliptycznej
można przeprowadzić linię.
\end{corollary}

Wniosek ten daje podstawę, na której można zdefiniować
działanie grupowe na krzywej eliptycznej.
