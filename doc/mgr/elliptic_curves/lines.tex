\section{Linie}

Specyficzną klasą wielomianów na krzywej eliptycznej
są odpowiedniki wielomianów liniowych dwóch zmiennych.
Stanowią one algebraiczny odpowiednik
geometrycznego pojęcia linii prostej.

\subsection*{Definicja}

Przyjmujemy następującą definicję linii na krzywej eliptycznej.

\begin{definition}
\emph{Linia na krzywej eliptycznej $E(\K)$}
to wielomian $l$ postaci $l = ax + by +c$,
gdzie $a, b, c \in \K$
oraz współczynniki $a$ i $b$ nie są jednocześnie równe $0$.
\end{definition}

\begin{definition}
Dana jest krzywa eliptyczna $E$ oraz linia $l$ na tej krzywej.
Mówimy, że \emph{linia $l$ przecina krzywą $E$ w punkcie $P$},
jeśli $l(P) = 0$.
\end{definition}

\begin{definition}
Dana jest krzywa eliptyczna $E$ oraz linia $l$ na tej krzywej.
Mówimy, że \emph{linia $l$ jest styczna do krzywej $E$ w punkcie $P$},
jeśli $\ord_P(l) > 1$.
\end{definition}

\subsection*{Własności}

Następujące własności są to po prostu podane wcześniej fakty
zastosowane do przypadku linii.

\begin{fact}
Dana jest linia $l = ax + by + c$.
Jej stopień wynosi $3$, gdy $b \neq 0$
oraz $2$, gdy $b = 0$.
\end{fact}

\begin{fact}\label{line_divisor_norm}
Dana jest linia $l$.
Wówczas $\abs{\rdiv(l)} \in \{2,3\}$.
\end{fact}

Następująca własność jest istotnym uściśleniem faktu \ref{line_divisor_norm}.

\begin{theorem}\label{possible_line_divisors}
Dana jest linia $l$ na krzywej eliptycznej $E$.
Niech $P$, $Q$ oraz $R$ oznaczają dowolne trzy parami różne punkty krzywej $E$.
Wówczas dywizor $\rdiv(l)$ linii $l$ może mieć jedną z następujących postaci:
\begin{itemize}
\item $\rdiv(l) = \divi{P} + \divi{Q} + \divi{R} - 3\divi{\ecident}$;
\item $\rdiv(l) = 2\divi{P} + \divi{Q} - 3\divi{\ecident}$;
\item $\rdiv(l) = 3\divi{P} - 3\divi{\ecident}$;
\item $\rdiv(l) = \divi{P} + \divi{Q} - 2\divi{\ecident}$;
\item $\rdiv(l) = 2\divi{P} - 2\divi{\ecident}$.
\end{itemize}
\end{theorem}

\begin{remark}
Składniki dywizorów wymienionych w twierdzeniu \ref{possible_line_divisors}
są pogrupowane. Czasami nie będziemy ich grupować,
tylko zapisywać w jednej z następujących ogólnych postaci:
\begin{itemize}
\item $\rdiv(l) = \divi{P} + \divi{Q} + \divi{R} - 3\divi{\ecident}$ --
jest to najbardziej ogólna forma,
może zdarzyć się, że punkty $P$, $Q$ oraz $R$ nie są parami różne
lub że jeden z nich nie jest skończony;
\item $\rdiv(l) = \divi{P} + \divi{Q} - 2\divi{\ecident}$ --
jest to ogólna postać linii pionowych,
tzn. takich, które mają współczynnik przy zmiennej $y$ równy $0$,
może zdarzyć się, że $P = Q$ (za to zawsze $Q = \overline{P}$).
\end{itemize}
\end{remark}

Następujące dwa twierdzenia dają podstawę,
która posłuży do zdefiniowania operacji grupowej
na zbiorze punktów krzywej eliptycznej.

\begin{theorem}\label{line_through_two_points_theorem}
Przez dowolne dwa punkty skończone krzywej eliptycznej
można przeprowadzić linię.
\end{theorem}

\begin{theorem}\label{line_tangent_at_point_theorem}
Przez dowolny punkt skończony krzywej eliptycznej
można poprowadzić linię
styczną do krzywej w tym punkcie.
\end{theorem}

Współczynniki linii, których istnienie postulują te twierdzenia,
da się obliczyć na podstawie współrzędnych danych punktów.
Dokładne obliczenia przeprowadzimy w następnym rozdziale.
