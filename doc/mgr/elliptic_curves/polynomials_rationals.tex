\section{Wielomiany i funkcje wymierne}

\noindent
Nasze rozważania w dalszej części pracy
będą często dotyczyć
wielomianów i funkcji wymiernych na krzywych eliptycznych,
dlatego też przeanalizujemy teraz te pojęcia.
Wykazują one podobieństwo do wielomianów i funkcji wymiernych jednej zmiennej,
są też pewne różnice.

\subsection*{Definicja wyrażenia wielomianowego}

\noindent
Przyjmujemy następującą definicję
wyrażenia wielomianowego na krzywej eliptycznej.

\begin{definition}
Dana jest krzywa eliptyczna $E$ nad ciałem $\K$.
\emph{Wyrażenie wielomianowe na krzywej $E$}
to element pierścienia ilorazowego
$\K[x,y]/(\kappa(E) - y^2)$.
Pierścień wszystkich wyrażeń wielomianowych na krzywej $E$
oznaczamy symbolem $\K[E]$.
\end{definition}

\noindent
Sens powyższej definicji jest następujący.
Dowolna krzywa eliptyczna nad ciałem $\K$ to podzbiór zbioru $\K \times \K$
(nie licząc punktu w nieskończoności),
dlatego pierwszym kandydatem na pierścień wyrażeń wielomianowych
jest pierścień $\K[x,y]$.
Współrzędne skończonych punktów krzywej spełniają równanie krzywej,
dlatego też wielomiany dwóch zmiennych różniące się
o wielokrotność wielomianu $\kappa- y^2$
dadzą tę samą funkcję wielomianową,
niezależnie od wyboru ciała $\K$.
Stąd pierścień $\K[x,y]$ dzielimy przez ideał $(\kappa- y^2)$
i uzyskujemy pierścień ilorazowy $K[x,y]/(\kappa - y^2)$.

\noindent
Wyrażenie wielomianowe na krzywej eliptycznej
jest w takim razie klasą abstrakcji
pewnej relacji równoważności na zbiorze wielomianów dwóch zmiennych.
Nie jest wygodnie myśleć o wyrażeniach wielomianowych w ten sposób.
Następujący lemat pokazuje,
w jaki sposób wybrać reprezentantów klas abstrakcji
do dalszych rozważań.

\begin{theorem}
Dane jest wyrażenie wielomianowe $f$ na krzywej eliptycznej nad ciałem $\K$.
Wówczas istnieją wyrażenia wielomianowe $u, v \in \K[x]$ takie,
że $f = u + yv$.
Wyrażenia $u$ i $v$ są wyznaczone jednoznacznie.
\end{theorem}

\begin{remark}
Zbiór wyrażeń wielomianowych na krzywej eliptycznej nad ciałem $\K$
będziemy odtąd utożsamiać ze zbiorem
złożonym z wielomianów postaci $u + yv$,
gdzie $u, v \in \K[x]$.
Zbiór ten będziemy oznaczać $(1 + y)\K[x]$.
\end{remark}

\noindent
Zwróćmy uwagę, że zbiór $(1 + y)\K[x]$ nie ma struktury pierścienia,
ponieważ nie jest określone mnożenie.
Nie jest to problem,
ponieważ z kontekstu zawsze będzie wynikało,
na jakiej krzywej rozpatrywane są wyrażenia wielomianowe.
Co więcej, czasami wystarczy,
że w miejsce powstałego podczas mnożenia czynnika $y^2$
będziemy podstawiać symbol $\kappa$
i nie będzie nam potrzebna dokładna znajomość parametrów $A$ i $B$ krzywej.

\begin{remark}
W zależności od kontekstu wielomian charakterystyczny krzywej eliptycznej
będziemy traktować jak wielomian jednej zmiennej, wielomian dwóch zmiennych
lub jak wyrażenie wielomianowe na krzywej eliptycznej.
\end{remark}

\subsection*{Definicja wyrażenia wymiernego}

\noindent
Przyjmujemy następującą definicję
wyrażenia wymiernego na krzywej eliptycznej.

\begin{definition}
Dana jest krzywa eliptyczna $E$ nad ciałem $\K$.
\emph{Wyrażenie wymierne na krzywej $E$}
to element ciała ułamków pierścienia $\K[E]$.
Ciało wszystkich wyrażeń wymiernych na krzywej $E$
oznaczamy symbolem $\K(E)$.
\end{definition}

\noindent
Sens tej definicji jest taki sam,
jak w przypadku każdego innego ciała ułamków --
bierzemy zbiór ułamków formalnych $\frac{f}{g}$,
gdzie $f$ i $g$ to wielomiany na krzywej eliptycznej,
po czym utożsamiamy ułamki $\frac{f_1}{g_1}$ oraz $\frac{f_2}{g_2}$,
jeżeli zachodzi następująca równość wyrażeń wielomianowych:
$f_1g_2 = f_2g_1$.

\begin{fact}
Zbiór wyrażeń wielomianowych na krzywej eliptycznej
można zanurzyć w zbiór wyrażeń wymiernych na tej krzywej,
przypisując wyrażeniu wielomianowemu $f$ wyrażenie wymierne $\frac{f}{1}$
\end{fact}

\begin{remark}
Wyrażenia wielomianowe na krzywej eliptycznej
będziemy odtąd utożsamiać z odpowiadającymi im wyrażeniami wymiernymi.
\end{remark}

\noindent
Podobnie jak wyrażenia wielomianowe,
wyrażenia wymierne są klasami abstrakcji pewnej relacji równoważności,
co nie jest wygodne.

\begin{theorem}
Dane jest wyrażenie wymierne $r$ na krzywej eliptycznej nad ciałem $\K$.
Wówczas istnieją wyrażenia wymierne $u, v \in \K(x)$ takie,
że $r = u + yv$.
Wyrażenia $u$ i $v$ są wyznaczone jednoznacznie.
\end{theorem}

\begin{remark}
Zbiór wyrażeń wymiernych na krzywej eliptycznej nad ciałem $\K$
będziemy odtąd utożsamiać ze zbiorem
złożonym z wyrażeń wymiernych postaci $u + yv$,
gdzie $u,v \in \K(x)$.
Zbiór ten będziemy oznaczać $(1 + y)\K(x)$.
\end{remark}

\noindent
Podobnie jak w przypadku zbioru $(1 + y)\K[x]$,
zbiór $(1 + y)\K(x)$ nie ma struktury ciała --
nie można mnożyć ani dzielić.
Jednak zawsze albo będą dane parametry krzywej eliptycznej,
albo w miejsce powstałego podczas mnożenia lub dzielenia czynnika $y^2$
będziemy podstawiać symbol $\kappa$.

\noindent
Wyrażenia wymierne możemy również przedstawiać w postaci ilorazu
elementów zbioru $(1 + y)\K[x]$.

\begin{theorem}
Dane jest wyrażenie wymierne $r$ na krzywej eliptycznej nad ciałem $\K$.
Wówczas istnieją wyrażenia wielomianowe $f, g \in (1 + y)\K[x]$ takie,
że $r = \frac{f}{g}$.
\end{theorem}

\subsection*{Funkcje wielomianowe i wymierne}

\noindent
Wyrażenia wielomianowe wyznaczają funkcje wielomianowe,
a wyrażenia wymierne -- funkcje wymierne.
Wyrażenia i funkcje wielomianowe będziemy określać
wspólnym mianem \emph{wielomian}.
W przypadku wyrażeń i funkcji wymiernych nie ma trzeciego określenia,
którym można by je wspólnie nazwać,
będziemy więc posługiwać się określeniem \emph{funkcja wymierna}.

\begin{remark}
Funkcję wymierną wyznaczoną przez wyrażenie wymierne $r$
oznaczać będziemy symbolem $r(x, y)$,
gdzie $x$ oraz $y$ to współrzędne punktu skończonego krzywej,
lub symbolem $r(P)$, gdzie $P$ to punkt skończony krzywej.
\end{remark}

\noindent
Jest jasne, jak na podstawie wyrażenia wymiernego
obliczyć wartość odpowiadającej mu funkcji wymiernej
w dowolnym punkcie skończonym krzywej.
Jedyna ewentualna trudność pojawia się,
gdy wartość mianownika jest równa zero.

\begin{definition}
Dana jest funkcja wymierna $r$ na krzywej eliptycznej $E$
oraz punkt skończony $P \in E$.
Funkcja $r$ ma w punkcie $P$ \emph{wartość nieskończoną},
co zapisujemy jako $r(P) = \infty$,
jeżeli istnieją wielomiany $f$ i $g$ na krzywej $E$ takie,
że $r = \frac{f}{g}$, $f(P) \neq 0$ oraz $g(P) = 0$.
\end{definition}

\begin{theorem}
Dana jest funkcja wymierna $r$ na krzywej eliptycznej $E$
oraz punkt skończony $P \in E$.
Niech $f$ i $g$ będą dowolnymi wielomianami na krzywej $E$ takimi,
że $r = \frac{f}{g}$.
Wówczas jeżeli $r(P) = 0$, to $f(P) = 0$.
\end{theorem}

\begin{theorem}
Dana jest funkcja wymierna $r$ na krzywej eliptycznej $E$
oraz punkt skończony $P \in E$.
Niech $f$ i $g$ będą dowolnymi wielomianami na krzywej $E$ takimi,
że $r = \frac{f}{g}$.
Wówczas jeżeli $r(P) = \infty$, to $g(P) = 0$.
\end{theorem}

\subsection*{Sprzężenie i norma}

\noindent
Następujące dwa pojęcia są bardzo przydatne,
ponieważ pozwalają sprowadzić
zagadnienie dotyczące wielomianów i funkcji wymiernych na krzywej eliptycznej
do przypadku wyrażeń jednej zmiennej.

\begin{definition}
Dana jest funkcja wymierna $r$ na krzywej eliptycznej nad ciałem $\K$
postaci $r = u + yv$, gdzie $u, v \in \K(x)$.
\emph{Funkcja wymierna sprzężona do funkcji $r$},
oznaczana symbolem $\overline{r}$,
to funkcja $u - yv$.
\end{definition}

\begin{definition}
Dana jest funkcja wymierna $r$ na krzywej eliptycznej nad ciałem $\K$
postaci $r = u + yv$, gdzie $u, v \in \K(x)$.
\emph{Norma funkcji wymiernej $r$},
oznaczana symbolem $N(r)$,
to funkcja wymierna jednej zmiennej
$r\overline{r} = (u + yv)(u - yv) = u^2 - y^2v^2 = u^2 - \kappa v^2$.
\end{definition}

\begin{fact}
Dla dowolnych funkcji wymiernych $r$ i $s$ na krzywej eliptycznej
zachodzą zależności
$\overline{rs} = \overline{r}\,\overline{s}$
oraz $N(rs) = N(r)N(s)$.
\end{fact}

\subsection*{Stopień wielomianu i funkcji wymiernej}

\noindent
Przyjmujemy następującą definicję stopnia wielomianu na krzywej eliptycznej.

\begin{definition}
Dany jest wielomian $f$ na krzywej eliptycznej.
\emph{Stopień wielomianu $f$},
oznaczany symbolem $\deg(f)$,
to stopień jego normy $N(f)$ traktowanej jak wielomian jednej zmiennej.
\end{definition}

\noindent
Sens tej definicji jest następujący.
W pierścieniu $\K[x, y]$ wielomiany $x$ i $y$ mają stopień równy $1$.
Na tej podstawie można obliczyć stopień dowolnego innego wielomianu.
Tę koncepcję przenosimy na wielomiany na krzywej eliptycznej,
ale w taki sposób, żeby uwzględnić równanie krzywej.
Zatem wielomianowi $x$ przypisujemy stopień $2$,
a wielomianowi $y$ stopień $3$,
dzięki czemu stopień wielomianów po obu stronach równania krzywej jest taki sam.
Ponadto, do obliczania stopnia wielomianu $f$
wybieramy reprezentanta klasy abstrakcji $f$ w postaci $u + yv$,
gdzie $u$ i $v$ to wielomiany zmiennej $x$.

\begin{remark}
Aby uniknąć nieporozumień,
stopień wielomianu jednej zmiennej $x$
oznaczać będziemy symbolem $\deg_x$.
\end{remark}

\begin{fact}
Dla dowolnego wielomianu $f$ na krzywej eliptycznej
zachodzi zależność $\deg(f) = \max(2\deg_x(u), 3 + 2\deg_x(v))$,
gdzie $u$ i $v$ są takimi wielomianami zmiennej $x$, że $f = u + yv$.
\end{fact}

\noindent
Dysponując stopniem wielomianu możemy określić stopień funkcji wymiernej.

\begin{definition}
Dana jest funkcja wymierna $r$ na krzywej eliptycznej nad ciałem $\K$
postaci $r = \frac{f}{g}$, gdzie $f, g \in \K[E]$.
\emph{Stopień funkcji wymiernej $r$},
oznaczany symbolem $\deg(r)$,
to wielkość $\deg(f) - \deg(g)$.
\end{definition}

\begin{theorem}
Dana jest funkcja wymierna $r$ na krzywej eliptycznej $E$.
Wówczas stopień funkcji $r$ jest dobrze określony,
tzn. dla dowolnych wielomianów $f_1, g_1, f_2, g_2$ na krzywej $E$ takich,
że $\frac{f_1}{g_1} = r = \frac{f_2}{g_2}$,
zachodzi $\deg(f_1) - \deg(g_1) = \deg(f_2) - \deg(g_2)$.
\end{theorem}

\noindent
Na stopnie wielomianów i funkcji wymiernych na krzywej eliptycznej
przenosi się zasadnicza własność znana z teorii wyrażeń jednej zmiennej.

\begin{theorem}
Dane są funkcje wymierne $r$ i $s$ na krzywej eliptycznej.
Wówczas $\deg(rs) = \deg(r) + \deg(s)$.
\end{theorem}

\subsection*{Wartość funkcji wymiernej w punkcie w nieskończoności}

\noindent
Chcemy określić wartość funkcji wymiernej (zatem także wielomianu)
w punkcie w nieskończoności.
W przypadku funkcji wymiernych nad ciałem liczb rzeczywistych
obliczamy po prostu granicę wartości funkcji,
gdy argument dąży do nieskończoności.
Uzyskana w ten sposób wartość, jeśli jest skończona,
jest po prostu ilorazem współczynników stojących przy najwyższych potęgach
w mianowniku i liczniku.
Dzięki tej obserwacji możemy w analogiczny sposób określić
wartość funkcji wymiernej w punkcie w nieskończoności.

\begin{definition}
Dana jest funkcja wymierna $r$ na krzywej eliptycznej.
Wartość funkcji $r$ w punkcie $\ecident$ ustalamy następująco:
\begin{itemize}
\item jeżeli $\deg(r) < 0$, to $r(\ecident) = 0$;
\item jeżeli $\deg(r) > 0$, to $r(\ecident) = \infty$;
\item jeżeli $\deg(r) = 0$,
to przedstawiamy funkcję $r$ w postaci $\frac{f}{g}$,
gdzie $f$ i $g$ to wielomiany na krzywej i wówczas:
\begin{itemize}
\item jeżeli stopnie wielomianów $f$ i $g$ są parzyste,
to ich wiodące składniki mają postać odpowiednio $ax^d$ i $bx^d$,
wówczas $r(\ecident) = \frac{a}{b}$;
\item jeżeli stopnie wielomianów $f$ i $g$ są nieparzyste,
to ich wiodące składniki mają postać odpowiednio $ayx^d$ i $byx^d$,
wówczas również $r(\ecident) = \frac{a}{b}$.
\end{itemize}
\end{itemize}
\end{definition}

\noindent
Pod wieloma względami funkcje wymierne zachowują się
w punkcie $\ecident$ i punktach skończonych podobnie,
co pokazuje następujące twierdzenie.

\begin{theorem}
Dane są funkcje wymierne $r$ i $s$ na krzywej eliptycznej.
Jeżeli $r(\ecident) \neq \infty$ oraz $s(\ecident) \neq \infty$,
to $(r + s)(\ecident) = r(\ecident) + s(\ecident)$
oraz $(rs)(\ecident) = r(\ecident)s(\ecident)$.
\end{theorem}

\subsection*{Miejsca zerowe i bieguny}

\noindent
Przyjmujemy następującą definicję
miejsca zerowego oraz bieguna funkcji wymiernej na krzywej eliptycznej.

\begin{definition}
Dana jest funkcja wymierna $r$ na krzywej eliptycznej $E$.
\emph{Miejsce zerowe funkcji $r$} (odpowiednio, \emph{biegun funkcji $r$})
to taki punkt $P \in E$, że $r(P) = 0$ (odpowiednio, $r(P) = \infty$).
\end{definition}

\begin{fact}
Jeżeli punkt skończony $P$
jest miejscem zerowym (biegunem) funkcji wymiernej $r$,
to punkt skończony $\overline{P}$
jest miejscem zerowym (biegunem) funkcji wymiernej $\overline{r}$.
\end{fact}

\begin{corollary}\label{zero_of_norm_coro}
Jeżeli punkt skończony $P$ postaci $P = (a, b)$
jest miejscem zerowym wielomianu $f$,
to wartość $a$ jest miejscem zerowym wielomianu $N(f)$.
\end{corollary}

\begin{fact}
Punkty rzędu dwa na krzywej eliptycznej
to miejsca zerowe wielomianu charakterystycznego tej krzywej.
Niezdegenerowana krzywa eliptyczna nad ciałem algebraicznie domkniętym
ma dokładnie trzy punkty rzędu dwa.
\end{fact}

\noindent
Podczas badania miejsc zerowych i biegunów
funkcji wymiernych na krzywej eliptycznej
będziemy chcieli uwolnić się od konieczności pilnowania
czy ciało, nad którym zdefiniowana jest krzywa,
jest algebraicznie domknięte czy nie.

\begin{fact}
Dane są dwa ciała $\K$ i $\fieldL$ takie, że $\K \subset \fieldL$
oraz parametry $A,B \in \K$.
Wówczas $E_{A,B}(\K) \subset E_{A,B}(\fieldL)$.
\end{fact}

\noindent
Fakt ten pozwala nam uwolnić się od pytania
czy ciało $\K$ jest algebraicznie domknięte --
rozważania na temat krzywej nad ciałem $\K$
zawsze możemy potraktować
jak rozważania na temat krzywej o tych samych parametrach
nad większym ciałem $\fieldL = \overline{\K}$,
która jest nadzbiorem danej krzywej.

\begin{definition}
Dane są dwa ciała $\K \subset \fieldL$ oraz parametry $A, B \in \K$.
\emph{Punkty $\K$-wymierne na krzywej $E_{A,B}(\fieldL)$}
to te punkty krzywej $E_{A,B}(\fieldL)$,
które są jednocześnie punktami krzywej $E_{A,B}(\K)$.
\end{definition}

\begin{remark}
Do końca tego rozdziału przyjmujemy,
że rozpatrywane krzywe są określone nad ciałami algebraicznie domkniętymi.
\end{remark}

\noindent
Podobnie jak w przypadku funkcji wymiernych jednej zmiennej,
chcemy wprowadzić pojęcie krotności miejsca zerowego i bieguna.

\begin{theorem}\label{uniformizer_existence_theorem}
Dana jest krzywa eliptyczna $E$ nad ciałem $\K$
oraz punkt $P \in E$.
Wówczas istnieje funkcja wymierna $u \in \K(E)$ taka, że $u(P) = 0$
oraz dla każdej funkcji wymiernej $r \in \K(E)$
istnieje liczba całkowita $d \in \Z$
oraz funkcja wymierna $s \in \K(E)$ taka,
że $s(P) \neq 0$ oraz zachodzi następująca równość:
\begin{equation}\label{uniformizer_decomposition_eqn}
r = u^ds
\end{equation}
Liczba $d$ nie zależy od wyboru funkcji $u$.

\noindent
Przykładowymi funkcjami, które spełniają powyższy warunek, są:
\begin{itemize}
\item jeżeli $P = (a, 0)$, to $u(x, y) = y$;
\item jeżeli $P = (a, b)$, gdzie $b \neq 0$, to $u(x, y) = x - a$;
\item jeżeli $P = \ecident$, to $u(x, y) = \frac{y}{x}$.
\end{itemize}
\end{theorem}

\noindent
Twierdzenie to ukazuje pewną wspólną cechę
funkcji wymiernych na krzywej eliptycznej
oraz zwykłych funkcji wymiernych,
dzięki czemu może posłużyć do przeniesienia na krzywe eliptyczne
pojęcia krotności miejsca zerowego lub bieguna.

\begin{definition}
Dany jest punkt $P$ na krzywej eliptycznej $E$.
\emph{Unifikator w punkcie $P$}
to dowolna funkcja wymierna $u$ na krzywej $E$,
której istnienie postuluje twierdzenie \ref{uniformizer_existence_theorem}.
\end{definition}

\begin{definition}
Dana jest funkcja wymierna $r$ na krzywej eliptycznej $E$
oraz punkt $P \in E$.
Niech $u$ będzie dowolnym unifikatorem w punkcie $P$.
\emph{Rząd funkcji $r$ w punkcie $P$},
oznaczany symbolem $\ord_P(r)$,
to liczba całkowita $d$
występująca w równości \ref{uniformizer_decomposition_eqn}.
Ponadto:
\begin{itemize}
\item jeżeli $d = 0$,
to funkcja $r$ nie ma w punkcie $P$ ani miejsca zerowego, ani bieguna;
\item jeżeli $d > 0$, to mówimy,
że \emph{funkcja $r$ ma w punkcie $P$ $d$-krotne miejsce zerowe};
\item jeżeli $d < 0$, to mówimy,
że \emph{funkcja $r$ ma w punkcie $P$ $\abs{d}$-krotny biegun}.
\end{itemize}
\end{definition}

\noindent
Rząd funkcji wymiernej spełnia zależność
podobną do tej związanej ze stopniem funkcji.

\begin{theorem}
Dane są funkcje wymierne $r$ i $s$ na krzywej eliptycznej $E$
oraz punkt skończony $P \in E$.
Wówczas $\ord_P(rs) = \ord_P(r) + \ord_P(s)$.
\end{theorem}

\noindent
Możemy teraz wyrazić szereg ważnych własności
wielomianów i funkcji wymiernych na krzywej eliptycznej
związanych z miejscami zerowymi i biegunami.

\begin{theorem}\label{polynomial_ord_deg_theorem}
Dany jest wielomian $f$ na krzywej eliptycznej $E$.
Wówczas:
\begin{equation*}
\sum_{P \in E\setminus \{\ecident\}} \ord_P(f) = \deg(f)
\end{equation*}
\end{theorem}

\begin{corollary}\label{function_order_sum_zero_corollary}
Dana jest funkcja wymiena $r$ na krzywej eliptycznej $E$.
Wówczas:
\begin{equation*}
\sum_{P \in E} \ord_P(r) = 0
\end{equation*}
\end{corollary}

\begin{corollary}
Funkcja wymierna na krzywej eliptycznej ma miejsce zerowe lub biegun
tylko w skończonej liczbie punktów.
\end{corollary}

\begin{corollary}
Krotność każdego miejsca zerowego lub bieguna
funkcji wymiernej na krzywej eliptycznej
jest skończona.
\end{corollary}

\noindent
Ze względu na definicję stopnia wielomianu na krzywej eliptycznej
możemy jeszcze wysnuć następujący wniosek.

\begin{corollary}\label{poly_no_single_zero_corollary}
Wielomian ma krzywej eliptycznej nie może mieć stopnia równego $1$,
zatem nie może mieć jednego jednokrotnego miejsca zerowego.
\end{corollary}
