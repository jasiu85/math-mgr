\section{Wielomiany i funkcje wymierne}

\subsection*{Definicja wyrażenia wielomianowego}

Przyjmujemy następującą definicję
wyrażenia wielomianowego nad krzywą eliptyczną.

\begin{definition}
\emph{Wyrażenie wielomianowe nad krzywą eliptyczną $E_{A,B}(\K)$}
to element pierścienia ilorazowego
$\K[x,y]/(x^3 + Ax + B - y^2)$.
\end{definition}

Sens powyższej definicji jest następujący.
Dowolna krzywa eliptyczna nad ciałem $\K$ to podzbiór zbioru $\K \times \K$
(nie licząc punktu w nieskończoności),
dlatego pierwszym kandydatem na pierścień wyrażeń wielomianowych
jest pierścień $\K[x,y]$.
Współrzędne skończonych punktów krzywej spełniają równanie krzywej,
dlatego też wielomiany dwóch zmiennych różniące się
o wielokrotność wielomianu $x^3 + Ax + B - y^2$
dadzą tę samą funkcję wielomianową.
Stąd pierścień $\K[x,y]$ dzielimy przez ideał $(x^3 + Ax + B - y^2)$
i uzyskujemy pierścień ilorazowy $K[x,y]/(x^3 + Ax + B - y^2)$.

Wyrażenie wielomianowe nad krzywą eliptyczną
jest w takim razie klasą abstrakcji
pewnej relacji równoważności na zbiorze wielomianów dwóch zmiennych.
Nie jest wygodnie myśleć o wyrażeniach wielomianowych w ten sposób.
Następujący lemat pokazuje,
w jaki sposób wybrać reprezentantów klas abstrakcji
do dalszych rozważań.

\begin{lemma}
Dana jest krzywa eliptyczna $E_{A,B}(\K)$.
Niech $f \in \K[x,y]$
będzie wielomianem dwóch zmiennych nad ciałem $\K$.
Wówczas istnieje taki wielomian $g \in \K[x,y]$,
że $g = u + yv$, gdzie $u, v \in \K[x]$
oraz $g$ różni się od $f$ o wielokrotność $s - y^2$.
Ponadto, wielomian $g$ jest wyznaczony jednoznacznie.
\end{lemma}

\begin{remark}
Pierścień wyrażeń wielomianowych nad krzywą eliptyczną $E_{A,B}(\K)$
będziemy odtąd utożsamiać ze zbiorem
złożonym z wielomianów postaci $u + yv$,
gdzie $u, v \in K[x]$.
Zbiór ten będziemy oznaczać $(1 + y)\K[x]$.
\end{remark}

Zwróćmy uwagę, że zbiór $(1 + y)\K[x]$ nie ma struktury pierścienia,
ponieważ nie jest określone mnożenie.
Nie jest to problem,
ponieważ z kontekstu zawsze będzie wynikało,
nad jaką krzywą rozpatrywane są wyrażenia wielomianowe.
Co więcej, czasami wystarczy,
że podczas mnożenia będziemy podstawiać $s$ w miejsce $y^2$
i nie będzie nam potrzebna dokładna znajomość parametrów $A$ oraz $B$.

Przy badaniu wyrażeń wielomianowych nad krzywą eliptyczną $E_{A,B}(\K)$
wyróżnioną pozycję zajmuje wyrażenie $x^3 + Ax + B$.

\begin{definition}
\emph{Wielomian charakterystyczny krzywej eliptycznej $E_{A,B}(\K)$}
to wyrażenie wielomianowe $s = x^3 + Ax + B$.
\end{definition}

W zależności od kontekstu wielomian charakterystyczny
może oznaczać wielomian jednej zmiennej, wielomian dwóch zmiennych
lub wyrażenie wielomianowe nad krzywą eliptyczną.

\subsection*{Definicja wyrażenia wymiernego}

Przyjmujemy następującą definicję
wyrażenia wymiernego nad krzywą eliptyczną.

\begin{definition}
\emph{Wyrażenie wymierne nad krzywą eliptyczną $E_{A,B}(\K)$}
to element ciała ułamków pierścienia wyrażeń wielomianowych nad tą krzywą.
\end{definition}

Sens tej definicji jest taki sam,
jak w przypadku każdego innego ciała ułamków --
bierzemy zbiór ułamków formalnych $\frac{f}{g}$,
gdzie $f$ i $g$ to wyrażenia wielomianowe nad krzywą eliptyczną,
po czym utożsamiamy ułamki $\frac{f_1}{g_1}$ oraz $\frac{f_2}{g_2}$,
jeżeli zachodzi równość wyrażeń $f_1g_2 = f_2g_1$.
Podobnie jak wyrażenia wielomianowe,
wyrażenia wymierne są klasami abstrakcji pewnej relacji równoważności,
co nie jest wygodne.

\begin{lemma}
Niech $r$ będzie wyrażeniem wymiernym nad krzywą eliptyczną.
Wówczas istnieją wyrażenia wymierne $u, v \in \K(x)$ takie,
że $r = u + yv$.
Wyrażenia te są wyznaczone jednoznacznie.
\end{lemma}

\begin{remark}
Ciało funkcji wymiernych nad krzywą eliptyczną $E_{A,B}(\K)$
będziemy odtąd utożsamiać ze zbiorem
złożonym z wyrażeń wymiernych postaci $u + yv$,
gdzie $u,v \in \K(x)$.
Zbiór ten będziemy oznaczać $(1 + y)\K(x)$.
\end{remark}

Podobnie jak w przypadku zbioru $(1 + y)\K[x]$,
zbiór $(1 + y)\K(x)$ nie ma struktury ciała,
ale zawsze albo będzie można podstawić symbol $s$ w miejsce wyrażenia $y^2$,
albo będą dane parametry krzywej eliptycznej.

Wyrażenia wymierne będziemy również przedstawiać w postaci ilorazu
elementów zbioru $(1 + y)\K[x]$.

\begin{lemma}
Dane jest wyrażenie wymierne $r$ nad krzywą eliptyczną.
Wówczas istnieją wyrażenia wymierne
$f = u_1 + yv_1$ oraz $g = u_2 + yv_2$ takie,
że $f = \frac{f}{g}$.
Wyrażenia $f$ i $g$ są wyznaczone jednoznacznie.
\end{lemma}

Zbiór wyrażeń wielomianowych można zanużyć w zbiorze wyrażeń wymiernych,
przypisując wyrażeniu wielomianowemu $f$ wyrażenie wymierne $\frac{f}{1}$

\begin{remark}
Wyrażenia wielomianowe nad krzywą eliptyczną
będziemy odtąd utożsamiać z odpowiadającymi im wyrażeniami wymiernymi.
\end{remark}

\subsection*{Funkcje wielomianowe i wymierne}

Wyrażenia wielomianowe wyznaczają funkcje wielomianowe,
a wyrażenia wymierne -- funkcje wymierne.
Wyrażenia i funkcjie wielomianowe będziemy określać
wspólną nazwą \emph{wielomian}.
W przypadku wyrażeń i funkcji wymiernych nie ma trzeciego określenia,
którym można by je wspólnie nazwać,
będziemy więc posługiwać się określeniem \emph{funkcje wymierne}.

\begin{remark}
Funkcję wymierną wyznaczoną przez wyrażenie wymierne $r$
oznaczać będziemy $r(x, y)$,
gdzie $x$ oraz $y$ to współrzędne punktu skończonego krzywej,
lub $r(P)$, gdzie $P$ to punkt skończony krzywej.
\end{remark}

\begin{remark}
Dana jest funkcja wymierna $r = \frac{f}{g}$ nad krzywą eliptyczną.
Jeżeli $P$ to punkt skończony krzywej oraz $g(P) = 0$,
ustalamy, że funkcja wymierna $r$ ma w punkcie $P$
wartość \emph{nieskończoną},
co zapisujem $r(P) = \infty$.
\end{remark}

\subsection*{Sprzężenie i norma}

Następujące dwa pojęcia są bardzo przydatne,
ponieważ pozwalają sprowadzić
zagadnienie dotyczące wielomianów i funkcji wymiernych nad krzywą eliptyczną
do przypadku wyrażeń jednej zmiennej.

\begin{definition}
Dana jest funkcja wymierna $r = u + yv$ nad krzywą eliptyczną.
\emph{Funkcja wymierna sprzężona do funkcji $f$}
to funkcja $\overline{r} = u - yv$.
\end{definition}

\begin{definition}
Dany jest funkcja wymierna $r = u + yv$ nad krzywą eliptyczną.
\emph{Norma funkcji wymiernej $r$}
to funkcja wymierna $N(r) = r\overline{r} = u^2 - sv^2$.
\end{definition}

\begin{fact}
Dane są funkcje wymierne $r$ oraz $s$ nad krzywą eliptyczną.
Wówczas $\overline{rs} = \overline{r}\,\overline{s}$
oraz $N(rs) = N(r)N(s)$.
\end{fact}

\subsection*{Stopień wielomianu i funkcji wymiernej}

Przyjmujemy następującą definicję stopnia wielomianu nad krzywą eliptyczną.

\begin{definition}
Dany jest wielomian $f$ nad krzywą eliptyczną.
\emph{Stopień wielomianu $f$},
oznaczany $\deg(f)$,
to stopień jego normy $N(f)$ traktowanej jak wielomian jednej zmiennej.
\end{definition}

Sens tej definicji jest następujący. Stopień wielomianu jednej zmiennej
to po prostu wykładnik najwyższej potęgi, w której występuje ta zmienna.
Podobnie rzecz ma się z wielomianami dwóch zmiennych.
Tę koncepcję przenosimy na wielomiany nad krzywą eliptyczną,
ale w taki sposób, żeby uwzględnić równanie krzywej.
Dlatego zmiennej $x$ przypisujemy stopień $2$,
a zmiennej $y$ -- stopień $3$.
Nieprzypadkowo, stopień liczony w ten sposób
jest równy stopniowi normy.

\begin{remark}
Aby uniknąć nieporozumień,
zwykły stopień wielomianu względem zmiennej $x$
oznaczać będziemy przez $\deg_x$.
\end{remark}

\begin{fact}
Niech $f = u + yv$ będzie wielomianem nad krzywą eliptyczną.
Wówczas $\deg(f) = \max(2\deg_x(u), 3 + 2\deg_x(v))$.
\end{fact}

Dysponując stopniem wielomianu, możemy określić stopień funkcji wymiernej.

\begin{definition}
Dana jest funkcja wymierna $r = \frac{f}{g}$ nad krzywą eliptyczną.
\emph{Stopień funkcji wymiernej $r$},
oznaczany $\deg(r)$,
to różnica $\deg(f) - \deg(g)$.
\end{definition}

Na stopnie wielomianów i funkcji wymiernych nad krzywą eliptyczną
przenosi się zasadnicza własność znana z teorii wyrażeń jednej zmiennej.

\begin{theorem}
Niech $r$ oraz $s$ będą dwoma funkcjami wymiernymi nad krzywą eliptyczną.
Wówczas $\deg(rs) = \deg(r) + \deg(s)$.
\end{theorem}

\subsection*{Wartość funkcji wymiernej w punkcie w nieskończoności}

Chcemy określić wartość funkcji wymiernej (zatem także wielomianu)
w punkcie w nieskończoności.
W przypadku funkcji wymiernych nad ciałem liczb rzeczywistych
obliczamy po prostu granicę wartości funkcji,
gdy argument dąży do nieskończoności.
Następująca definicja określa
wartość funkcji wymiernej w punkcie w nieskończoności w sposób,
który ma wiele cech wspólnych ze wspomnianym przejściem granicznym.

\begin{definition}
Dana jest funkcja wymierna $r = \frac{f}{g}$ nad krzywą eliptyczną.
Wartość funkcji $r$ w punkcie w nieskończoności $\ecident$
ustalamy następująco:
\begin{itemize}
\item jeżeli $\deg(r) < 0$, to $r(\ecident) = 0$;
\item jeżeli $\deg(r) > 0$, to $r(\ecident) = \infty$;
\item jeżeli $\deg(r) = 0$, to:
\begin{itemize}
\item stopnie wielomianów $f$ i $g$ są parzyste,
ich wiodące składniki mają postać odpowiednio $ax^d$ i $bx^d$,
wówczas $r(\ecident) = \frac{a}{b}$;
\item stopnie wielomianów $f$ i $g$ są nieparzyste,
ich wiodące składniki mają postać odpowiednio $ayx^d$ i $byx^d$,
wówczas również $r(\ecident) = \frac{a}{b}$.
\end{itemize}
\end{itemize}
\end{definition}

\begin{lemma}
Dane są dwie funkcje wymierne $r$ oraz $s$ takie,
że $r(\ecident) \neq \infty$ oraz $s(\ecident) \neq \infty$.
Wówczas $(r \pm s)(\ecident) = r(\ecident) \pm s(\ecident)$
oraz $(rs)(\ecident) = r(\ecident)s(\ecident)$.
Co więcej, jeśli $s(\ecident) \neq 0$,
to $(\frac{r}{s})(\ecident) = \frac{r(\ecident)}{s(\ecident)}$.
\end{lemma}

\subsection*{Miejsca zerowe i bieguny}

Przyjmujemy następującą definicję
miejsca zerowego oraz bieguna funkcji wymiernej nad krzywą eliptyczną.

\begin{definition}
Dany jest funkcja wymierna $r$ nad krzywą eliptyczną.
\emph{Miejsce zerowe funkcji $r$} (odpowiednio, \emph{biegun funkcji $f$})
to taki punkt $P$ krzywej, że $r(P) = 0$ (odpowiednio, $r(P) = \infty$).
\end{definition}

\begin{fact}
Jeżeli punkt $P$ jest miejscem zerowym (biegunem) funkcji wymiernej $r$,
to punkt $\overline{P}$ jest miejscem zerowym funkcji wymiernej $\overline{r}$.
\end{fact}

Podczas badania miejsc zerowych i biegunów
funkcji wymiernych nad krzywą eliptyczną
będziemy chcieli uwolnić się od konieczności pilnowania,
czy ciało, nad którym zdefiniowana jest krzywa,
jest algebraicznie domknięte czy nie.

\begin{remark}
Symbolem $E_{A,B}(\overline{\K})$
oznaczamy krzywą eliptyczną nad ciałem $\overline{\K}$,
o parametrach $A, B \in \K$.
Ciało $\overline{\K}$ to algebraiczne domknięcie ciała $\K$.
\end{remark}

Miejsca zerowe wielomianu charakterystycznego krzywej eliptycznej
są szczególnie istotne,
ponieważ, jak się okaże w dalszej części pracy,
wiele rozumowań należy przeprowadzić dla nich osobno.

\begin{fact}
Punkty rzędu dwa na krzywej eliptycznej
to miejsca zerowe wielomianu charakterystycznego tej krzywej.
Niezdegenerowana krzywa eliptyczna nad ciałem algebraicznie domkniętym
ma dokładnie trzy punkty rzędu dwa.
\end{fact}

Podobnie jak w przypadku funkcji wymiernych jednej zmiennej,
chcemy wprowadzić pojęcie krotności miejsca zerowego i bieguna.

\begin{lemma}\label{uniformizer_existence_lemma}
Dla każdego punktu $P = (a, b)$ na krzywej eliptycznej
istnieje taka funkcja wymierna $u$,
że każdą funkcję wymierną $r$ można zapisać w postaci $r = u^ds$,
gdzie $d \in \Z$ oraz $s$ jest taką funkcją wymierną, że $s(P) \neq 0$.

Przykładowymi funkcjami, które spełniają powyższy warunek, są:
\begin{itemize}
\item dla punktów rzędu dwa $u(x, y) = y$;
\item dla pozostałych punktów skończonych $u(x, y) = x - a$;
\item dla punktu w nieskończoności $u(x, y) = \frac{y}{x}$.
\end{itemize}

Wartość wykładnika $d$ nie zależy od wyboru funkcji $u$.
\end{lemma}

Lemat ten pokazuje,
że miejsca zerowe i bieguny funkcji wymiernych nad krzywą eliptyczną
mają cechy podobne
do miejsc zerowych i biegunów zwykłych funkcji wymiernych,
więc posłuży nam za definicję krotności.

\begin{definition}
Dany jest punkt $P$ na krzywej eliptycznej.
\emph{Unifikator w punkcie $P$}
to dowolna funkcja wymierna $u$
spełniająca warunki podane w lemacie \ref{uniformizer_existence_lemma}.
\end{definition}

\begin{definition}
Dana jest funkcja wymierna $r$ nad krzywą eliptyczną,
punkt $P$ na tej krzywej oraz unifikator $u$ w tym punkcie.
\emph{Rząd funkcji $r$ w punkcie $P$},
oznaczany $\ord_P(r)$,
to taka liczba całkowita $d$,
że $r = u^ds$,
gdzie $s$ jest pewną funkcją wymierną taką, że $s(P) \neq 0$.
Dodatkowo:
\begin{itemize}
\item jeżeli $d = 0$,
to funkcja $r$ nie ma w punkcie $P$ ani miejsca zerowego, ani bieguna;
\item jeżeli $d > 0$, to mówimy,
że \emph{funkcja $r$ ma w punkcie $P$ $d$-krotne miejsce zerowe};
\item jeżeli $d < 0$, to mówimy,
że \emph{funkcja $r$ ma w punkcie $P$ $\abs{d}$-krotny biegun}.
\end{itemize}
\end{definition}
