\section{Dywizory}

Rozmieszczenie miejsc zerowych i biegunów
funkcji wymiernych nad krzywą eliptyczną
wykazuje wiele regularności.
Narzędziem, które pozwoli nam je badać, są dywizory.

Na potrzeby tej sekcji pracy przyjmujemy,
że krzywe eliptyczne są rozpatrywane nad ciałem algebraicznie domkniętym.

\subsection*{Definicja}

Aby móc zdefiniować dywizor,
przypomnijmy, czym jest abelowa grupa wolna.

\begin{definition}

Dany jest zbiór $S$.
\emph{Suma formalna nad zbiorem $S$} to wyrażenie w postaci
\begin{equation*}
\sum_{k=1}^n a_k\divi{s_k}
\end{equation*}
gdzie $a_k \in \Z$ oraz $s_k \in S$ dla $k = 1, 2, \ldots, n$.

\emph{Abelowa grupa wolna generowana przez zbiór $S$}
to zbiór złożony ze wszystich sum formalnych nad zbiorem $S$
wyposażony w działanie polegające na formalnym dodaniu dwóch sum
i pogrupowaniu składników ze względu na elementy zbioru $S$.
\end{definition}

\begin{definition}
Dana jest krzywa eliptyczna $E$.
\emph{Dywizor na krzywej eliptycznej $E$}
to element abelowej grupy wolnej generowanej przez punkty krzywej $E$.
Grupę wszystkich dywizorów na krzywej $E$ oznaczamy $\Div(E)$.
\end{definition}

\begin{remark}
Dywizory na krzywej eliptycznej $E$ będziemy zapisywać w postaci
\begin{equation*}
\sum_{P \in E} a(P)\divi{P}
\end{equation*}
mimo, że postać ta nie sugeruje,
że dywizor jest kombinacją skończonej liczby składników.
\end{remark}

\subsection*{Dywizory funkcji wymiernych}

Przydatność dywizorów polega na tym, że można za ich pomocą
zakodować informacje
o wszystkich miejscach zerowych i biegunach funkcji wymiernej
oraz ich krotnościach.

\begin{definition}
Dana jest funkcja wymierna $r$ nad krzywą eliptyczną $E$.
\emph{Dywizor funkcji $r$},
oznaczany $\rdiv(r)$,
to dywizor określony następującym wzorem:
\begin{equation}\label{function_divisor_equation}
\rdiv(r) = \sum_{P \in E} \ord_P(r)\divi{P}
\end{equation}
\end{definition}

Sumowanie we wzorze \ref{function_divisor_equation}
odbywa się po potencjalnie nieskończonej ilości składników.
Okazuje się jednak,
że funkcja wymierna ma niezerowy rząd
tylko w skończonej ilości punktów.

\begin{lemma}\label{function_order_sum_zero_lemma}
Dana jest funkcja wymierna $r$ nad krzywą eliptyczną $E$.
Wówczas
\begin{equation*}
\sum_{P \in E} \ord_P(r) = 0
\end{equation*}
\end{lemma}

\begin{corollary}
Dana jest funkcja wymierna $r$ nad krzywą eliptyczną $E$.
Wówczas
\begin{equation*}
\ord_\ecident(r) = -\sum_{P \in E \setminus \ecident} \ord_P(r)
\end{equation*}
\end{corollary}

\begin{corollary}
Dana jest funkcja wymierna $r$ nad krzywą eliptyczną $E$.
Zbiór punktów krzywej $E$,
w których funkcja $r$ ma niezerowy rząd,
jest skończony.
\end{corollary}

Dywizor niesie w sobie informację
o miejscach zerowych i biegunach funkcji wymiernej.
Okazuje się, że w ten sposób wyznacza ją niemalże jednoznacznie.

\begin{theorem}
Dane są dwie funkcje wymierne nad krzywą eliptyczną $r$ oraz $s$ takie,
że $\rdiv(r) = \rdiv(s)$.
Wówczas iloraz $\frac{r}{s}$ jest stały.
\end{theorem}

\begin{corollary}
Funkcja wymierna nad krzywą eliptyczną nie ma miejsca zerowego ani bieguna,
wtedy i tylko wtedy, gdy jest stała.
\end{corollary}

\subsection*{Własności dywizorów}

Dla dywizorów określamy stopień oraz normę.

\begin{definition}
Dany jest dywizor $\Delta$ na krzywej eliptycznej $E$.
\emph{Stopień dywizora $\Delta$},
oznaczany $\deg(\Delta)$,
to suma współczynników
stojących przy punktach krzywej wchodzących w jego skład,
to znaczy, gdy dywizor $\Delta$ dany jest wzorem
\begin{equation*}
\Delta = \sum_{P \in E} a(P)\divi{P}
\end{equation*}
to jego stopień dany jest wzorem
\begin{equation*}
\deg(\Delta) = \sum_{P \in E} a(P)
\end{equation*}
\end{definition}

\begin{definition}
Dany jest dywizor $\Delta$ na krzywej eliptycznej $E$.
\emph{Norma dywizora $\Delta$},
oznaczana $\abs{\Delta}$,
to suma wartości bezwzględnych współczynników
stojących przy wszytkich punktach krzywej wchodzących w jego skład
z wyjątkiem punktu w nieskończoności,
to znaczy, gdy dywizor $\Delta$ dany jest wzorem
\begin{equation*}
\Delta = \sum_{P \in E} a(P)\divi{P}
\end{equation*}
to jego norma dana jest wzorem
\begin{equation*}
\abs{\Delta} = \sum_{P \in E \setminus \ecident} \abs{a(P)}
\end{equation*}
\end{definition}

Zwróćmy uwagę, że w związku z lematem \ref{function_order_sum_zero_lemma}
dywizor funkcji wymiernej ma stopień równy zero.

\begin{definition}
Dywizor na krzywej eliptycznej jest \emph{główny},
jeśli jest on dywizorem pewnej funkcji wymiernej nad tą krzywą.
\end{definition}

\begin{fact}
Każdy dywizor główny ma stopień równy zero.
\end{fact}

Dywizory główne spełniają następującą ważną zależność.

\begin{theorem}
Dane są funkcje wymierne $r$ oraz $s$ nad krzywą eliptyczną.
Wówczas $\rdiv(rs) = \rdiv(r) + \rdiv(s)$.
\end{theorem}

\begin{corollary}
Dywizory główne na krzywej eliptycznej $E$ tworzą grupę,
oznaczaną $\Prin(E)$.
Jest to podgrupa grupy $\Div(E)$.
\end{corollary}

Dywizory główne pozwalają zdefiniować ważną relację równoważności.

\begin{definition}
Dywizory $\Delta_1$ oraz $\Delta_2$ są \emph{równoważne},
co zapisujemy $\Delta_1 \sim \Delta_2$,
jeśli ich różnica $\Delta_1 - \Delta_2$ jest dywizorem głównym.
\end{definition}

\begin{lemma}\label{divisor_linear_reduction_lemma}
Dla każdego dywizora $\Delta$ istnieje dywizor $\tilde{\Delta}$ taki,
że $\Delta \sim \tilde{\Delta}$, $\deg(\Delta) = \deg(\tilde{\Delta})$
oraz $\abs{\tilde{\Delta}} \leq 1$.
\end{lemma}

\subsection*{Grupy dywizorów}

Pojęcie równoważności dywizorów prowadzi do definicji kilku grup,
które obrazują strukturę
możliwych miejsc zerowych i biegunów funkcji wymiernych nad krzywą eliptyczną.

\begin{definition}
Dana jest krzywa eliptyczna $E$.
\emph{Grupa Picarda krzywej $E$},
oznaczana $\Pic(E)$,
to grupa ilorazowa $\Div(E)/\Prin(E)$.
\end{definition}

\begin{definition}
Dana jest krzywa eliptyczna $E$.
\emph{Grupa zerowych dywizorów krzywej $E$},
oznaczana $\Div^0(E)$,
to podgrupa grupy $\Div(E)$ składająca się
z dywizorów o zerowym stopniu.
\end{definition}

\begin{definition}
Dana jest krzywa eliptyczna $E$.
\emph{Zerowa grupa Picarda krzywej $E$},
oznaczana $\Pic^0(E)$,
to grupa ilorazowa $\Div^0(E)/\Prin(E)$.
\end{definition}

Z grupą zerowych dywizorów oraz zerową grupą Picarda
powiązane są następujące bardzo istotne wyniki.

\begin{theorem}
Dana jest krzywa eliptyczna $E$ oraz dywizor $\Delta \in \Div^0(E)$.
Wówczas istnieje taki punkt krzywej $P \in E$,
że $\Delta \sim \divi{P} - \divi{\ecident}$.
\end{theorem}

\begin{corollary}
Dana jest krzywa eliptyczna $E$.
Istnieje wzajemna jednoznaczność między punktami krzywej $E$,
a jej zerową grupą Picarda $\Pic^0(E)$.
\end{corollary}
