\section{Dywizory}

Rozmieszczenie miejsc zerowych i biegunów
funkcji wymiernych na krzywej eliptycznej
wykazuje wiele regularności.
Narzędziem, które pozwoli nam je badać, są dywizory.

\subsection*{Definicja}

Aby móc zdefiniować dywizor,
przypomnijmy, czym jest abelowa grupa wolna.

\begin{definition}
Dany jest zbiór $S$.
\emph{Suma formalna nad zbiorem $S$} to wyrażenie w postaci
\begin{equation*}
\sum_{k=1}^n a_k\divi{s_k}
\end{equation*}
gdzie $a_k \in \Z$ oraz $s_k \in S$ dla $k = 1, 2, \ldots, n$.

\emph{Abelowa grupa wolna generowana przez zbiór $S$}
to zbiór złożony ze wszystich sum formalnych nad zbiorem $S$
wyposażony w działanie polegające na formalnym dodaniu dwóch sum
i pogrupowaniu składników ze względu na elementy zbioru $S$.
\end{definition}

\begin{remark}
Sumę formalną nad zbiorem $S$ będziemy często zapisywać w postaci
\begin{equation*}
\sum_{s \in S}a(s)\divi{s}
\end{equation*}
gdzie funkcja $a \colon S \to \Z$ przyjmuje wartość niezerową
tylko skończoną ilość razy.
\end{remark}

\begin{definition}
Dana jest krzywa eliptyczna $E$.
\emph{Dywizor na krzywej eliptycznej $E$}
to element abelowej grupy wolnej generowanej przez punkty krzywej $E$.
Grupę wszystkich dywizorów na krzywej $E$ oznaczamy $\Div(E)$.
\end{definition}

\begin{remark}
Zgodnie z poczynioną przed chwilą uwagą,
dywizory na krzywej eliptycznej $E$ będziemy zapisywać w postaci
\begin{equation*}
\sum_{P \in E} a(P)\divi{P}
\end{equation*}
mimo, że postać ta nie sugeruje,
że dywizor jest kombinacją skończonej liczby składników.
\end{remark}

\subsection*{Własności dywizorów}

Dla dywizorów określamy stopień oraz normę.

\begin{definition}
Dany jest dywizor $\Delta$ na krzywej eliptycznej $E$.
\emph{Stopień dywizora $\Delta$},
oznaczany $\deg(\Delta)$,
to suma współczynników
stojących przy punktach krzywej wchodzących w jego skład,
to znaczy, gdy dywizor $\Delta$ dany jest wzorem
\begin{equation*}
\Delta = \sum_{P \in E} a(P)\divi{P}
\end{equation*}
to jego stopień dany jest wzorem
\begin{equation*}
\deg(\Delta) = \sum_{P \in E} a(P)
\end{equation*}
\end{definition}

\begin{definition}
Dany jest dywizor $\Delta$ na krzywej eliptycznej $E$.
\emph{Norma dywizora $\Delta$},
oznaczana $\abs{\Delta}$,
to suma wartości bezwzględnych współczynników
stojących przy wszytkich punktach krzywej wchodzących w jego skład
z wyjątkiem punktu w nieskończoności,
to znaczy, gdy dywizor $\Delta$ dany jest wzorem
\begin{equation*}
\Delta = \sum_{P \in E} a(P)\divi{P}
\end{equation*}
to jego norma dana jest wzorem
\begin{equation*}
\abs{\Delta} = \sum_{P \in E \setminus \{\ecident\}} \abs{a(P)}
\end{equation*}
\end{definition}

\subsection*{Dywizory funkcji wymiernych}

Przydatność dywizorów polega na tym, że można za ich pomocą
reprezentować informacje
o wszystkich miejscach zerowych i biegunach funkcji wymiernej
oraz ich krotnościach.

\begin{definition}
Dana jest funkcja wymierna $r$ na krzywej eliptycznej $E$.
\emph{Dywizor funkcji $r$},
oznaczany $\rdiv(r)$,
to dywizor określony następującym wzorem:
\begin{equation*}
\rdiv(r) = \sum_{P \in E} \ord_P(r)\divi{P}
\end{equation*}
\end{definition}

\begin{definition}
Dywizor na krzywej eliptycznej jest \emph{główny},
jeśli jest on dywizorem pewnej funkcji wymiernej na tej krzywej.
\end{definition}

Następująca ważna cecha dywizorów głównych
jest konsekwencją wniosku \ref{function_order_sum_zero_corollary}.

\begin{fact}
Dywizor główny ma stopień równy zero.
\end{fact}

Dywizor główny niemal jednoznacznie wyznacza funkcję wymierną.

\begin{fact}
Dwie funkcje wymierne różniące się o czynnik stały mają taki sam dywizor.
\end{fact}

\begin{theorem}
Dane są dwie funkcje wymierne na krzywej eliptycznej $r$ oraz $s$ takie,
że $\rdiv(r) = \rdiv(s)$.
Wówczas iloraz $\frac{r}{s}$ jest stały.
\end{theorem}

\begin{corollary}
Funkcja wymierna na krzywej eliptycznej nie ma miejsca zerowego ani bieguna,
wtedy i tylko wtedy, gdy jest stała i nie jest zerowa.
\end{corollary}

\begin{corollary}
Dywizor wyznacza funkcję wymierną z dokładnością do stałego czynnika.
\end{corollary}

Dywizory główne pozwalają zdefiniować ważną relację równoważności.

\begin{definition}
Dywizory $\Delta_1$ oraz $\Delta_2$ są \emph{równoważne},
co zapisujemy $\Delta_1 \sim \Delta_2$,
jeśli ich różnica $\Delta_1 - \Delta_2$ jest dywizorem głównym.
\end{definition}

\subsection*{Szczególne grupy dywizorów}

Dywizory główne spełniają zależność
podobną do tej, którą spełniają stopnie funkcji wymiernych.

\begin{theorem}\label{fun_mul_divi_add_theorem}
Dane są funkcje wymierne $r$ oraz $s$ na krzywej eliptycznej.
Wówczas $\rdiv(rs) = \rdiv(r) + \rdiv(s)$.
\end{theorem}

\begin{corollary}
Zbiór dywizorów głównych na krzywej eliptycznej tworzy grupę,
która jest podgrupą grupy $\Div(E)$.
\end{corollary}

\begin{definition}
\emph{Grupę dywizorów głównych} na krzywej eliptycznej $E$
oznaczamy $\Prin(E)$.
\end{definition}

Pojęcie równoważności dywizorów prowadzi do definicji kolejnych grup.

\begin{definition}
Dana jest krzywa eliptyczna $E$.
\emph{Grupa Picarda krzywej $E$},
oznaczana $\Pic(E)$,
to grupa ilorazowa $\Div(E)/\Prin(E)$.
\end{definition}

Następujące twierdzenie obrazuje strukturę grupy Picarda.

\begin{theorem}\label{divisor_linear_reduction_theorem}
Dana jest krzywa eliptyczna $E$.
Dla każdego dywizora $\Delta \in \Div(E)$
istnieje dywizor $\tilde{\Delta} \in \Div(E)$ taki,
że $\Delta \sim \tilde{\Delta}$, $\deg(\Delta) = \deg(\tilde{\Delta})$
oraz $\tilde{\Delta} = \divi{P} + n\divi{\ecident}$.
\end{theorem}

Grupa Picarda daje pewną intuicję,
jak wyglądają dywizory główne na krzywej eliptycznej,
ponieważ obrazuje zależność między grupą dywizorów głównych $\Prin(E)$,
a grupą $\Div(E)$, która jest abelową grupą wolną,
ma zatem dosyć prostą strukturę.
Warto porównać grupę $\Prin(E)$ z inną grupą,
która jest mniejsza od grupy $\Div(E)$,
lecz również ma prostą strukturę.

\begin{definition}
Dana jest krzywa eliptyczna $E$.
\emph{Grupa dywizorów stopnia zero na krzywej $E$},
oznaczana $\Div^0(E)$,
to podgrupa grupy $\Div(E)$ składająca się
z dywizorów o zerowym stopniu.
\end{definition}

\begin{definition}
Dana jest krzywa eliptyczna $E$.
\emph{Zerowa grupa Picarda krzywej $E$},
oznaczana $\Pic^0(E)$,
to grupa ilorazowa $\Div^0(E)/\Prin(E)$.
\end{definition}

Strukturę zerowej grupy Picarda obrazuje następujące twierdzenie,
bardzo podobne do twierdzenia \ref{divisor_linear_reduction_theorem}.

\begin{theorem}\label{zerodeg_divisor_linear_reduction_theorem}
Dana jest krzywa eliptyczna $E$.
Dla każdego dywizora $\Delta \in \Div^0(E)$
istnieje dokładnie jeden punkt krzywej $P \in E$ taki,
że $\Delta \sim \divi{P} - \divi{\ecident}$.
\end{theorem}

\begin{corollary}\label{piczero_curvepts_bijection_corollary}
Istnieje wzajemna jednoznaczność między punktami krzywej eliptycznej $E$,
a elementami jej zerowej grupy Picarda $\Pic^0(E)$.
\end{corollary}
