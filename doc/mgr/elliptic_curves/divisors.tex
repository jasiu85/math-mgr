\section{Dywizory}

Rozmieszczenie miejsc zerowych i biegunów
funkcji wymiernych na krzywej eliptycznej
wykazuje wiele regularności.
Narzędziem, które pozwoli nam je badać, są dywizory.

\subsection*{Definicja}

Aby móc zdefiniować dywizor,
przypomnijmy, czym jest abelowa grupa wolna.

\begin{definition}
Dany jest zbiór $S$.
\emph{Suma formalna nad zbiorem $S$} to wyrażenie w postaci
\begin{equation*}
\sum_{k=1}^n a_k\divi{s_k}
\end{equation*}
gdzie $a_k \in \Z$ oraz $s_k \in S$ dla $k = 1, 2, \ldots, n$.

\emph{Abelowa grupa wolna generowana przez zbiór $S$}
to zbiór złożony ze wszystich sum formalnych nad zbiorem $S$
wyposażony w działanie polegające na formalnym dodaniu dwóch sum
i pogrupowaniu składników ze względu na elementy zbioru $S$.
\end{definition}

\begin{remark}
Sumę formalną nad zbiorem $S$ będziemy często zapisywać w postaci
\begin{equation*}
\sum_{s \in S}a(s)\divi{s}
\end{equation*}
gdzie funkcja $a \colon S \to \Z$ przyjmuje wartość niezerową
tylko skończoną ilość razy.
\end{remark}

\begin{definition}
Dana jest krzywa eliptyczna $E$.
\emph{Dywizor na krzywej eliptycznej $E$}
to element abelowej grupy wolnej generowanej przez punkty krzywej $E$.
\end{definition}

\begin{remark}
Zgodnie z poczynioną przed chwilą uwagą,
dywizory na krzywej eliptycznej $E$ będziemy zapisywać w postaci
\begin{equation*}
\sum_{P \in E} a(P)\divi{P}
\end{equation*}
mimo, że postać ta nie sugeruje,
że dywizor jest kombinacją skończonej liczby składników.
\end{remark}

\subsection*{Własności dywizorów}

Dla dywizorów określamy stopień oraz normę.

\begin{definition}
Dany jest dywizor $\Delta$ na krzywej eliptycznej $E$.
\emph{Stopień dywizora $\Delta$},
oznaczany $\deg(\Delta)$,
to suma współczynników
stojących przy punktach krzywej wchodzących w jego skład,
to znaczy, gdy dywizor $\Delta$ dany jest wzorem
\begin{equation*}
\Delta = \sum_{P \in E} a(P)\divi{P}
\end{equation*}
to jego stopień dany jest wzorem
\begin{equation*}
\deg(\Delta) = \sum_{P \in E} a(P)
\end{equation*}
\end{definition}

\begin{definition}
Dany jest dywizor $\Delta$ na krzywej eliptycznej $E$.
\emph{Norma dywizora $\Delta$},
oznaczana $\abs{\Delta}$,
to suma wartości bezwzględnych współczynników
stojących przy wszytkich punktach krzywej wchodzących w jego skład
z wyjątkiem punktu w nieskończoności,
to znaczy, gdy dywizor $\Delta$ dany jest wzorem
\begin{equation*}
\Delta = \sum_{P \in E} a(P)\divi{P}
\end{equation*}
to jego norma dana jest wzorem
\begin{equation*}
\abs{\Delta} = \sum_{P \in E \setminus \{\ecident\}} \abs{a(P)}
\end{equation*}
\end{definition}

\subsection*{Dywizory i funkcje wymierne}

Przydatność dywizorów polega na tym, że można za ich pomocą
reprezentować informacje
o wszystkich miejscach zerowych i biegunach funkcji wymiernej
oraz ich krotnościach.

\begin{definition}
Dana jest funkcja wymierna $r$ na krzywej eliptycznej $E$.
\emph{Dywizor funkcji $r$},
oznaczany $\rdiv(r)$,
to dywizor określony następującym wzorem:
\begin{equation*}
\rdiv(r) = \sum_{P \in E} \ord_P(r)\divi{P}
\end{equation*}
\end{definition}

Okazuje się, że dywizor niosący informacje o miejscach zerowych i biegunach
funkcji wymiernej wyznacza ją niemalże jednoznacznie.

\begin{fact}
Dwie funkcje wymierne różniące się o czynnik stały mają taki sam dywizor.
\end{fact}

\begin{theorem}
Dane są dwie funkcje wymierne na krzywej eliptycznej $r$ oraz $s$ takie,
że $\rdiv(r) = \rdiv(s)$.
Wówczas iloraz $\frac{r}{s}$ jest stały.
\end{theorem}

\begin{corollary}
Funkcja wymierna na krzywej eliptycznej nie ma miejsca zerowego ani bieguna,
wtedy i tylko wtedy, gdy jest stała i nie jest zerowa.
\end{corollary}

\begin{corollary}
Dywizor wyznacza funkcję wymierną z dokładnością do stałego czynnika.
\end{corollary}
