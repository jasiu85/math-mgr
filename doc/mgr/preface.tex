\chapter*{Wstęp}
\addcontentsline{toc}{chapter}{Wstęp}

\noindent
Matematyka uważana jest za naukę oderwaną od rzeczywistości
i mającą niewielki wpływ na życie ludzi.
Jest jednak zupełnie odwrotnie:
codziennie korzystamy z różnych zdobyczy cywilizacyjnych
takich jak komputery i telefony komórkowe;
przemieszczamy się za pomocą samochodów, statków i samolotów;
leczymy się za pomocą nowoczesnych leków;
mieszkamy w zaawansowanych technicznie budynkach.
Wszystko to jest wynikiem postępu różnych nauk,
a te, od nauk ścisłych
aż do ,,niematematycznych'' nauk społecznych i humanistycznych,
potrzebują narzędzi matematycznych.
Jesteśmy zatem zależni od matematyki w każdej dziedzinie życia,
nawet jeśli nieświadomie.
Z tego właśnie powodu matematyka nazywana jest królową nauk.

\noindent
Różne nauki w różnym stopniu korzystają ze zdobyczy
poszczególnych działów matematyki.
I tak na przykład
mechanika klasyczna opiera się na analizie,
głównie na rachunku różniczkowym i całkowym;
równania różniczkowe cząstkowe służą
do opisywania i symulowania wielu zjawisk,
w tym prognozowania pogody czy projektowania powierzchni aerodynamicznych
(kadłubów samolotów, karoserii samochodów itp.);
rachunek prawdopodobieństwa i statystyka
znajdują zastosowanie w ekonomii i przy różnego rodzaju grach losowych,
są także nieodzowne
podczas przeprowadzania eksperymentów psychologicznych i socjologicznych;
współczesne problemy fizyki
związane z ogólną teorią względności oraz mechaniką kwantową
formułowane są w języku zaawansowanej algebry.

\noindent
Jednym z działów matematyki jest teoria liczb.
Przez długi czas była ona uważana za gałąź matematyki,
która, chociaż ma niemały wpływ na matematykę samą w sobie,
to znajduje niewiele zastosowań poza nią.
Znaczenie teorii liczb w życiu codziennym
niesamowicie wzrosło wraz z pojawieniem się komputerów.

\noindent
W obecnych czasach komputery wspomagają nas w wykonywaniu wielu czynności,
w tym pozwalają nam komunikować się drogą elektroniczną.
To spowodowało, że należało rozwiązać komputerowe odpowiedniki
klasycznych problemów związanych z komunikacją:
potwierdzeniem tożsamości nadawcy komunikatu
oraz zapewnieniem, że komunikat zostanie odczytany jedynie przez odbiorcę.
Problem zaprojektowania systemu komputerowego,
który miałby te dwie cechy,
dał początek współczesnej kryptografii.

\noindent
Komputerowe systemy kryptograficzne bazują na fakcie,
że pewne problemy są trudne obliczeniowo.
Egzemplarz takiego problemu spreparowany w taki sposób,
że znane jest jego rozwiązanie,
stanowi ,,sekret'' pozwalający dwóm stronom na bezpieczną komunikację.
Jest to komputerowy odpowiednik kłódki i klucza,
którym dysponują tylko dwie osoby.
Tak się przy tym składa,
że problemy, które dają podstawę systemom kryptograficznym,
bardzo często związane są z teorią liczb.
Przykładowo, problem rozkładu na czynniki pierwsze leży u podstaw systemu RSA,
a na problemie logarytmu dyskretnego opiera się protokół Diffiego-Hellmana.

\noindent
Teoria liczb i kryptografia stanowią zatem furtkę,
poprzez którą pozornie abstrakcyjne obiekty matematyczne
mogą trafić do codziennego (nawet jeśli nieświadomego) użytku.
W ten sposób zastosowanie praktyczne znalalazły
np. ciała skończone czy krzywe eliptyczne,
a także wiele pojęć z nimi stowarzyszonych,
w tym pojęcie kluczowe w niniejszej pracy: iloczyn Weila.

\noindent
Iloczyn Weila to pewna dwuargumentowa operacja działająca
na punktach zadanego, skończonego rzędu na krzywej eliptycznej
i prowadząca w zbiór pierwiastków z jedności.
Dwie kluczowe cechy iloczynu Weila to niezdegenerowanie
oraz dwuliniowość.
W kontekście iloczynu Weila niezdegenerowanie oznacza,
że jego wartościami są również pierwiastki pierwotne.
Dwuliniowość zaś to odpowiednik pojęcia znanego z algebry liniowej
przeniesiony na grupy abelowe.
Z punktu widzenia krpytografii
szczególnie dwuliniowość jest istotna, ponieważ powoduje,
że pewne problemy decyzyjne na krzywych eliptycznych stają się łatwe.

\noindent
Celem niniejszej pracy jest przedstawienie na przykładzie iloczynu Weila
procesu wdrażania za pomocą teorii liczb i kryptografii
pojęcia matematycznego do codziennego użytku
oraz pokazanie, że proces ten może być względnie nieskomplikowany.
Wybór iloczynu Weila na pojęcie centralne w całej pracy
podyktowany jest dwoma czynnikami.
Po pierwsze, na podstawie iloczynu Weila udało się skonstruować
wiele ciekawych kryptosystemów,
które wykazują niespotykane wcześniej cechy.
Po drugie, za pomocą iloczynu Weila można przeprowadzać
ataki na istniejące kryptosystemy oparte na krzywych eliptycznych.
Widać więc, że jest iloczyn Weila pojęciem niezwykle interesującym
z kryptograficznego punktu widzenia.

\noindent
Decydującym czynnikiem,
który umożliwia zastosowanie iloczynu Weila w praktyce,
jest algorytm Millera, który pozwala obliczać jego wartości w wydajny sposób.
Częścią niniejszej pracy jest autorska implementacja tego algorytmu.
Przejrzystość powstałego kodu źródłowego
oraz to, że powstał on szybko i bez większych trudności,
jest dodatkowym potwierdzeniem tezy,
że zaawansowane pojęcia matematyczne mogą łatwo trafić ,,pod strzechy''.

\noindent
Praca składa się z pięciu rozdziałów.
W rozdziale pierwszym znajduje się krótkie wprowadzenie
do tematyki krzywych eliptycznych.
Rozdział drugi zawiera bardziej szczegółowe studium
wyjątkowo ważnego pojęcia związanego z krzywymi eliptycznymi:
struktury grupy abelowej na krzywej eliptycznej.
Celem rozdziału trzeciego
jest zdefiniowanie i zanalizowanie własności iloczynu Weila.
Rozdział czwarty zawiera omówienie algorytmu Millera
oraz jego autorskiej implementacji.
Wreszcie, rozdział piąty opisuje zastosowania iloczynu Weila w kryptografii.
