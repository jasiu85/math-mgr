\chapter*{Wstęp}
\addcontentsline{toc}{chapter}{Wstęp}

Matematyka uważana jest za naukę oderwaną od rzeczywistości,
mającą niewielki wpływ na życie ludzi.
Jest jednak zupełnie odwrotnie:
codziennie korzystamy z różnych zdobyczy cywilizacyjnych
takich jak komputery, telefony komórkowe, nowoczesne leki;
przemieszczamy się za pomocą samochodów, statków i samolotów;
leczymy się za pomocą nowoczesnych leków;
mieszkamy w zaawansowanych technicznie budynkach.
Wszystko to jest wynikiem postępu różnych nauk,
a te, od fizyki i chemii
aż do ,,niematematycznych'' nauk społecznych i humanistycznych,
korzystają z narzędzi matematycznych.
Jesteśmy zatem zależni od matematyki w każdej dziedzinie życia,
nawet jeśli nieświadomie.
Z tego właśnie powodu matematyka nazywana jest królową nauk.

Różne nauki w różnym stopniu korzystają ze zdobyczy
poszczególnych działów matematyki.
I tak na przykład
mechanika klasyczna opiera się na analizie,
głównie na rachunku różniczkowym i całkowym;
równania różniczkowe cząstkowe służą dziś
do opisywania i symulowania wielu zjawisk,
w tym prognozowania pogody, projektowania powierzchni aerodynamicznych
(kadłubów samolotów, karoserii samochodów itp.);
rachunek prawdopodobieństwa i statystyka
znajdują zastosowanie w ekonomii i przy różnego rodzaju grach losowych,
są także nieodzowne
podczas przeprowadzania eksperymentów psychologicznych i socjologicznych;
współczesne problemy fizyki
związane z ogólną teorią względności oraz mechaniką kwantową
formułowane są w języku zaawansowanej algebry.

Jednym z działów matematyki jest teoria liczb.
Przez długi czas była ona uważana za gałąź matematyki,
która, chociaż ma niemały wpływ na matematykę samą w sobie,
to znajduje niewiele zastosowań poza nią.
Znaczenie teorii liczb w życiu codziennym
niesamowicie wzrosło wraz z pojawieniem się komputerów.

W obecnych czasach komputery wspomagają nas w wykonywaniu wielu czynności,
w tym pozwalają nam komunikować się drogą elektroniczną.
To spowodowało, że należało rozwiązać komputerowe odpowiedniki
klasycznych problemów związanych z komunikacją:
potwierdzeniem tożsamości nadawcy komunikatu
oraz zapewnieniem, że komunikat zostanie odczytany jedynie przez odbiorcę.
Problem zaprojektowania systemu komputerowego,
który miałby te dwie cechy,
dał początek współczesnej kryptografii.

Komputerowe systemy kryptograficzne bazują na fakcie,
że pewne problemy są trudne obliczeniowo.
Egzemplarz takiego problemu spreparowany w taki sposób,
że znane jest jego rozwiązanie,
stanowi ,,sekret'' pozwalający dwóm stronom na bezpieczną komunikację.
Jest to komputerowy odpowiednik kłódki i klucza,
którym dysponują tylko dwie osoby.
Tak się przy tym składa,
że problemy, które dają podstawę systemom kryptograficznym,
bardzo często związane są z teorią liczb,
skąd wynika wzrost jej znaczenia.
Przykładowo, problem rozkładu na czynniki pierwsze leży u podstaw systemu RSA,
a na problemie logarytmu dyskretnego opiera się protokół Diffiego-Hellmana.

Celem niniejszej pracy jest przedstawienie procesu,
na skutek którego pojęcie matematyczne
może poprzez teorię liczb i kryptografię oraz techniki komputerowe
trafić do codziennego (nawet jeśli nieświadomego) użytku.
Proces ten zostanie zobrazowany na przykładzie iloczynu Weila --
pewnej dwuargumentowej operacji
działającej na punktach krzywej eliptycznej.

Praca składa się z pięciu rozdziałów.
W rozdziale pierwszym znajduje się krótkie wprowadzenie
do tematyki krzywych eliptycznych.
Rozdział drugi zawiera bardziej szczegółowe studium
wyjątkowo ważnego pojęcia związanego z krzywymi eliptycznymi:
struktury grupy abelowej na krzywej eliptycznej.
Celem rozdziału trzeciego
jest zdefiniowanie i zanalizowanie własności iloczynu Weila.
Rozdział czwarty opisuje algorytm Millera,
który pozwala efektywnie obliczać wartości iloczynu Weila.
Wreszcie, rozdział piąty opisuje zastosowania iloczynu Weila w kryptografii.
