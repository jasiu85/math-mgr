% define values of constants/counters/spaces etc.
\setlipsumdefault{1-2}

% define AMS theorems
\numberwithin{equation}{section}

\theoremstyle{plain}
\newtheorem{fact}[equation]{Fakt}
\newtheorem{lemma}[equation]{Lemat}
\newtheorem{theorem}[equation]{Twierdzenie}
\newtheorem{corollary}[equation]{Wniosek}

\theoremstyle{definition}
\newtheorem{definition}[equation]{Definicja}
\newtheorem{example}[equation]{Przykład}

\theoremstyle{remark}
\newtheorem{remark}[equation]{Uwaga}

% define AMS operators

\DeclareMathOperator{\Div}{Div}
\DeclareMathOperator{\Prin}{Prin}
\DeclareMathOperator{\Pic}{Pic}
\DeclareMathOperator{\ord}{ord}
\DeclareMathOperator{\rdiv}{div}
\DeclareMathOperator{\fieldchar}{char}

% define new commands
\newcommand{\todo}[1]{TODO: #1}
\newcommand{\todolipsum}[1]{TODO: #1 \lipsum}

\newcommand{\ecident}{\mathcal{O}}
\newcommand{\F}{\mathbb{F}}
\newcommand{\G}{\mathbb{G}}
\newcommand{\GF}{\mathbb{GF}}
\newcommand{\K}{\mathbb{K}}
\newcommand{\fieldL}{\mathbb{L}}
\newcommand{\N}{\mathbb{N}}
\newcommand{\Z}{\mathbb{Z}}
\newcommand{\Q}{\mathbb{Q}}
\newcommand{\R}{\mathbb{R}}
\newcommand{\C}{\mathbb{C}}
\newcommand{\abs}[1]{\lvert#1\rvert}
\newcommand{\norm}[1]{\lVert#1\rVert}
\newcommand{\divi}[1]{\left\langle#1\right\rangle}

\newcommand{\fundeftwo}[4]
{\left\{
\begin{array}{ll}
#1 & \textrm{gdy }#2\textrm{,} \\
#3 & \textrm{gdy }#4\textrm{.}
\end{array}
\right. }

\newcommand{\fundefthree}[6]
{\left\{
\begin{array}{ll}
#1 & \textrm{gdy }#2\textrm{,} \\
#3 & \textrm{gdy }#4\textrm{,} \\
#5 & \textrm{gdy }#6\textrm{.}
\end{array}
\right. }
