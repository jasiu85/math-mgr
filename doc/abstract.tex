\begin{abstract}

\noindent
W niniejszej pracy opisano iloczyn Weila,
algorytm obliczający jego wartości
oraz jego zastosowania w kryptografii.
Praca zawiera również
krótkie wprowadzenie do krzywych eliptycznych.

\noindent
Iloczyn Weila to funkcja $w\colon E[n] \times E[n] \to \K$,
która prowadzi z podgrupy $n$-torsyjnej $E[n]$
na krzywej eliptycznej $E$ nad ciałem $\K$
(czyli podgrupy składającej się z tych punktów $P$ na krzywej $E$,
które spełniają zależność $nP = \ecident$)
w grupę pierwiastków $n$-tego stopnia z jedności w ciele $\K$.
Cechy charakterystyczne iloczynu Weila
to dwuliniowość, antysymetria oraz niezdegenerowanie.

\noindent
Przedstawiony w pracy algorytm obliczający wartości iloczynu Weila
został zaproponowany przez Millera.
Algorytm ten realizuje schemat ,,podwajaj-i-dodawaj''
podobny do tego występującego
w algorytmie szybkiego podnoszenia do $n$-tej potęgi.
Częścią pracy jest autorska implementacja tego algorytmu.

\noindent
Zastosowania iloczynu Weila zaprezentowane w pracy obejmują
redukcję MOV (Me\-ne\-zes-Oka\-mo\-to-Van\-stone)
umożliwiającą przeprowadzanie ataków
na systemy kryptograficzne oparte na krzywych eliptycznych
oraz konstrukcje kryptosystemów opartych na tożsamości,
tzn. pozwalających na takie szyfrowanie i podpisywanie wiadomości,
że jednym z kluczy jest dowolny ciąg bitów, np. adres poczty elektronicznej.

\end{abstract}
