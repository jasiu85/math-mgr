\chapter{Zastosowania iloczynu Weila}

\noindent
Iloczyn Weila jest narzędziem używanym w kryptografii wtedy,
gdy rozpatrywane są zagadnienia związane z krzywymi eliptycznymi.
Jak każde narzędzie, można go wykorzystać na dwa sposoby:
sposób ,,dobry'' polega na tworzeniu kryptosystemów z pomocą iloczynu Weila,
zaś sposób ,,zły'' polega na przeprowadzaniu ataków na nie.

\noindent
W rozdziale tym przedstawimy zastosowania iloczynu Weila obu rodzajów.
Omówimy jeden atak kryptograficzny
oraz grupę kryptosystemów,
które mają pewną specyficzną cechę wspólną.

\input{weil_pairing_usage/supersingular_curves}

\section{Redukcja MOV}

Pierwsze zastosowanie iloczynu Weila
związane jest
z problemem logarytmu dyskretnego oraz protokołem Diffiego-Hellmana.

Podajmy najpierw definicje obu problemów.

\begin{problem}[Logarytm dyskretny]
Dana jest grupa cykliczna $\G$,
jej generator $g \in \G$
oraz jej element $a \in \G$.
Znaleźć liczbę całkowitą $k \in \Z$ taką, że:
\begin{equation}\label{discrete_log_eqn}
g^k = a
\end{equation}
\end{problem}

\begin{remark}
Zwyczajowo problem logarytmu dyskretnego opisuje się,
stosując zapis multiplikatywny działania grupowego.
W przypadku zapisu addytywnego równanie \ref{discrete_log_eqn}
przybiera postać:
\begin{equation}\label{discrete_log_additive_eqn}
kg = a
\end{equation}
\end{remark}

\begin{remark}
Problem logarytmu dyskretnego można rozpatrywać w przypadku dowolnej grupy.
Należy wówczas rozpatrywać jej podgrupę $(g)$
generowaną przez pewien jej element $g$.
\end{remark}

\begin{remark}
Problem logarytmu dyskretnego w grupie $\Z / n\Z$ jest prosty,
ponieważ dysponujemy algorytmem obliczającym rozwiązania
równania \ref{discrete_log_additive_eqn} w grupie $\Z / n\Z$.
\end{remark}

\begin{remark}
Problem logarytmu dyskretnego w grupie multiplikatywnej ciała $\GF(p^e)$
jest obecnie uważany za problem trudny obliczeniowo.
\end{remark}

Jak wiemy, grupa multiplikatywna ciała $\GF(p^e)$
jest izomorficzna z grupą $\Z / (p^e-1)\Z$.
Mimo to problemy logarytmu dyskretnego w tych grupach
różnią się trudnością,
co jest spowodowane różnicą w sposobie,
w jaki reprezentujemy elementy obu grup
oraz obliczamy wyniki działania grupowego.

Problemem spokrewnionym z problemem logarytmu dyskretnego
jest zagadnienie złamania protokołu Difiego-Hellmana,
zwane w skrócie ,,problemem Diffiego-Hellmana''.

\begin{problem}[Diffie-Hellman]
Dana jest grupa cykliczna $\G$,
jej generator $g \in G$
oraz elementy $a = g^k$ i $b = g^l$,
gdzie $k, l \in \Z$.
Znaleźć element $c \in \G$ taki, że:
\begin{equation}
c = g^{kl}
\end{equation}
\end{problem}

\begin{remark}
Podobnie jak w przypadku problemu logarytmu dyskretnego,
możemy stosować zapis addytywny (szukamy wówczas wartości $klg$)
oraz rozpatrywać problem w dowolnej grupie.
\end{remark}

Pokrewieństwo obu problemów polega na tym,
że jeden można zredukować do drugiego.

\begin{theorem}
Problem Diffiego-Hellmana w grupie $\G$ można
w czasie wielomianowym w sposób deterministyczny
zredukować do problemu logarytmu dyskretnego w grupie $\G$.
\end{theorem}

\begin{proof}
Redukcja jest bardzo prosta.
Jeżeli dysponujemy algorytmem rozwiązującym
problem logarytmu dyskretnego w grupie $\G$,
to postępujemy następująco:
\begin{enumerate}
\item na podstawie wartości $\G$, $g$ i $a = g^k$ obliczamy wartość $k$;
\item na podstawie wartości $\G$, $g$ i $b = g^l$ obliczamy wartość $l$;
\item obliczamy $kl$;
\item obliczamy $g^{kl}$.
\end{enumerate}
Jak widać, aby rozwiązać egzemplarz problemu Diffiego-Hellmana
wystarczy dwa razy zastosować rozwiązanie problemu logarytmu dyskretnego.
\end{proof}

Czy problem Diffiego-Hellmana można rozwiązać inaczej?
Być może wystarczy odpowiednio wykorzystać inne operacje,
które daje się efektywnie obliczać.
Zagadnienie to jest o tyle istotne,
że na trudności problemu Diffiego-Hellmana
opiera się wiele kryptosystemów,
których implementacje są wykorzystywane na codzień,
m.in. w protokole SSL używanym w sieci Internet.
Wydaje się, że oba problemy są równoważne (por. \ref{maurer}),
a to oznacza trudność problemu Diffiego-Hellmana,
a zatem bezpieczeństwo używanych protokołów.

Oba podane problemy zostały przedstawione w wersji obliczeniowej.
W teorii złożoności obliczeniowej często rozpatruje się wersje decyzyjne
problemów. Problem Diffiego-Hellmana przybiera wówczas następującą postać.

\begin{problem}
Dana jest grupa $\G$,
jej generator $g \in \G$
oraz elementy $a = g^k$, $b = g^l$ i $c = g^m$
gdzie $k, l, m \in \Z$.
Stwierdzić, czy zachodzi następująca zależność:
\begin{equation}
kl \equiv m \quad (\mathrm{mod}\ \abs{\G})
\end{equation}
\end{problem}

Jest jasne, że problem decyzyjny można zredukować do problemu obliczeniowego,
zatem jest on nietrudniejszy. Może się zdarzyć, że jest dużo prostszy.

W przypadku krzywych eliptycznych obie wersje problemu Diffiego-Hellmana
są następujące.

\begin{problem}
Dana jest krzywa eliptyczna $E(\K)$,
punkt $G \in E$ skończonego rzędu $n$
oraz punkty $P = kG$ i $Q = lG$,
gdzie $k, l \in \Z$.
Znaleźć punkt $R \in E$ taki, że:
\begin{equation}
R = klG
\end{equation}
\end{problem}

\begin{problem}
Dana jest krzywa eliptyczna $E(\K)$,
punkt $G \in E$ skończonego rzędu $n$
oraz punkty $P = kG$, $Q = lG$ i $R = mG$,
gdzie $k, l, m \in \Z$.
Stwierdzić, czy zachodzi następująca zależność:
\begin{equation}
kl \equiv m \quad (\mathrm{mod}\ n)
\end{equation}
\end{problem}

Czy problem Diffiego-Hellmana na krzywej eliptycznej jest trudny?
Okazuje się, że iloczyn Weila ma wpływ na tę kwestię.

\begin{theorem}[Redukcja MOV]
Obliczeniowy problem logarytmu dyskretnego
na krzywej eliptycznej $E = E(\GF(p^e))$
można w czasie wielomianowym w sposób deterministyczny zredukować
do obliczeniowego problemu logarytmu dyskretnego
w grupie multiplikatywnej ciała $\GF(p^{ef})$.
\end{theorem}

\begin{proof}
Niech punkt $G \in E$ rzędu $n$
oraz punkt $P = kG$ będą instancją
obliczeniowego problemu logarytmu dyskretnego na krzywej $E$.
Rozważmy rozszerzenie $\GF(p^{ef})$ ciała $\GF(p^e)$ dostatecznie duże,
aby istniał punkt $H \in E(\GF(p^{ef})$ taki,
że $\ord(H) = n$ oraz $H \notin (G)$.
Jest to możliwe, gdy $p \nmid n$.
Wartość $w(G, H)$ jest wówczas
pierwiastkiem pierwotnym $n$-tego stopnia z jedności.
Oznaczmy $\mu = w(G, H)$ i policzmy wartość $w(P, H)$:
\begin{eqnarray*}
w(P, H)
& = & w(kG, H) \\
& = & w(G, H)^k \\
& = & \mu^k
\end{eqnarray*}
Elementy $\mu$ i $w(P, H)$ stanowią zatem
egzemplarz problemu logarytmu dyskretnego
w grupie multiplikatywnej ciała $\GF(p^{ef})$,
który ma takie samo rozwiązanie, jak egzemplarz pierwotnego problemu.
\end{proof}

\begin{corollary}
Obliczeniowy problem Diffiego-Hellmana na krzywej eliptycznej $E(\GF(p^e))$
jest nietrudniejszy od obliczeniowego problemu logarytmu dyskretnego
w grupie multiplikatywnej ciała $\GF(p^{ef})$.
\end{corollary}

\begin{theorem}
Decyzyjny problem Diffiego-Hellmana
na krzywej superosobliwej $E = E_{0,1}(\F(p))$
jest łatwy.
\end{theorem}

\begin{proof}
Niech punkt $G \in E$ rzędu $n$
oraz punkty $P = kG$, $Q = lG$ i $R = mG$
będą instancją decyzyjnego problemu Diffiego-Hellmana na krzywej $E$.
Oznaczmy $G = (a, b)$.
Niech $\xi$ oznacza pierwiastek trzeciego stopnia z jedności w ciele $\GF(p^2)$.
Jak wiemy, punkt $H = (\xi a, b)$ również ma rząd $n$
i razem z punktem $G$ generuje grupę $E[n]$.

Oznaczmy odwzorowanie $(x, y) \to (\xi x, y)$ przez $f$.
Zauważmy, że zachodzą następujące zależności:
\begin{eqnarray*}
w(P, f(Q)) & = & w(G, H)^{kl} \\
w(R, f(G)) & = & w(G, H)^m
\end{eqnarray*}
Ponieważ $w(G, H)$ to pierwiastek pierwotny $n$-tego stopnia z jedności,
widzimy, że $kl \equiv m \quad (\mathrm{mod}\ n)$ wtedy i tylko wtedy, gdy
$w(P, f(Q)) = w(R, f(G))$.
\end{proof}

\begin{remark}
Kluczową rolę w dowodzie odegrał fakt, że izomorfizm $f$
między grupami generowanymi przez punkty $G$ i $H$ daje się łatwo obliczać.
Rozumowanie to można uogólnić na wszystkie krzywe,
dla których jesteśmy w stanie wskazać analogiczny izomorfizm,
który można efektywnie obliczać.
\end{remark}


\section{Kryptosystemy bazujące na tożsamości}

\subsection*{Motywacja}

\noindent
\todo{Motywacja.}

\subsection*{Schemat}

\noindent
\todo{Schemat.}

\subsection*{Zastosowanie iloczynu Weila}

\noindent
\todo{Zastosowanie iloczynu Weila.}


\section{Podpisy cyfrowe oparte na tożsamości}

\noindent
Zagadnieniem blisko spokrewnionym z szyfrowaniem są podpisy cyfrowe.
Szyfrowanie gwarantuje,
że wiadomość elektroniczną odczyta tylko wybrany przez nadawcę odbiorca,
zaś podpisy cyfrowe pozwalają odbiorcy stwierdzić,
czy wiadomość jest autentyczna.

\noindent
W przypadku niektórych kryptosystemów
szyfrowanie i wystawianie podpisów cyfrowych jest tak samo skomplikowane.
Przykładowo,
w systemie RSA operacje szyfrowania i odczytywania wiadomości są przemienne,
dzięki czemu podpis cyfrowy można wystawić
poprzez zaszyfrowanie wiadomości kluczem prywatnym, a nie publicznym.

\noindent
W przypadku systemów opartych na tożsamości,
które wykorzystują odwzorowania dwuliniowe,
sytuacja nie jest aż taka prosta.
System Boneha i Franklina jest swego rodzaju kamieniem milowym
w swojej dziedzinie --
trudno wskazać bardziej popularny system szyfrujący oparty na tożsamości.
Niestety, system ten nie pozwala na wystawianie podpisów cyfrowych.

\noindent
Udało się opracować wiele różnych rozwiązań
korzystających z odwzorowań dwuliniowych,
dzięki którym można wystawiać podpisy cyfrowe oparte na tożsamości,
jednak o żadnym z nich nie można powiedzieć,
że jest ono tak istotne dla kryptografii,
jak system Boneha i Franklina.

\noindent
Omówimy teraz jedno z takich rozwiązań
opracowane przez Yi \cite{yi}.
Wybór tego konkretnego rozwiązania na obiekt naszych rozważań
jest podyktowany tym,
że rozwiązanie to jest najbliższe systemowi Boneha-Franklina.
Dzięki występowaniu wielu elementów wspólnych
oba systemy można potraktować niemalże jak części jednego większego systemu.

\subsection*{Motywacja}

\noindent
Podpis cyfrowy oparty na tożsamości można zweryfikować
bez znajomości klucza publicznego domniemanego nadawcy.
Bardzo często tożsamość domniemanego nadawcy
jest wskazana w treści weryfikowanej wiadomości.
Dzięki temu weryfikowanie podpisów jest bardzo łatwe.

\noindent
Rozwiązanie takie może być przydatne w dużych instytucjach,
w których w obiegu pozostaje wiele dokumentów.
Często jest tak, że dokumenty krążą między osobami,
które nie miały wcześniej ze sobą styczności
i nie wymieniły się swoimi kluczami publicznymi.
Systemy oparte na infrastrukturze klucza publicznego
prowadzą centralny rejestr kluczy publicznych
wszystkich członków organizacji.
W rozwiązaniu korzystającym z podpisów opartych na tożsamości
nie występuje punkt centralny,
zatem jest ono bardziej wydajne i skalowalne.

\noindent
Systemy podpisów cyfrowych opartych na tożsamości
mogą też wpłynąć na popularyzację kryptografii
w systemach światowej poczty elektronicznej,
ponieważ znacząco obniżają trud i skomplikowanie
czynności niezbędnych do sprawdzenia autentyczności wiadomości.

\subsection*{Szczegółowy opis działania systemu}

\noindent
Podobnie jak w przypadku systemu Boneha-Franklina,
w skład systemu wchodzą cztery algorytmy:
tworzenie parametrów systemu, tworzenie klucza prywatnego,
podpisywanie wiadomości i weryfikowanie podpisu.

\begin{algorithm}[Tworzenie parametrów systemu]
Następujący algorytm wybiera i przekazuje jako wynik
parametry instancji systemu Yi
i jej klucz główny.

\begin{codebox}
\Procname{$\proc{YI-Create-Instance}()$}
\li
Wybieramy liczby pierwsze $p$ i $q$ takie,
że $p+1 = 12q$.
\li
Ustalamy, że $\G_1$ oznacza podgrupę rzędu $q$ krzywej $E_{0,1}(\F(p))$.
\li
Ustalamy, że $\G_2$ oznacza grupę pierwiastków $q$-tego stopnia w ciele $\GF(p^2)$.
\li
Ustalamy, że odwzorowanie dwuliniowe $b\colon \G_1 \times \G_1 \to \G_2$
jest określone wzorem \ref{modified_weil_pairing_eqn}.
\li
Wybieramy liczbę naturalną $m$.
\li
Ustalamy, że przestrzenią wiadomości $\mathcal{M}$
jest zbiór $\{0, 1\}^m$.
\li
Ustalamy, że przestrzenią podpisów $\mathcal{S}$
jest zbiór $\F(p) \times \F(p)$.
\li
Ustalamy, że przestrzenią identyfikatorów użytkowników $\mathcal{I}$
jest zbiór $\{0, 1\}^\star$.
\li
Wybieramy funkcje haszujące
$H_1\colon \mathcal{I} \times \Z \to \F(p)$
i $H_2\colon \mathcal{M} \times \G_1^\star \to \Z$.
\li
Wybieramy generator $P$ grupy $\G_1$.
\li
Wybieramy element $s$ z grupy $(\Z/q\Z)^\star$.
\li
Obliczamy wartość $Q = sP$.
\li
Parametry systemu to krotka
$\left\langle p, q, \G_1, \G_2, b, m, H_1, H_2, P, Q \right\rangle$.
\li
Klucz główny to element $s$.
\end{codebox}
\end{algorithm}

\begin{algorithm}[Tworzenie klucza prywatnego]
Dany jest identyfikator podpisującego $\const{id}$.
Następujący algorytm na podstawie wartości $\const{id}$
oraz klucza głównego $s$ instancji systemu Yi i jej parametrów
oblicza i przekazuje jako wynik
klucz prywatny odpowiadający identyfikatorowi $\const{id}$.

\begin{codebox}
\Procname{$\proc{YI-Create-Private-Key}(
    \const{id},
    s,
    \left\langle p, q, \G_1, \G_2, b, m, H_1, H_2, P, Q \right\rangle
)$}
\li
Przyjmujemy $k = 0$.
\li
Obliczamy $a = H_1(\const{id}, k)^3 + 1$.
\li
Jeżeli $a^{\frac{p-1}{2}} \neq 1$, to powiększamy wartość $k$ o $1$ i wracamy do kroku 2.
\li
Obliczamy $b = a^{\frac{p+1}{4}}$.
\li
Jeżeli $b > -b\ (\text{mod}\ p)$, to zmieniamy znak wartości $b$.
\li
Obliczamy $U = 12(a, b)$.
\li
Obliczamy $S = sU$.
\li
Klucz prywatny to punkt $S$.
\end{codebox}
\end{algorithm}

\begin{algorithm}[Podpisywanie wiadomości]
Dana jest wiadomość $M$
i klucz prywatny podpisującego $S$.
Następujący algorytm na podstawie wartości $M$ i $S$
oraz parametrów instancji systemu Yi
oblicza i przekazuje jako wynik
podpis wiadomości $M$ odpowiadający kluczowi $S$.

\begin{codebox}
\Procname{$\proc{YI-Sign}(
    M,
    S,
    \left\langle p, q, \G_1, \G_2, b, m, H_1, H_2, P, Q \right\rangle
)$}
\li
Wybieramy element $r$ z grupy $(\Z/q\Z)^\star$.
\li
Obliczamy $R = rP$.
\li
Jeżeli $y(R) \geq -y(R)\ (\text{mod}\ p)$,
to obliczamy $T = rQ + H_2(M, R)S$.
\li
W przeciwnym razie obliczamy $T = -rQ + H_2(M, -R)S$.
\li
Podpis to para $\left\langle x(R), x(T) \right\rangle$.
\end{codebox}
\end{algorithm}

\begin{algorithm}[Weryfikowanie podpisu]
Dana jest wiadomość $M$,
podpis $V$ w postaci $V = \left\langle a, c \right\rangle$
i identyfikator podpisującego $\const{id}$.
Następujący algorytm na podstawie wartości $M$, $V$ i $\const{id}$
oraz parametrów instancji systemu Yi
stwierdza, czy podpis $V$ jest autentyczny,
tzn. został utworzony dla wiadomości $M$
przez podpisującego dysponującego kluczem prywatnym
odpowiadającym identyfikatorowi $\const{id}$.

\begin{codebox}
\Procname{$\proc{YI-Verify}(
    M,
    \left\langle a, c \right\rangle,
    \const{id},
    \left\langle p, q, \G_1, \G_2, b, m, H_1, H_2, P, Q \right\rangle
)$}
\li
Obliczamy $b = (a^3 + 1)^{\frac{p+1}{4}}$.
\li
Obliczamy $d = (c^3 + 1)^{\frac{p+1}{4}}$.
\li
Jeżeli $b \geq -b\ (\text{mod}\ p)$, to przyjmujemy $R' = (a, b)$.
\li
W przeciwnym razie przyjmujemy $R' = (a, -b)$.
\li
Przyjmujemy $T' = (c, d)$.
\li
Obliczamy wartość $U$ tak samo jak w algorytmie tworzenia klucza prywatnego.
\li
Obliczamy $u = b(T', P)$.
\li
Obliczamy $v = b(R' + H_2(M, R')U, Q)$.
\li
Jeżeli $u = v$ lub $u = v^{-1}$, to podpis jest autentyczny.
\end{codebox}
\end{algorithm}

\noindent
Zauważmy, że system Yi nie jest aż tak ogólny, jak system Boneha i Franklina --
jest oparty na superosobliwej krzywej eliptycznej,
nie zaś na dowolnej grupie.
Ponadto, w zaproponowanej przez Boneha i Franklina
konkretnej realizacji systemu
liczba $q$ mogła być dowolnym dzielnikiem liczby $p+1$.
W systemie Yi dodatkowo musi być spełniony warunek $12q = p+1$.
Obostrzenia te wprowadzone są po to,
żeby można było łatwo pierwiastkować w ciele $\F(p)$.

\begin{remark}
Jeżeli $p \equiv 3\ (\text{mod}\ 4)$,
to element $a$ z grupy $(\Z/p\Z)^\star$
jest resztą kwadratową wtedy i tylko wtedy,
gdy $a^{\frac{p-1}{2}} \equiv 1\ (\text{mod}\ p)$.
Ponadto, jeżeli element $a$ jest resztą kwadratową,
to wartość $a^{\frac{p+1}{4}}$ jest jego pierwiastkiem.
\end{remark}

\begin{remark}
Zmienna $k$ w procedurze \proc{YI-Create-Private-Key}
jest powiększana tak długo,
aż wartość $H_1(\const{id}, k)^3 + 1$ będzie resztą kwadratową.
Prawdopodobieństwo zajścia tej sytuacji w jednym kroku wynosi $\frac{1}{2}$,
zatem algorytm generowania klucza prywatnego
prawie na pewno kończy się sukcesem.
\end{remark}

\noindent
Sprawdźmy teraz, że weryfikacja autentycznego podpisu
faktycznie kończy się stwierdzeniem jego autentyczności.

\begin{theorem}
Dana jest wiadomość $M$, identyfikator nadawcy $\const{id}$,
odpowiadający mu klucz prywatny $S$
i parametry $\mathcal{P}$ instancji systemu Yi.
Wówczas zachodzi następująca zależność:

\begin{equation*}
\proc{YI-Verify}(
    M,
    \proc{YI-Sign}(
        M,
        S,
        \mathcal{P}
    ),
    \const{id},
    \mathcal{P}
) = \const{true}
\end{equation*}
\end{theorem}

\begin{proof}
Zauważmy, że punkt $T'$ obliczany w procedurze $\proc{YI-Verify}$
jest równy punktowi $T$ (obliczanemu w procedurze $\proc{YI-Sign}$
lub punktowi $-T$.
Podobnie jest z punktem $R'$.
Rozważmy dwa przypadki.

\begin{enumerate}
\item
Jeżeli $y(R) \geq -y(R)\ (\text{mod}\ p)$, to $R' = R$ i $T' = \pm T$.
Wówczas:

\begin{eqnarray*}
u
& = & b(T', P)
\\
& = & b(\pm T', P)
\\
& = & b(T, P)^{\pm 1}
\\
& = & b(rQ + H_2(M, R)S, P)^{\pm 1}
\\
& = & b(rsP + sH_2(M, R)U, P)^{\pm 1}
\\
& = & b(rP + H_2(M, R)U, P)^{\pm s}
\\
& = & b(R + H_2(M, R)U, Q)^{\pm 1}
\\
& = & b(R' + H_2(M, R')U, Q)^{\pm 1}
\\
& = & v^{\pm 1}
\end{eqnarray*}

\item
Jeżeli $y(R) < -y(R)\ (\text{mod}\ p)$, to $R' = -R$ i $T' = \pm T$.
Wówczas:

\begin{eqnarray*}
u
& = & b(T', P)
\\
& = & b(\pm T', P)
\\
& = & b(T, P)^{\pm 1}
\\
& = & b(-rQ + H_2(M, -R)S, P)^{\pm 1}
\\
& = & b(-rsP + sH_2(M, -R)U, P)^{\pm 1}
\\
& = & b(-rP + H_2(M, -R)U, P)^{\pm s}
\\
& = & b(-R + H_2(M, -R)U, Q)^{\pm 1}
\\
& = & b(R' + H_2(M, R')U, Q)^{\pm 1}
\\
& = & v^{\pm 1}
\end{eqnarray*}
\end{enumerate}
\end{proof}

\noindent
We wspomnianej wcześniej pracy Yi znajduje się analiza bezpieczeństwa
opisanego systemu.

