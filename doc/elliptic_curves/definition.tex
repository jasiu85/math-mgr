\section{Definicja}

\noindent
Na potrzeby niniejszej pracy przyjmujemy
następującą definicję krzywej eliptycznej.

\begin{definition}\label{elliptic_curve_definition}
Dane jest ciało $\K$ oraz dwa jego elementy $A$ i $B$.
\emph{Krzywa eliptyczna nad ciałem $\K$ o parametrach $A$ i $B$},
oznaczana symbolem $E_{A,B}(\K)$ (w skrócie $E(\K)$ lub $E$),
to zbiór składający się
ze wszystkich elementów $(x, y)$ zbioru $\K \times \K$,
dla których zachodzi następująca zależność:
\begin{equation}\label{elliptic_curve_equation}
y^2 = x^3 + Ax + B
\end{equation}
Ponadto, każda krzywa eliptyczna zawiera jeszcze jeden dodatkowy element
oznaczany symbolem $\ecident$.
\end{definition}

\noindent
Ustalamy ponadto następujące określenia.

\begin{definition}
Dana jest krzywa eliptyczna $E$.
Element $\ecident$ krzywej $E$ zwany jest
\emph{punktem w nieskończoności na krzywej $E$} lub jej \emph{identycznością}.
Pozostałe elementy krzywej $E$
zwane są \emph{punktami skończonymi na krzywej $E$}.
\end{definition}

\begin{definition}
Dana jest krzywa eliptyczna $E$.
Punkty skończone $P$ na krzywej $E$ postaci $P = (a, 0)$
zwane są \emph{punktami rzędu dwa na krzywej $E$}.
\end{definition}

\begin{definition}
Dany jest punkt skończony $P$ na krzywej eliptycznej $E$ postaci $P = (a, b)$.
\emph{Punkt sprzężony do punktu $P$},
oznaczany symbolem $\overline{P}$,
to punkt skończony $(a, -b)$.
Ponadto, punkt w nieskończoności na krzywej eliptycznej
uznajemy za sprzężony do samego siebie.
\end{definition}

\begin{fact}\label{conjugate_exists_fact}
Dla każdego punktu na krzywej eliptycznej istnieje dokładnie jeden
punkt sprzężony do niego.
\end{fact}

\begin{fact}
Punkty rzędu dwa na krzywej eliptycznej są sprzężone do samych siebie.
\end{fact}

\begin{definition}
Dana jest krzywa eliptyczna $E$.
Równanie \ref{elliptic_curve_equation} zwane jest \emph{równaniem krzywej $E$}.
Wielomian $x^3 + Ax + B$ występujący po jego prawej stronie,
oznaczany symbolem $\kappa(E)$ (w skrócie $\kappa$),
zwany jest \emph{wielomianem charakterystycznym krzywej $E$}.
\end{definition}

\noindent
W definicji \ref{elliptic_curve_definition} ciało $\K$ może być dowolne,
w szczególności może być skończone lub nie
oraz może mieć dowolną charakterystykę.
Charakterystyka równa $2$ lub $3$ jest źródłem wielu trudności,
np. już sama definicja krzywej eliptycznej
nie jest odpowiednia w takiej sytuacji.

\begin{remark}
Zakładamy odtąd, o ile nie będzie zaznaczone inaczej,
że charakterystyka ciał, nad którymi rozważamy krzywe eliptyczne,
jest różna od $2$ i $3$.
\end{remark}

\noindent
Jest jeszcze jedno źródło trudności,
którym nie będziemy zajmować się w niniejszej pracy.

\begin{definition}
Dana jest krzywa eliptyczna $E$ nad ciałem $\K$.
Jest ona \emph{zdegenerowana},
jeżeli jej wielomian charakterystyczny $\kappa$
ma w domknięciu algebraicznym $\overline{\K}$ ciała $\K$
pierwiastek wielokrotny.
Jeżeli krzywa nie jest zdegenerowana,
to mówimy, że jest \emph{niezdegenerowana}.
\end{definition}

\begin{remark}
Zakładamy odtąd, o ile nie będzie zaznaczone inaczej,
że krzywe eliptyczne, które rozpatrujemy,
są niezdegenerowane.
\end{remark}

\begin{theorem}
Dane jest ciało $\K$ o charakterystyce różnej od $2$ i $3$.
Wówczas krzywa eliptyczna $E$ nad ciałem $\K$ o parametrach $A$ i $B$
jest zdegenerowana wtedy i tylko wtedy,
gdy $\frac{A^3}{27} + \frac{B^2}{4} = 0$.
\end{theorem}

\noindent
Następujące przykłady przedstawiają
krzywe eliptyczne określone nad ciałem liczb rzeczywistych.
Nie będziemy zajmować się takimi krzywymi,
ale posłużą nam one do wyrobienia sobie ,,geometrycznej intuicji''
na temat rozważań w dalszej części pracy.

\begin{example}
Na rysunku \ref{real_nondeg_curves_fig} przedstawione są fragmenty wykresów
niezdegenerowanych krzywych eliptycznych nad ciałem liczb rzeczywistych.
Krzywe różnią się od siebie kształtem:
ilością spójnych składowych lub punktów przegięcia.
Symetria wykresów względem osi odciętych
jest konsekwencją faktu \ref{conjugate_exists_fact}.

\begin{figure}[h]
\centering
\subfloat[$E_{-1,1}(\R)$]{\includegraphics{elliptic_curves_1}}
\hspace{1cm}
\subfloat[$E_{-1,0}(\R)$]{\includegraphics{elliptic_curves_2}}
\hspace{1cm}
\subfloat[$E_{0,-1}(\R)$]{\includegraphics{elliptic_curves_3}}
\caption{Niezdegenerowane krzywe eliptyczne nad ciałem liczb rzeczywistych}
\label{real_nondeg_curves_fig}
\end{figure}
\end{example}

\begin{example}
Na rysunku \ref{real_degen_curves_fig} przedstawione są fragmenty wykresów
zdegenerowanych krzywych eliptycznych nad ciałem liczb rzeczywistych.
Przyczyną zdegenerowania jest raz podwójny, raz potrójny
pierwiastek wielomianu charakterystycznego.

\begin{figure}[h]
\centering
\subfloat[$E_{-3,2}(\R)$]{\includegraphics{elliptic_curves_4}}
\hspace{1cm}
\subfloat[$E_{0,0}(\R)$]{\includegraphics{elliptic_curves_5}}
\caption{Zdegenerowane krzywe eliptyczne nad ciałem liczb rzeczywistych}
\label{real_degen_curves_fig}
\end{figure}
\end{example}

\noindent
Dla kontrastu przedstawiamy również
przykłady krzywych eliptycznych nad ciałem skończonym.

\begin{example}
Na rysunku \ref{finite_curves_fig} przedstawione są wykresy
krzywych eliptycznych nad ciałem skończonym.
Punkt w nieskończoności nie jest zaznaczony.

\begin{figure}[h]
\centering
\subfloat[$E_{2,6}(\F(13))$]{\includegraphics{elliptic_curves_6}}
\hspace{0.5cm}
\subfloat[$E_{7,3}(\F(13))$]{\includegraphics{elliptic_curves_7}}
\caption{Krzywe eliptyczne nad ciałem skończonym $\F(13)$}
\label{finite_curves_fig}
\end{figure}
\end{example}
