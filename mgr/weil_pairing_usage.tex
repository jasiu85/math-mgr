\chapter{Zastosowania iloczynu Weila}

\noindent
Iloczyn Weila jest narzędziem używanym w kryptografii wtedy,
gdy rozpatrywane są zagadnienia związane z krzywymi eliptycznymi.
Jak każde narzędzie, można go wykorzystać na dwa sposoby:
sposób ,,dobry'' polega na tworzeniu kryptosystemów z pomocą iloczynu Weila,
zaś sposób ,,zły'' polega na przeprowadzaniu ataków na nie.

\noindent
W rozdziale tym przedstawimy zastosowania iloczynu Weila obu rodzajów.
Omówimy jeden atak kryptograficzny
oraz grupę kryptosystemów,
które mają pewną specyficzną cechę wspólną.

\section{Superosobliwe krzywe eliptyczne}

\noindent
Opiszemy najpierw pewien szczególny rodzaj krzywych eliptycznych,
które dobrze nadają do się zastosowań związanych z iloczynem Weila.

\begin{definition}
Dana jest liczba pierwsza $p$ taka, że $p \equiv 2\ (\mathrm{mod}\ 3)$.
\emph{Superosobliwa krzywa eliptyczna nad ciałem $\F(p)$}
to krzywa $E_{0,1}(\F(p))$.
\end{definition}

\begin{remark}
Krzywe superosobliwe definiuje się tak naprawdę w zupełnie inny,
bardziej ogólny sposób. Podana definicja to jedynie przykład
pewnej rodziny krzywych, które są superosobliwe w sensie ogólnej definicji.
Inną rodziną krzywych superosobliwych są krzywe postaci $E_{1,0}(\F(p))$,
gdzie $p \equiv 3\ (\mathrm{mod}\ 4)$.
\end{remark}

\noindent
Pierwszą interesującą własnością krzywych superosobliwych
jest ich struktura grupowa.

\begin{theorem}
Krzywa superosobliwa $E_{0,1}(\F(p))$ składa się z $p + 1$ punktów.
\end{theorem}

\begin{proof}
Rozważmy równanie $x^3 = c$ w ciele $\F(p)$.
Weźmy dowolne dwa elementy $a, b \in \F(p)$ takie, że $a^3 = c$ i $b^3 = c$.
Gdy $c = 0$, to $a = b = 0$.
Gdy $c \neq 0$, zapisujemy elementy $a$ i $b$ jako potęgi generatora:
$a = g^k$, $b = g^l$.
Wiemy, że $a^3 = b^3$, skąd dostajemy $g^{3(k-l)} = 1$.
Zatem $p-1 \mid 3(k-l)$.
Ponieważ liczba $p-1$ jest względnie pierwsza z $3$,
otrzymujemy $p-1 \mid k-l$.
Stąd $a = b$.

\noindent
Z rozważań tych wynika następujący wniosek:
funkcja $x^3 + 1$ jest bijekcją w zbiorze $\F(p)$,
w jej obrazie znajdują się wszystkie reszty kwadratowe,
zatem krzywa superosobliwa $E_{0,1}(\F(p))$ zawiera dokładnie $p+1$ punktów.
\end{proof}

\begin{theorem}
Grupa na krzywej superosobliwej $E_{0,1}(\F(p))$
jest cykliczna.
\end{theorem}

\begin{proof}
Zgodnie z twierdzeniem \ref{finite_curve_structure_theorem}
krzywa $E$ jest izomorficzna z grupą $(\Z / k\Z) \times (\Z / l\Z)$,
gdzie $kl = p+1$ oraz $l \mid \gcd(k, p-1)$.
Skoro $k \mid p+1$, to $l \mid \gcd(p+1, p-1)$,
zatem $l = 1$ lub $l = 2$.
W pierwszym przypadku twierdzenie jest udowodnione.
Pokażemy, że drugi jest niemożliwy.

\noindent
Przyjmijmy, że $l = 2$.
Jeżeli $2 \nmid k$,
to grupa $(\Z / k\Z) \times (\Z / l\Z)$ jest izomorficzna
z żądaną grupą $\Z / (p+1)\Z$.
Natomiast jeśli $2 \mid k$, to na krzywej $E$ są dwa punkty rzędu dwa,
a więc wielomian $x^3 + 1$ ma w ciele $\F(p)$ dwa miejsca zerowe.
To jest niemożliwe, bo wielomian ten jest bijekcją w zbiorze $\F(p)$,
jak zauważyliśmy wcześniej.
\end{proof}

\noindent
W kontekście iloczynu Weila krzywe superosobliwe są ważne dlatego,
że łatwo wskazywać ich podgrupy $n$-torsyjne.

\begin{theorem}
Dana jest liczba pierwsza $p$ taka, że $p \equiv 2\ (\mathrm{mod}\ 3)$
oraz liczba naturalna $n$ będąca dzielnikiem liczby $p + 1$.
Wówczas podgrupa $n$-torsyjna krzywej eliptycznej $E_{0,1}(\GF(p^2))$
jest izomorficzna z grupą $(\Z / n\Z) \times (\Z / n\Z)$.
\end{theorem}

\begin{proof}
Ponieważ $3 \mid p^2-1$, w ciele $\GF(p^2)$
istnieje nietrywialny pierwiastek trzeciego stopnia z jedności.
Oznaczmy go przez $\xi$.
Wówczas automorfizmem krzywej $E_{0,1}(\GF(p^2))$ jest
odwzorowanie $\phi\colon E_{0,1}(\F(p^2)) \to E_{0,1}(\GF(p^2))$
określone następująco:
\begin{equation}\label{supersingular_curve_automorphism_eqn}
\phi(x, y) = (\xi x, y)
\end{equation}

\noindent
Niech $P$ będzie takim punktem krzywej $E_{0,1}(\F(p))$, że $nP = \ecident$.
Punkt $\phi(P)$ również ma rząd $n$.
Zauważmy przy tym, że $\phi(P) \notin E_{0,1}(\F(p))$,
bo $\xi \notin \F(p)$.
Wynika stąd już, że podgrupa $n$-torsyjna krzywej $E_{0,1}(\GF(p^2))$
ma żądaną postać, a jej generatorami są punkty $P$ i $\phi(P)$.
\end{proof}


\section{Redukcja MOV}

Pierwsze zastosowanie iloczynu Weila
związane jest
z problemem logarytmu dyskretnego oraz protokołem Diffiego-Hellmana.

Podajmy najpierw definicje obu problemów.

\begin{problem}[Logarytm dyskretny]
Dana jest grupa cykliczna $\G$,
jej generator $g \in \G$
oraz jej element $a \in \G$.
Znaleźć liczbę całkowitą $k \in \Z$ taką, że:
\begin{equation}\label{discrete_log_eqn}
g^k = a
\end{equation}
\end{problem}

\begin{remark}
Zwyczajowo problem logarytmu dyskretnego opisuje się,
stosując zapis multiplikatywny działania grupowego.
W przypadku zapisu addytywnego równanie \ref{discrete_log_eqn}
przybiera postać:
\begin{equation}\label{discrete_log_additive_eqn}
kg = a
\end{equation}
\end{remark}

\begin{remark}
Problem logarytmu dyskretnego można rozpatrywać w przypadku dowolnej grupy.
Należy wówczas rozpatrywać jej podgrupę $(g)$
generowaną przez pewien jej element $g$.
\end{remark}

\begin{remark}
Problem logarytmu dyskretnego w grupie $\Z / n\Z$ jest prosty,
ponieważ dysponujemy algorytmem obliczającym rozwiązania
równania \ref{discrete_log_additive_eqn} w grupie $\Z / n\Z$.
\end{remark}

\begin{remark}
Problem logarytmu dyskretnego w grupie multiplikatywnej ciała $\GF(p^e)$
jest obecnie uważany za problem trudny obliczeniowo.
\end{remark}

Jak wiemy, grupa multiplikatywna ciała $\GF(p^e)$
jest izomorficzna z grupą $\Z / (p^e-1)\Z$.
Mimo to problemy logarytmu dyskretnego w tych grupach
różnią się trudnością,
co jest spowodowane różnicą w sposobie,
w jaki reprezentujemy elementy obu grup
oraz obliczamy wyniki działania grupowego.

Problemem spokrewnionym z problemem logarytmu dyskretnego
jest zagadnienie złamania protokołu Difiego-Hellmana,
zwane w skrócie ,,problemem Diffiego-Hellmana''.

\begin{problem}[Diffie-Hellman]
Dana jest grupa cykliczna $\G$,
jej generator $g \in G$
oraz elementy $a = g^k$ i $b = g^l$,
gdzie $k, l \in \Z$.
Znaleźć element $c \in \G$ taki, że:
\begin{equation}
c = g^{kl}
\end{equation}
\end{problem}

\begin{remark}
Podobnie jak w przypadku problemu logarytmu dyskretnego,
możemy stosować zapis addytywny (szukamy wówczas wartości $klg$)
oraz rozpatrywać problem w dowolnej grupie.
\end{remark}

Pokrewieństwo obu problemów polega na tym,
że jeden można zredukować do drugiego.

\begin{theorem}
Problem Diffiego-Hellmana w grupie $\G$ można
w czasie wielomianowym w sposób deterministyczny
zredukować do problemu logarytmu dyskretnego w grupie $\G$.
\end{theorem}

\begin{proof}
Redukcja jest bardzo prosta.
Jeżeli dysponujemy algorytmem rozwiązującym
problem logarytmu dyskretnego w grupie $\G$,
to postępujemy następująco:
\begin{enumerate}
\item na podstawie wartości $\G$, $g$ i $a = g^k$ obliczamy wartość $k$;
\item na podstawie wartości $\G$, $g$ i $b = g^l$ obliczamy wartość $l$;
\item obliczamy $kl$;
\item obliczamy $g^{kl}$.
\end{enumerate}
Jak widać, aby rozwiązać egzemplarz problemu Diffiego-Hellmana
wystarczy dwa razy zastosować rozwiązanie problemu logarytmu dyskretnego.
\end{proof}

Czy problem Diffiego-Hellmana można rozwiązać inaczej?
Być może wystarczy odpowiednio wykorzystać inne operacje,
które daje się efektywnie obliczać.
Zagadnienie to jest o tyle istotne,
że na trudności problemu Diffiego-Hellmana
opiera się wiele kryptosystemów,
których implementacje są wykorzystywane na codzień,
m.in. w protokole SSL używanym w sieci Internet.
Wydaje się, że oba problemy są równoważne (por. \ref{maurer}),
a to oznacza trudność problemu Diffiego-Hellmana,
a zatem bezpieczeństwo używanych protokołów.

Oba podane problemy zostały przedstawione w wersji obliczeniowej.
W teorii złożoności obliczeniowej często rozpatruje się wersje decyzyjne
problemów. Problem Diffiego-Hellmana przybiera wówczas następującą postać.

\begin{problem}
Dana jest grupa $\G$,
jej generator $g \in \G$
oraz elementy $a = g^k$, $b = g^l$ i $c = g^m$
gdzie $k, l, m \in \Z$.
Stwierdzić, czy zachodzi następująca zależność:
\begin{equation}
kl \equiv m \quad (\mathrm{mod}\ \abs{\G})
\end{equation}
\end{problem}

Jest jasne, że problem decyzyjny można zredukować do problemu obliczeniowego,
zatem jest on nietrudniejszy. Może się zdarzyć, że jest dużo prostszy.

W przypadku krzywych eliptycznych obie wersje problemu Diffiego-Hellmana
są następujące.

\begin{problem}
Dana jest krzywa eliptyczna $E(\K)$,
punkt $G \in E$ skończonego rzędu $n$
oraz punkty $P = kG$ i $Q = lG$,
gdzie $k, l \in \Z$.
Znaleźć punkt $R \in E$ taki, że:
\begin{equation}
R = klG
\end{equation}
\end{problem}

\begin{problem}
Dana jest krzywa eliptyczna $E(\K)$,
punkt $G \in E$ skończonego rzędu $n$
oraz punkty $P = kG$, $Q = lG$ i $R = mG$,
gdzie $k, l, m \in \Z$.
Stwierdzić, czy zachodzi następująca zależność:
\begin{equation}
kl \equiv m \quad (\mathrm{mod}\ n)
\end{equation}
\end{problem}

Czy problem Diffiego-Hellmana na krzywej eliptycznej jest trudny?
Okazuje się, że iloczyn Weila ma wpływ na tę kwestię.

\begin{theorem}[Redukcja MOV]
Obliczeniowy problem logarytmu dyskretnego
na krzywej eliptycznej $E = E(\GF(p^e))$
można w czasie wielomianowym w sposób deterministyczny zredukować
do obliczeniowego problemu logarytmu dyskretnego
w grupie multiplikatywnej ciała $\GF(p^{ef})$.
\end{theorem}

\begin{proof}
Niech punkt $G \in E$ rzędu $n$
oraz punkt $P = kG$ będą instancją
obliczeniowego problemu logarytmu dyskretnego na krzywej $E$.
Rozważmy rozszerzenie $\GF(p^{ef})$ ciała $\GF(p^e)$ dostatecznie duże,
aby istniał punkt $H \in E(\GF(p^{ef})$ taki,
że $\ord(H) = n$ oraz $H \notin (G)$.
Jest to możliwe, gdy $p \nmid n$.
Wartość $w(G, H)$ jest wówczas
pierwiastkiem pierwotnym $n$-tego stopnia z jedności.
Oznaczmy $\mu = w(G, H)$ i policzmy wartość $w(P, H)$:
\begin{eqnarray*}
w(P, H)
& = & w(kG, H) \\
& = & w(G, H)^k \\
& = & \mu^k
\end{eqnarray*}
Elementy $\mu$ i $w(P, H)$ stanowią zatem
egzemplarz problemu logarytmu dyskretnego
w grupie multiplikatywnej ciała $\GF(p^{ef})$,
który ma takie samo rozwiązanie, jak egzemplarz pierwotnego problemu.
\end{proof}

\begin{corollary}
Obliczeniowy problem Diffiego-Hellmana na krzywej eliptycznej $E(\GF(p^e))$
jest nietrudniejszy od obliczeniowego problemu logarytmu dyskretnego
w grupie multiplikatywnej ciała $\GF(p^{ef})$.
\end{corollary}

\begin{theorem}
Decyzyjny problem Diffiego-Hellmana
na krzywej superosobliwej $E = E_{0,1}(\F(p))$
jest łatwy.
\end{theorem}

\begin{proof}
Niech punkt $G \in E$ rzędu $n$
oraz punkty $P = kG$, $Q = lG$ i $R = mG$
będą instancją decyzyjnego problemu Diffiego-Hellmana na krzywej $E$.
Oznaczmy $G = (a, b)$.
Niech $\xi$ oznacza pierwiastek trzeciego stopnia z jedności w ciele $\GF(p^2)$.
Jak wiemy, punkt $H = (\xi a, b)$ również ma rząd $n$
i razem z punktem $G$ generuje grupę $E[n]$.

Oznaczmy odwzorowanie $(x, y) \to (\xi x, y)$ przez $f$.
Zauważmy, że zachodzą następujące zależności:
\begin{eqnarray*}
w(P, f(Q)) & = & w(G, H)^{kl} \\
w(R, f(G)) & = & w(G, H)^m
\end{eqnarray*}
Ponieważ $w(G, H)$ to pierwiastek pierwotny $n$-tego stopnia z jedności,
widzimy, że $kl \equiv m \quad (\mathrm{mod}\ n)$ wtedy i tylko wtedy, gdy
$w(P, f(Q)) = w(R, f(G))$.
\end{proof}

\begin{remark}
Kluczową rolę w dowodzie odegrał fakt, że izomorfizm $f$
między grupami generowanymi przez punkty $G$ i $H$ daje się łatwo obliczać.
Rozumowanie to można uogólnić na wszystkie krzywe,
dla których jesteśmy w stanie wskazać analogiczny izomorfizm,
który można efektywnie obliczać.
\end{remark}


\section{Szyfrowanie oparte na tożsamości}

\noindent
Opiszemy teraz system szyfrujący
opracowany przez Boneha i Franklina \cite{bonehfranklin}.
Iloczyn Weila odgrywa w tym systemie kluczową rolę.

\noindent
System pozwala szyfrować wiadomości w sposób asymetryczny,
tzn. występują w nim klucze publiczne do szyfrowania wiadomości
oraz klucze prywatne do ich odczytywania.
W skład systemu wchodzą cztery algorytmy,
których przeznaczenie jest następujące.

\begin{itemize}
\item
Algorytm tworzenia parametrów systemu
jest wykorzystywany przez zarządcę systemu raz,
podczas tworzenia instancji systemu.
Wynikiem jego działania są parametry systemu,
które zarządca powinien udostępnić
wszystkim użytkownikom chcącym korzystać z systemu
oraz klucz główny, którego nie powinien ujawniać.

\item
Algorytm tworzenia klucza prywatnego
jest wykorzystywany przez zarządcę wtedy,
gdy nowy użytkownik chce rozpocząć korzystanie z systemu.
Za pomocą tego algorytmu zarządca
tworzy dla nowego użytkownika klucz prywatny,
którego użytkownik nie powinien ujawniać.
Klucz prywatny jest wyznaczany jednoznacznie
na podstawie parametrów systemu, klucza głównego
i dowolnie wybranego ciągu bitów.
Ciąg ten będzie identyfikatorem nowego użytkownika,
który inni użytkownicy będą wykorzystywać,
aby wysyłać do niego zaszyfrowane wiadomości.

\item
Wiadomości są szyfrowane za pomocą algorytmu wykorzystującego
parametry systemu i identyfikator odbiorcy.
Dodatkowo, w procesie szyfrowania używany jest losowo wybrany parametr,
dlatego w wyniku wielokrotnego zaszyfrowania tej samej wiadomości
można otrzymać różne kryptogramy.

\item
Kryptogramy odczytywane są za pomocą algorytmu wykorzystującego
parametry systemu i klucz prywatny odbiorcy.
\end{itemize}

\noindent
Jak widać, kluczową cechą odróżniającą ten system od innych rozwiązań
jest możliwość użycia dowolnego ciągu bitów jako klucza publicznego.
O systemie mającym tę cechę mówimy, że jest ,,oparty na tożsamości''.
Takie podejście, jak zaraz zobaczymy, oferuje możliwości,
których nie dają inne kryptosystemy.

\subsection*{Motywacja}

\noindent
Przedstawiamy dwa typy problemów,
których rozwiązanie za pomocą kryptosystemów
nieopierających się na tożsamości
jest albo niemożliwe,
albo bardzo trudne.
Natomiast łatwo jest rozwiązać je za pomocą systemu Boneha-Franklina.

\begin{itemize}
\item
Dana jest grupa użytkowników (np. instytucja państwowa lub korporacja),
którzy chcą przesyłać między sobą zaszyfrowane wiadomości.
Zastosowanie w tym celu systemu Boneha-Franklina daje następujące korzyści.

\begin{itemize}
\item
Użytkownicy nie muszą przechowywać kluczy publicznych innych użytkowników.
Ułatwia to nawiązanie korespondencji z nową osobą,
bo łatwiej jest przekazać innym kanałem (np. za pomocą wizytówki)
swój identyfikator (np. adres poczty elektronicznej)
niż swój klucz publiczny.
Dalej, nie ma ryzyka utracenia listy kluczy publicznych innych użytkowników.
Listę taką zawsze można odtworzyć, ale może być to bardziej żmudne
niż odtworzenie listy adresów poczty elektronicznej.

\item
Użytkownicy nie muszą przechowywać nawet swojego klucza prywatnego --
zawsze można ponownie poprosić zarządcę systemu o jego wygenerowanie
(uprzednio potwierdzając swoją tożsamość innymi metodami).
Utrata klucza prywatnego w innych systemach może spowodować
nieodwracalne konsekwencje.

\item
Można wysyłać wiadomości do przyszłych członków grupy,
dla których nie został jeszcze utworzony klucz prywatny.
Wystarczy, że znany jest identyfikator, którym będą się posługiwać.

\item
Łatwo jest wykluczać użytkowników z grupy.
Wystarczy, że kluczem publicznym będzie ciąg bitów będący wynikiem
połączenia identyfikatora odbiorcy z rokiem (odpowiednio, rokiem i miesiącem)
wysłania wiadomości.
W ten sposób każdy członek grupy musi co rok (odpowiednio, co miesiąc)
poprosić zarządcę o wygenerowanie nowego klucza prywatnego,
który będzie ważny przez kolejny rok (odpowiednio, miesiąc).

\item
Można wysyłać wiadomości, które będą mogły być odczytane dopiero w przyszłości.
Wystarczy zmodyfikować poprzedni pomysł:
zamiast bieżącej daty można do identyfikatora odbiorcy
dokleić datę przyszłą, która określa,
kiedy wiadomość będzie mogła być odczytana.

\item
Członkom grupy można nadawać poziomy uprawnień do odczytywania wiadomości.
W tym celu kluczem publicznym powinien być ciąg bitów składający się
z identyfikatora odbiorcy, daty oraz nazwy uprawnienia,
które jest wymagane do odczytu nadawanej wiadomości.
Jest to uogólnienie poprzednich dwóch pomysłów.
\end{itemize}

\noindent
Ponieważ nie trzeba przechowywać kluczy publicznych,
szyfrowanie elektronicznej korespondencji staje się bardziej dostępne
dla przeciętnego użytkownika poczty elektronicznej.
Zauważmy przy tym, że nierozwiązany pozostaje problem przesyłania
zaszyfrowanych wiadomości między członkami różnych grup,
np. między użytkownikami dwóch różnych serwerów poczty elektronicznej.
Dlatego też nie można do końca zrezygnować
z infrastruktury klucza publicznego.

\noindent
Zauważmy, że zastosowanie systemu Boneha i Franklina w opisany sposób
prowadzi do następującego problemu: zarządca systemu może odczytać
dowolną zaszyfrowaną wiadomość.
Z kryptograficznego punktu widzenia jest to zjawisko zdecydowanie niepożądane,
jednak obecnie praktykuje się właśnie takie rozwiązania:
administrator serwera poczty elektronicznej ma dostęp do wszystkich wiadomości.

\item
Przypuśćmy, że pewien użytkownik
systemu opartego na infrastrukturze klucza publicznego
chce zabezpieczyć się przed ujawnieniem swojego klucza prywatnego.
Sytuacja taka może mieć miejsce,
jeżeli będzie przechowywał klucz prywatny na niezaufanych urządzeniach
(np. laptopie lub telefonie komórkowym, które mogą zostać skradzione)
lub powierzy go niezaufanym osobom
(np. swoim asystentom, których zadaniem jest
pomoc w prowadzeniu korespondencji).
W tym celu może wykorzystać system Boneha-Franklina.
Powinien utworzyć własną instancję systemu
i poprosić wszystkich, którzy wysyłają do niego wiadomości,
żeby szyfrowali je kluczem,
który powstaje z połączenia daty wysłania wiadomości
z kategorią tematyczną wiadomości.
Opisane problemy można wówczas rozwiązać następująco.

\begin{itemize}
\item
Użytkownik ma dostęp do całej swojej korespondencji,
ponieważ jest w posiadaniu klucza głównego
i może z jego pomocą odszyfrować dowolną wiadomość.

\item
Swoim asystentom może wydać klucze prywatne,
które pozwalają na odczytanie wiadomości jedynie z tej kategorii tematycznej,
za którą są odpowiedzialni.

\item
Na zagrożone kradzieżą urządzenie
użytkownik może nagrać klucze prywatne pozwalające na odczytywanie wiadomości
wysłanych tylko w zadanym okresie czasu (np. jeden tydzień).
Jeżeli urządzenie zostanie skradzione,
złodziej nie będzie w stanie odczytać
wiadomości wysłanych po upływie zadanego okresu.
\end{itemize}
\end{itemize}

\subsection*{Szczegółowy opis działania systemu}

\noindent
Przedstawimy teraz działanie wszystkich algorytmów
wchodzących w skład systemu Boneha-Franklina.
Opiszemy również rodzaje danych pojawiających się w systemie.

\begin{algorithm}[Tworzenie parametrów systemu]
Następujący algorytm wybiera i przekazuje jako wynik
parametry instancji systemu Boneha-Franklina
i jej klucz główny.

\begin{codebox}
\Procname{$\proc{BF-Create-Instance}()$}
\li
Wybieramy liczbę pierwszą $q$.
\li
Wybieramy grupy cykliczne $\G_1$ i $\G_2$ rzędu $q$.
\li
Wybieramy odwzorowanie dwuliniowe $b\colon \G_1 \times \G_1 \to G_2$.
\li
Wybieramy liczbę naturalną $m$.
\li
Ustalamy, że przestrzenią wiadomości $\mathcal{M}$
jest zbiór $\{0, 1\}^m$.
\li
Ustalamy, że przestrzenią kryptogramów $\mathcal{C}$
jest zbiór $\G_1^\star \times \mathcal{M}$.
\li
Ustalamy, że przestrzenią identyfikatorów użytkowników $\mathcal{I}$
jest zbiór $\{0, 1\}^\star$.
\li
Wybieramy funkcje haszujące
$H_1\colon \mathcal{I} \to \G_1^\star$
i $H_2\colon \G_2 \to \mathcal{M}$.
\li
Wybieramy generator $P$ grupy $\G_1$.
\li
Wybieramy element $s$ z grupy $(\Z / q\Z)^\star$.
\li
Obliczamy wartość $Q = sP$.
\li
Parametry systemu to krotka
$\left\langle q, \G_1, \G_2, b, m, H_1, H_2, P, Q \right\rangle$.
\li
Klucz główny to element $s$.
\end{codebox}
\end{algorithm}

\begin{algorithm}[Tworzenie klucza prywatnego]
Dany jest identyfikator $\const{id}$.
Następujący algorytm na podstawie wartości $\const{id}$
oraz klucza głównego $s$ instancji systemu Boneha-Franklina i jej parametrów
oblicza i przekazuje jako wynik
klucz prywatny odpowiadający identyfikatorowi $\const{id}$.

\begin{codebox}
\Procname{$\proc{BF-Create-Private-Key}(
    \const{id},
    s,
    \left\langle q, \G_1, \G_2, b, m, H_1, H_2, P, Q \right\rangle
)$}
\li
Obliczamy wartość $R = H_1(\const{id})$.
\li
Obliczamy wartość $S = sR$.
\li
Klucz prywatny to punkt $S$.
\end{codebox}
\end{algorithm}

\begin{algorithm}[Szyfrowanie wiadomości]
Dana jest wiadomość $M$ i identyfikator adrestata $\const{id}$.
Następujący algorytm na podstawie wartości $M$ i $\const{id}$
oraz parametrów instancji systemu Boneha-Franklina
oblicza i przekazuje jako wynik
kryptogram odpowiadający wiadomości $M$.

\begin{codebox}
\Procname{$\proc{BF-Encrypt}(
    M,
    \const{id},
    \left\langle q, \G_1, \G_2, b, m, H_1, H_2, P, Q \right\rangle
)$}    
\li
Obliczamy wartość $R = H_1(\const{id})$.
\li
Wybieramy element $r$ z grupy $(\Z / q\Z)^\star$.
\li
Obliczamy wartość $U = rP$.
\li
Obliczamy wartość $V = M\ \kw{xor}\ b(R, Q)^r$.
\li
Kryptogram to para
$\left\langle U, V \right\rangle$.
\end{codebox}
\end{algorithm}

\begin{algorithm}[Odczytywanie kryptogramu]
Dany jest kryptogram $C$ w postaci $C = \left\langle U, V \right\rangle$
i klucz prywatny $S$ odpowiadający identyfikatorowi $\const{id}$.
Następujący algorytm
na podstawie wartości $C$ i $S$
oraz parametrów instancji systemu Boneha-Franklina
oblicza i przekazuje jako wynik
wiadomość odpowiadającą kryptogramowi $C$.

\begin{codebox}
\Procname{$\proc{BF-Decrypt}(
    \left\langle U, V \right\rangle,
    S,
    \left\langle q, \G_1, \G_2, b, m, H_1, H_2, P, Q \right\rangle
)$}
\li
Obliczamy wartość $M = V\ \kw{xor}\ b(S, U)$.
\li
Wiadomość to wartość $M$.
\end{codebox}
\end{algorithm}

\begin{remark}
Zauważmy, że nie sprecyzowaliśmy, jaka jest struktura grup $\G_1$ i $\G_2$.
Kwestię tę omówimy za chwilę.
\end{remark}

\noindent
Pokażmy przede wszystkim, że w systemie Boneha-Franklina
prawidłowo działa szyfrowanie i odczytywanie wiadomości.

\begin{theorem}
Dana jest wiadomość $M$,
identyfikator adresata $\const{id}$,
odpowiadający mu klucz prywatny $S$,
i parametry $\mathcal{P}$ instancji systemu Boneha-Franklina.
Wówczas zachodzi następująca zależność:
\begin{equation*}
\proc{BF-Decrypt}(
    \proc{BF-Encrypt}(
        M,
        \const{id},
        \mathcal{P}
    ),
    S,
    \mathcal{P}
) = M
\end{equation*}
\end{theorem}

\begin{proof}
Wiadomość $M$ jest szyfrowana symetrycznie za pomocą operacji \kw{xor}.
Wystarczy zatem sprawdzić,
że klucze użyte przy szyfrowaniu i odczytywaniu wiadomości są takie same.
Istotnie:
\begin{eqnarray*}
b(S, U)
& = & b(sR, U) \\
& = & b(R, U)^s \\
& = & b(R, rP)^s \\
& = & b(R, P)^{sr} \\
& = & b(R, sP)^r \\
& = & b(R, Q)^r
\end{eqnarray*}
\end{proof}

\noindent
Sednem sposobu szyfrowania w systemie Boneha-Franklina
jest wykorzystanie prostego algorytmu symetrycznego do szyfrowania wiadomości,
stosowanie w tym algorytmie kluczy jednorazowych
oraz sprytny sposób przekazania informacji o użytym kluczu.
Sekretną informacją potrzebną do odczytania wiadomości
jest losowa wartość $r$, którą nadawca ,,rozdziela'' na dwie części.
Jedynie odbiorca jest w stanie połączyć je z powrotem w całość
i odczytać wiadomość.
Złączenie to odbywa się za pomocą operacji dwuliniowej $b$.
Sposób, w jaki sekret jest rozdzielony i przekazany w częściach,
koncepcyjnie przypomina działanie protokołu Diffiego-Hellmana.

\noindent
Konkretna realizacja systemu Boneha-Franklina
wymaga wybrania konkretnych grup $\G_1$ i $\G_2$
oraz odwzorowania dwuliniowego $b$.
W swojej pracy Boneh i Franklin opisują następującą realizację systemu
opartą na superosobliwych krzywych eliptycznych.
\begin{itemize}
\item
Podstawą systemu jest superosobliwa krzywa eliptyczna $E_{0,1}(\F(p))$,
przy czym $q \mid p+1$.
Zauważmy, że w tej sytuacji $E[q] \subset E_{0,1}(\GF(p^2))$.
\item
Rolę grupy $\G_1$ pełni grupa generowana
przez punkt $P$ rzędu $q$ na krzywej $E$.
\item
Grupa $\G_2$ to grupa pierwiastków $q$-tego stopnia z jedności
w ciele $\GF(p^2)$.
\item
Niech $\phi$ będzie funkcją określoną
równaniem \ref{supersingular_curve_automorphism_eqn}.
Odwzorowanie dwuliniowe $b$ używane w systemie
jest określone następująco:
\begin{equation}\label{modified_weil_pairing_eqn}
b(P, Q) = w(P, \phi(Q))
\end{equation}
\end{itemize}

\begin{remark}
Zauważmy, że iloczyn Weila jest zdegenerowany na krzywej $E_{0,1}(\F(p))$,
dlatego w definicji funkcji $b$ jeden z argumentów jest zmodyfikowany
za pomocą automorfizmu $\phi$.
Można sprawdzić, że uzyskane w ten sposób odwzorowanie jest dwuliniowe.
\end{remark}

\noindent
W pracy Boneha i Franklina można znaleźć analizę bezpieczeństwa systemu.
Zaznaczmy, że aby uzyskać gwarancje bezpieczeństwa na tyle rozsądne,
żeby można było korzystać z systemu w praktyce,
należy zastosować zmodyfikowaną wersję opisanego tutaj schematu.
Opis tej modyfikacji można znaleźć w pracy \cite{fujisakiokamoto}.


\section{Podpisy cyfrowe oparte na tożsamości}

\noindent
Zagadnieniem blisko spokrewnionym z szyfrowaniem są podpisy cyfrowe.
Szyfrowanie gwarantuje,
że wiadomość elektroniczną odczyta tylko wybrany przez nadawcę odbiorca,
zaś podpisy cyfrowe pozwalają odbiorcy stwierdzić,
czy wiadomość jest autentyczna.

\noindent
W przypadku niektórych kryptosystemów
szyfrowanie i wystawianie podpisów cyfrowych jest tak samo skomplikowane.
Przykładowo,
w systemie RSA operacje szyfrowania i odczytywania wiadomości są przemienne,
dzięki czemu podpis cyfrowy można wystawić
poprzez zaszyfrowanie wiadomości kluczem prywatnym, a nie publicznym.

\noindent
W przypadku systemów opartych na tożamości,
które wykorzystują odwzorowania dwuliniowe,
sytuacja nie jest aż taka prosta.
System Boneha i Franklina jest swego rodzaju kamieniem milowym
w swojej dziedzinie --
trudno wskazać bardziej popularny system szyfrujący oparty na tożsamości.
Niestety, system ten nie pozwala na wystawianie podpisów cyfrowych.

\noindent
Udało się opracować wiele różnych rozwiązań
korzystających z odwzorowań dwuliniowych,
dzięki którym można wystawiać podpisy cyfrowe oparte na tożsamości,
jednak o żadnym z nich nie można powiedzieć,
że jest ono tak istotne dla kryptografii,
jak system Boneha i Franklina.

\noindent
Omówimy teraz jedno z takich rozwiązań
opracowane przez Yi \cite{yi}.
Wybór tego konkretnego rozwiązania na obiekt naszych rozważań
jest podyktowany tym,
że rozwiązanie to jest najbliższe systemowi Boneha-Franklina.
Dzięki występowaniu wielu elementów wspólnych
oba systemy można potraktować niemalże jak części jednego większego systemu.

\subsection*{Motywacja}

\noindent
Podpis cyfrowy oparty na tożsamości można zweryfikować
bez znajomości klucza publicznego domniemanego nadawcy.
Bardzo często tożsamość domniemanego nadawcy
jest wskazana w treści weryfikowanej wiadomości.
Dzięki temu weryfikowanie podpisów jest bardzo łatwe.

\noindent
Rozwiązanie takie może być przydatne w dużych instytucjach,
w których w obiegu pozostaje wiele dokumentów.
Często jest tak, że dokumenty krążą między osobami,
które nie miały wcześniej ze sobą styczności
i nie wymieniły się swoimi kluczami publicznymi.
Systemy oparte na infrastrukturze klucza publicznego
prowadzą centralny rejestr kluczy publicznych
wszystkich członków organizacji.
W rozwiązaniu korzystającym z podpisów opartych na tożsamości
nie występuje punkt centralny,
zatem jest ono bardziej wydajne i skalowalne.

\noindent
Systemy podpisów cyfrowych opartych na tożsamości
mogą też wpłynąć na popularyzację kryptografii
w systemach światowej poczty elektronicznej,
ponieważ znacząco obniżają trud i skomplikowanie
czynności niezbędnych do sprawdzenia autentyczności wiadomości.

\subsection*{Szczegółowy opis działania systemu}

\noindent
Podobnie jak w przypadku systemu Boneha-Franklina,
w skład systemu wchodzą cztery algorytmy:
tworzenie parametrów systemu, tworzenie klucza prywatnego,
podpisywanie wiadomości i weryfikowanie podpisu.

\begin{algorithm}[Tworzenie parametrów systemu]
Następujący algorytm wybiera i przekazuje jako wynik
parametry instancji systemu Yi
i jej klucz główny.

\begin{codebox}
\Procname{$\proc{YI-Create-Instance}()$}
\li
Wybieramy liczby pierwsze $p$ i $q$ takie,
że $p+1 = 12q$.
\li
Ustalamy, że $\G_1$ oznacza podgrupę rzędu $q$ krzywej $E_{0,1}(\F(p))$.
\li
Ustalamy, że $\G_2$ oznacza grupę pierwiastków $q$-tego stopnia w ciele $\GF(p^2)$.
\li
Ustalamy, że odwzorowanie dwuliniowe $b\colon \G_1 \times \G_1 \to \G_2$
jest określone wzorem \ref{modified_weil_pairing_eqn}.
\li
Wybieramy liczbę naturalną $m$.
\li
Ustalamy, że przestrzenią wiadomości $\mathcal{M}$
jest zbiór $\{0, 1\}^m$.
\li
Ustalamy, że przestrzenią podpisów $\mathcal{S}$
jest zbiór $\F(p) \times \F(p)$.
\li
Ustalamy, że przstrzenią identyfikatorów użytkowników $\mathcal{I}$
jest zbiór $\{0, 1\}^\star$.
\li
Wybieramy funkcje haszujące
$H_1\colon \mathcal{I} \times \Z \to \F(p)$
i $H_2\colon \mathcal{M} \times \G_1^\star \to \Z$.
\li
Wybieramy generator $P$ grupy $\G_1$.
\li
Wybieramy element $s$ z grupy $(\Z/q\Z)^\star$.
\li
Obliczamy wartość $Q = sP$.
\li
Parametry systemu to krotka
$\left\langle p, q, \G_1, \G_2, b, m, H_1, H_2, P, Q \right\rangle$.
\li
Klucz główny to element $s$.
\end{codebox}
\end{algorithm}

\begin{algorithm}[Tworzenie klucza prywatnego]
Dany jest identyfikator podpisującego $\const{id}$.
Następujący algorytm na podstawie wartości $\const{id}$
oraz klucza głównego $s$ instancji systemu Yi i jej parametrów
oblicza i przekazuje jako wynik
klucz prywatny odpowiadający identyfikatorowi $\const{id}$.

\begin{codebox}
\Procname{$\proc{YI-Create-Private-Key}(
    \const{id},
    s,
    \left\langle p, q, \G_1, \G_2, b, m, H_1, H_2, P, Q \right\rangle
)$}
\li
Przyjmujemy $k = 0$.
\li
Obliczamy $a = H_1(\const{id}, k)^3 + 1$.
\li
Jeżeli $a^{\frac{p-1}{2}} \neq 1$, to powiększamy wartość $k$ o $1$ i wracamy do kroku 2.
\li
Obliczamy $b = a^{\frac{p+1}{4}}$.
\li
Jeżeli $b > -b\ (\text{mod}\ p)$, to zmieniamy znak wartości $b$.
\li
Obliczamy $U = 12(a, b)$.
\li
Obliczamy $S = sU$.
\li
Klucz prywatny to punkt $S$.
\end{codebox}
\end{algorithm}

\begin{algorithm}[Podpisywanie wiadomości]
Dana jest wiadomość $M$
i klucz prywatny podpisującego $S$.
Następujący algorytm na podstawie wartości $M$ i $S$
oraz parametrów instancji systemu Yi
oblicza i przekazuje jako wynik
podpis wiadomości $M$ odpowiadający kluczowi $S$.

\begin{codebox}
\Procname{$\proc{YI-Sign}(
    M,
    S,
    \left\langle p, q, \G_1, \G_2, b, m, H_1, H_2, P, Q \right\rangle
)$}
\li
Wybieramy element $r$ z grupy $(\Z/q\Z)^\star$.
\li
Obliczamy $R = rP$.
\li
Jeżeli $y(R) \geq -y(R)\ (\text{mod}\ p)$,
to obliczamy $T = rQ + H_2(M, R)S$.
\li
W przeciwnym razie obliczamy $T = -rQ + H_2(M, -R)S$.
\li
Podpis to para $\left\langle x(R), x(T) \right\rangle$.
\end{codebox}
\end{algorithm}

\begin{algorithm}[Weryfikowanie podpisu]
Dana jest wiadomość $M$,
podpis $V$ w postaci $V = \left\langle a, c \right\rangle$
i identyfikator podpisującego $\const{id}$.
Następujący algorytm na podstawie wartości $M$, $V$ i $\const{id}$
oraz parametrów instancji systemu Yi
stwierdza, czy podpis $V$ jest autentyczny,
tzn. został utworzony dla wiadomości $M$
przez podpisującego dysponującego kluczem prywatnym
odpowiadającym identyfikatorowi $\const{id}$.

\begin{codebox}
\Procname{$\proc{YI-Verify}(
    M,
    \left\langle a, c \right\rangle,
    \const{id},
    \left\langle p, q, \G_1, \G_2, b, m, H_1, H_2, P, Q \right\rangle
)$}
\li
Obliczamy $b = (a^3 + 1)^{\frac{p+1}{4}}$.
\li
Obliczamy $d = (c^3 + 1)^{\frac{p+1}{4}}$.
\li
Jeżeli $b \geq -b\ (\text{mod}\ p)$, to przyjmujemy $R' = (a, b)$.
\li
W przeciwnym razie przyjmujemy $R' = (a, -b)$.
\li
Przyjmujemy $T' = (c, d)$.
\li
Obliczamy wartość $U$ tak samo jak w algorytmie tworzenia klucza prywatnego.
\li
Obliczamy $u = b(T', P)$.
\li
Obliczamy $v = b(R' + H_2(M, R')U, Q)$.
\li
Jeżeli $u = v$ lub $u = v^{-1}$, to podpis jest autentyczny.
\end{codebox}
\end{algorithm}

\noindent
Zauważmy, że system Yi nie jest aż tak ogólny, jak system Boneha i Franklina --
jest oparty na superosobliwej krzywej eliptycznej,
nie zaś na dowolnej grupie.
Ponadto, w zaproponowanej przez Boneha i Franklina
konretnej realizacji systemu
liczba $q$ mogła być dowolnym dzielnikiem liczby $p+1$.
W systemie Yi dodatkowo musi być spełniony warunek $12q = p+1$.
Obostrzenia te wprowadzone są po to,
żeby można było łatwo pierwiastkować w ciele $\F(p)$.

\begin{remark}
Jeżeli $p \equiv 3\ (\text{mod}\ 4)$,
to element $a$ z grupy $(\Z/p\Z)^\star$
jest resztą kwadratową wtedy i tylko wtedy,
gdy $a^{\frac{p-1}{2}} \equiv 1\ (\text{mod}\ p)$.
Pondato, jeżeli element $a$ jest resztą kwadratową,
to wartość $a^{\frac{p+1}{4}}$ jest jego pierwiastkiem.
\end{remark}

\begin{remark}
Zmienna $k$ w procedurze \proc{YI-Create-Private-Key}
jest powiększana tak długo,
aż wartość $H_1(\const{id}, k)^3 + 1$ będzie resztą kwadratową.
Prawdopodobieństwo zajścia tej sytuacji w jednym kroku wynosi $\frac{1}{2}$,
zatem algorytm generowania klucza prywatnego
prawie na pewno kończy się sukcesem.
\end{remark}

\noindent
Sprawdźmy teraz, że weryfikacja autentycznego podpisu
faktycznie kończy się stwierdzeniem jego autentyczności.

\begin{theorem}
Dana jest wiadomość $M$, identyfikator nadawcy $\const{id}$,
odpowiadający mu klucz prywatny $S$
i parametry $\mathcal{P}$ instancji systemu Yi.
Wówczas zachodzi następująca zależność:

\begin{equation*}
\proc{YI-Verify}(
    M,
    \proc{YI-Sign}(
        M,
        S,
        \mathcal{P}
    ),
    \const{id},
    \mathcal{P}
) = \const{true}
\end{equation*}
\end{theorem}

\begin{proof}
Zauważmy, że punkt $T'$ obliczany w procedurze $\proc{YI-Verify}$
jest równy punktowi $T$ (obliczanemu w procedurze $\proc{YI-Sign}$
lub punktowi $-T$.
Podobnie jest z punktem $R'$.
Rozważmy dwa przypadki.

\begin{enumerate}
\item
Jeżeli $y(R) \geq -y(R)\ (\text{mod}\ p)$, to $R' = R$ i $T' = \pm T$.
Wówczas:

\begin{eqnarray*}
u
& = & b(T', P)
\\
& = & b(\pm T', P)
\\
& = & b(T, P)^{\pm 1}
\\
& = & b(rQ + H_2(M, R)S, P)^{\pm 1}
\\
& = & b(rsP + sH_2(M, R)U, P)^{\pm 1}
\\
& = & b(rP + H_2(M, R)U, P)^{\pm s}
\\
& = & b(R + H_2(M, R)U, Q)^{\pm 1}
\\
& = & b(R' + H_2(M, R')U, Q)^{\pm 1}
\\
& = & v^{\pm 1}
\end{eqnarray*}

\item
Jeżeli $y(R) < -y(R)\ (\text{mod}\ p)$, to $R' = -R$ i $T' = \pm T$.
Wówczas:

\begin{eqnarray*}
u
& = & b(T', P)
\\
& = & b(\pm T', P)
\\
& = & b(T, P)^{\pm 1}
\\
& = & b(-rQ + H_2(M, -R)S, P)^{\pm 1}
\\
& = & b(-rsP + sH_2(M, -R)U, P)^{\pm 1}
\\
& = & b(-rP + H_2(M, -R)U, P)^{\pm s}
\\
& = & b(-R + H_2(M, -R)U, Q)^{\pm 1}
\\
& = & b(R' + H_2(M, R')U, Q)^{\pm 1}
\\
& = & v^{\pm 1}
\end{eqnarray*}
\end{enumerate}
\end{proof}

\noindent
We wspomnianej wcześniej pracy Yi znajduje się analiza bezpieczeństwa
opisanego systemu.

