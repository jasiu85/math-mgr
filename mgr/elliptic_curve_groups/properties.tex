\section{Własności}

\noindent
Przytoczymy teraz bez dowodu kilka najważniejszych własności
grup na krzywych eliptycznych.

\noindent
Przede wszystkim, ze względu na charakter wzorów występujących
w definicji działania grupowego,
ciało, nad którym krzywe rozpatrujemy, nie musi być algebraicznie domknięte.

\begin{fact}
Dana jest krzywa eliptyczna $E$ nad ciałem $\fieldL$
o parametrach z ciała $\K \subset \fieldL$.
Wówczas suma dwóch punktów $\K$-wymiernych na krzywej $E$
jest punktem $\K$-wymiernym.
\end{fact}

\noindent
Jest jasne, że w przypadku ciał algebraicznie domkniętych krzywa eliptyczna
ma nieskończenie wiele elementów.
W przypadku ciał skończonych sytuacja wygląda zupełnie inaczej.

\begin{theorem}[Hasse]
Rząd grupy na krzywej eliptycznej $E$
nad ciałem skończonym $\GF(q)$,
spełnia następującą nierówność:
\begin{equation}
q + 1 - 2\sqrt{q} \leq \abs{E} \leq q + 1 + 2\sqrt{q}
\end{equation}
\end{theorem}

\noindent
Dowód tego twierdzenia można znaleźć np. w pracy \cite{ecintro1},
zaś intuicyjne wyjaśnienie jest następujące.
Wielomian charakterystyczny $\kappa$ rzadko przyjmuje tę samą wartość
więcej niż raz, zatem możemy uznać, że jest on ,,prawie''
permutacją zbioru $\GF(q)$.
Stąd dla mniej-więcej połowy elementów $a$ z ciała $\GF(q)$
można z wartości $\kappa(a)$ wyciągnąć pierwiastek kwadratowy.
Zatem krzywa eliptyczna $E(\GF(q))$ ma około $2\frac{q}{2}$ punktów skończonych.
Doliczając punkt w nieskończoności otrzymujemy około $q + 1$ punktów.

\noindent
Choć nie ma jawnego wzoru
na rząd grupy na krzywej eliptycznej nad ciałem skończonym,
następujące twierdzenie,
którego dowód można odnaleźć w książce \cite{tsfasmanvladutnogin},
charakteryzuje ogólną strukturę takiej grupy.

\begin{theorem}\label{finite_curve_structure_theorem}
Grupa na krzywej eliptycznej $E$ nad ciałem $\GF(q)$
jest izomorficzna z grupą $(\Z/m\Z) \times (\Z/n\Z)$,
przy czym $n \mid \gcd(m, q-1)$.
\end{theorem}

\noindent
Na koniec wspomnijmy o algorytmie,
który pozwala dosyć efektywnie policzyć
rząd grupy na krzywej nad ciałem skończonym.
Został on opublikowany po raz pierwszy w pracy \cite{schoof}.

\begin{theorem}[Schoof]
Istnieje algorytm, który pozwala obliczyć rząd grupy
na krzywej eliptycznej $E$ nad ciałem $\GF(q)$
za pomocą $O(\log^8 q)$ operacji na bitach.
\end{theorem}
